%% Generated by Sphinx.
\def\sphinxdocclass{report}
\documentclass[letterpaper,10pt,english]{sphinxmanual}
\ifdefined\pdfpxdimen
   \let\sphinxpxdimen\pdfpxdimen\else\newdimen\sphinxpxdimen
\fi \sphinxpxdimen=.75bp\relax
\ifdefined\pdfimageresolution
    \pdfimageresolution= \numexpr \dimexpr1in\relax/\sphinxpxdimen\relax
\fi
%% let collapsible pdf bookmarks panel have high depth per default
\PassOptionsToPackage{bookmarksdepth=5}{hyperref}

\PassOptionsToPackage{booktabs}{sphinx}
\PassOptionsToPackage{colorrows}{sphinx}

\PassOptionsToPackage{warn}{textcomp}
\usepackage[utf8]{inputenc}
\ifdefined\DeclareUnicodeCharacter
% support both utf8 and utf8x syntaxes
  \ifdefined\DeclareUnicodeCharacterAsOptional
    \def\sphinxDUC#1{\DeclareUnicodeCharacter{"#1}}
  \else
    \let\sphinxDUC\DeclareUnicodeCharacter
  \fi
  \sphinxDUC{00A0}{\nobreakspace}
  \sphinxDUC{2500}{\sphinxunichar{2500}}
  \sphinxDUC{2502}{\sphinxunichar{2502}}
  \sphinxDUC{2514}{\sphinxunichar{2514}}
  \sphinxDUC{251C}{\sphinxunichar{251C}}
  \sphinxDUC{2572}{\textbackslash}
\fi
\usepackage{cmap}
\usepackage[T1]{fontenc}
\usepackage{amsmath,amssymb,amstext}
\usepackage{babel}



\usepackage{tgtermes}
\usepackage{tgheros}
\renewcommand{\ttdefault}{txtt}



\usepackage[Bjarne]{fncychap}
\usepackage{sphinx}

\fvset{fontsize=auto}
\usepackage{geometry}


% Include hyperref last.
\usepackage{hyperref}
% Fix anchor placement for figures with captions.
\usepackage{hypcap}% it must be loaded after hyperref.
% Set up styles of URL: it should be placed after hyperref.
\urlstyle{same}


\usepackage{sphinxmessages}




\title{DLMDWPMP01}
\date{Jan 12, 2023}
\release{1.2.0}
\author{Georg Grunsky}
\newcommand{\sphinxlogo}{\vbox{}}
\renewcommand{\releasename}{Release}
\makeindex
\begin{document}

\ifdefined\shorthandoff
  \ifnum\catcode`\=\string=\active\shorthandoff{=}\fi
  \ifnum\catcode`\"=\active\shorthandoff{"}\fi
\fi

\pagestyle{empty}
\sphinxmaketitle
\pagestyle{plain}
\sphinxtableofcontents
\pagestyle{normal}
\phantomsection\label{\detokenize{index::doc}}


\sphinxAtStartPar
The \sphinxstylestrong{main effort} of this package is to gain basic python programming skills
and therefore solve a simple classification task. This package is designed to
fulfill the given task and complete the IU Python Module ‘DLMDWPMP01’.

\sphinxAtStartPar
The \sphinxstylestrong{task} was to:

\sphinxAtStartPar
Evaluate ideal functions for a set of training data (1) and assign
values of a test\sphinxhyphen{}dataset to those ideal functions (2)

\sphinxAtStartPar
Used criteria for evaluation:

\sphinxAtStartPar
(1) to match training data and ideal functions:
minimum SummedSquaredError (SSE)

\sphinxAtStartPar
(2) to match ideal functions and test data:
precalculated SSE (1) * SquareRoot(2)

\sphinxAtStartPar
After \sphinxstylestrong{installing} the package by typing ‘pip install \sphinxhyphen{}e .’ in your console, the
program ‘functionfinder’ can be called from the console, using the CLI\sphinxhyphen{}command \sphinxstylestrong{‘ff’}.

\sphinxAtStartPar
This documentation was automatically generated by Sphinx, using the docstrings of following modules:


\begin{savenotes}\sphinxattablestart
\sphinxthistablewithglobalstyle
\sphinxthistablewithnovlinesstyle
\centering
\begin{tabulary}{\linewidth}[t]{\X{1}{2}\X{1}{2}}
\sphinxtoprule
\sphinxtableatstartofbodyhook
\sphinxAtStartPar
{\hyperref[\detokenize{_autosummary/functionfinder:module-functionfinder}]{\sphinxcrossref{\sphinxcode{\sphinxupquote{functionfinder}}}}}
&
\sphinxAtStartPar

\\
\sphinxhline
\sphinxAtStartPar
{\hyperref[\detokenize{_autosummary/tests:module-tests}]{\sphinxcrossref{\sphinxcode{\sphinxupquote{tests}}}}}
&
\sphinxAtStartPar

\\
\sphinxbottomrule
\end{tabulary}
\sphinxtableafterendhook\par
\sphinxattableend\end{savenotes}

\sphinxstepscope


\chapter{functionfinder}
\label{\detokenize{_autosummary/functionfinder:module-functionfinder}}\label{\detokenize{_autosummary/functionfinder:functionfinder}}\label{\detokenize{_autosummary/functionfinder::doc}}\index{module@\spxentry{module}!functionfinder@\spxentry{functionfinder}}\index{functionfinder@\spxentry{functionfinder}!module@\spxentry{module}}\subsubsection*{Modules}


\begin{savenotes}\sphinxattablestart
\sphinxthistablewithglobalstyle
\sphinxthistablewithnovlinesstyle
\centering
\begin{tabulary}{\linewidth}[t]{\X{1}{2}\X{1}{2}}
\sphinxtoprule
\sphinxtableatstartofbodyhook
\sphinxAtStartPar
{\hyperref[\detokenize{_autosummary/functionfinder.classes:module-functionfinder.classes}]{\sphinxcrossref{\sphinxcode{\sphinxupquote{functionfinder.classes}}}}}
&
\sphinxAtStartPar
Definition of object classes.
\\
\sphinxhline
\sphinxAtStartPar
{\hyperref[\detokenize{_autosummary/functionfinder.config:module-functionfinder.config}]{\sphinxcrossref{\sphinxcode{\sphinxupquote{functionfinder.config}}}}}
&
\sphinxAtStartPar
User configurable parameters.
\\
\sphinxhline
\sphinxAtStartPar
{\hyperref[\detokenize{_autosummary/functionfinder.datafunctions:module-functionfinder.datafunctions}]{\sphinxcrossref{\sphinxcode{\sphinxupquote{functionfinder.datafunctions}}}}}
&
\sphinxAtStartPar
Functions used for data handling.
\\
\sphinxhline
\sphinxAtStartPar
{\hyperref[\detokenize{_autosummary/functionfinder.exceptions:module-functionfinder.exceptions}]{\sphinxcrossref{\sphinxcode{\sphinxupquote{functionfinder.exceptions}}}}}
&
\sphinxAtStartPar
Userdefined exceptions.
\\
\sphinxhline
\sphinxAtStartPar
{\hyperref[\detokenize{_autosummary/functionfinder.ffrunner:module-functionfinder.ffrunner}]{\sphinxcrossref{\sphinxcode{\sphinxupquote{functionfinder.ffrunner}}}}}
&
\sphinxAtStartPar
Main program.
\\
\sphinxhline
\sphinxAtStartPar
{\hyperref[\detokenize{_autosummary/functionfinder.log:module-functionfinder.log}]{\sphinxcrossref{\sphinxcode{\sphinxupquote{functionfinder.log}}}}}
&
\sphinxAtStartPar
Definition of logging parameters.
\\
\sphinxhline
\sphinxAtStartPar
{\hyperref[\detokenize{_autosummary/functionfinder.setuplog:module-functionfinder.setuplog}]{\sphinxcrossref{\sphinxcode{\sphinxupquote{functionfinder.setuplog}}}}}
&
\sphinxAtStartPar
Definition of logging parameters for the installation process.
\\
\sphinxbottomrule
\end{tabulary}
\sphinxtableafterendhook\par
\sphinxattableend\end{savenotes}

\sphinxstepscope


\section{functionfinder.classes}
\label{\detokenize{_autosummary/functionfinder.classes:module-functionfinder.classes}}\label{\detokenize{_autosummary/functionfinder.classes:functionfinder-classes}}\label{\detokenize{_autosummary/functionfinder.classes::doc}}\index{module@\spxentry{module}!functionfinder.classes@\spxentry{functionfinder.classes}}\index{functionfinder.classes@\spxentry{functionfinder.classes}!module@\spxentry{module}}
\sphinxAtStartPar
Definition of object classes.

\sphinxAtStartPar
This module defines three classes used to store data information. The latter
two are subclasses of the first class projectdata.
\subsubsection*{Classes}


\begin{savenotes}\sphinxattablestart
\sphinxthistablewithglobalstyle
\sphinxthistablewithnovlinesstyle
\centering
\begin{tabulary}{\linewidth}[t]{\X{1}{2}\X{1}{2}}
\sphinxtoprule
\sphinxtableatstartofbodyhook
\sphinxAtStartPar
{\hyperref[\detokenize{_autosummary/functionfinder.classes.idealdata:functionfinder.classes.idealdata}]{\sphinxcrossref{\sphinxcode{\sphinxupquote{idealdata}}}}}({[}dataname, ylabel, xlabel, ...{]})
&
\sphinxAtStartPar
Subclass of class \textquotesingle{}projectdata\textquotesingle{} to store and handle data operations.
\\
\sphinxhline
\sphinxAtStartPar
{\hyperref[\detokenize{_autosummary/functionfinder.classes.projectdata:functionfinder.classes.projectdata}]{\sphinxcrossref{\sphinxcode{\sphinxupquote{projectdata}}}}}({[}dataname, ylabel, xlabel, ...{]})
&
\sphinxAtStartPar
Storage defined to handle necessary data operations.
\\
\sphinxhline
\sphinxAtStartPar
{\hyperref[\detokenize{_autosummary/functionfinder.classes.testdata:functionfinder.classes.testdata}]{\sphinxcrossref{\sphinxcode{\sphinxupquote{testdata}}}}}({[}dataname, ylabel, xlabel, ...{]})
&
\sphinxAtStartPar
Subclass of class \textquotesingle{}projectdata\textquotesingle{} to store and handle data operations.
\\
\sphinxbottomrule
\end{tabulary}
\sphinxtableafterendhook\par
\sphinxattableend\end{savenotes}

\sphinxstepscope


\subsection{functionfinder.classes.idealdata}
\label{\detokenize{_autosummary/functionfinder.classes.idealdata:functionfinder-classes-idealdata}}\label{\detokenize{_autosummary/functionfinder.classes.idealdata::doc}}\index{idealdata (class in functionfinder.classes)@\spxentry{idealdata}\spxextra{class in functionfinder.classes}}

\begin{fulllineitems}
\phantomsection\label{\detokenize{_autosummary/functionfinder.classes.idealdata:functionfinder.classes.idealdata}}
\pysigstartsignatures
\pysiglinewithargsret{\sphinxbfcode{\sphinxupquote{class\DUrole{w}{  }}}\sphinxcode{\sphinxupquote{functionfinder.classes.}}\sphinxbfcode{\sphinxupquote{idealdata}}}{\emph{\DUrole{n}{dataname}\DUrole{o}{=}\DUrole{default_value}{\textquotesingle{}train\textquotesingle{}}}, \emph{\DUrole{n}{ylabel}\DUrole{o}{=}\DUrole{default_value}{\textquotesingle{}Y\textquotesingle{}}}, \emph{\DUrole{n}{xlabel}\DUrole{o}{=}\DUrole{default_value}{\textquotesingle{}X\textquotesingle{}}}, \emph{\DUrole{n}{plottitle}\DUrole{o}{=}\DUrole{default_value}{\textquotesingle{}no title\textquotesingle{}}}, \emph{\DUrole{n}{plotfile}\DUrole{o}{=}\DUrole{default_value}{\textquotesingle{}testplot.png\textquotesingle{}}}}{}
\pysigstopsignatures
\sphinxAtStartPar
Bases: {\hyperref[\detokenize{_autosummary/functionfinder.classes.projectdata:functionfinder.classes.projectdata}]{\sphinxcrossref{\sphinxcode{\sphinxupquote{projectdata}}}}}

\sphinxAtStartPar
Subclass of class ‘projectdata’ to store and handle data operations.

\sphinxAtStartPar
Contains recurring methods for the ideal functions dataset.


\subsubsection{Attributes}
\label{\detokenize{_autosummary/functionfinder.classes.idealdata:attributes}}
\sphinxAtStartPar
As in class ‘projectdata’ plus additionally:
matched : dictionary
\begin{quote}

\sphinxAtStartPar
Results of matching training data to ideal functions
\end{quote}


\subsubsection{Methods}
\label{\detokenize{_autosummary/functionfinder.classes.idealdata:methods}}
\sphinxAtStartPar
As in class ‘projectdata’ plus additionally:
matched\_functions(match\_result=dict())
\begin{quote}

\sphinxAtStartPar
Assign dictionary to ‘matched’ attribute
\end{quote}
\index{\_\_init\_\_() (functionfinder.classes.idealdata method)@\spxentry{\_\_init\_\_()}\spxextra{functionfinder.classes.idealdata method}}

\begin{fulllineitems}
\phantomsection\label{\detokenize{_autosummary/functionfinder.classes.idealdata:functionfinder.classes.idealdata.__init__}}
\pysigstartsignatures
\pysiglinewithargsret{\sphinxbfcode{\sphinxupquote{\_\_init\_\_}}}{\emph{\DUrole{n}{dataname}\DUrole{o}{=}\DUrole{default_value}{\textquotesingle{}train\textquotesingle{}}}, \emph{\DUrole{n}{ylabel}\DUrole{o}{=}\DUrole{default_value}{\textquotesingle{}Y\textquotesingle{}}}, \emph{\DUrole{n}{xlabel}\DUrole{o}{=}\DUrole{default_value}{\textquotesingle{}X\textquotesingle{}}}, \emph{\DUrole{n}{plottitle}\DUrole{o}{=}\DUrole{default_value}{\textquotesingle{}no title\textquotesingle{}}}, \emph{\DUrole{n}{plotfile}\DUrole{o}{=}\DUrole{default_value}{\textquotesingle{}testplot.png\textquotesingle{}}}}{}
\pysigstopsignatures
\sphinxAtStartPar
Construct attributes for this class.


\paragraph{Parameters}
\label{\detokenize{_autosummary/functionfinder.classes.idealdata:parameters}}\begin{description}
\sphinxlineitem{dataname}{[}string{]}
\sphinxAtStartPar
Provide SQLite table name to read data from.
The default is “train”.

\sphinxlineitem{ylabel}{[}string, optional{]}
\sphinxAtStartPar
Provide y\sphinxhyphen{}axis title for plotting. The default is “Y”.

\sphinxlineitem{xlabel}{[}string, optional{]}
\sphinxAtStartPar
Provide x\sphinxhyphen{}axis title for plotting. The default is “X”.

\sphinxlineitem{plottitle}{[}string{]}
\sphinxAtStartPar
Provide plot\sphinxhyphen{}title for contained data. The default is “no title”.

\sphinxlineitem{plotfile}{[}string{]}
\sphinxAtStartPar
Provide name of PNG to save plot. The default is “testplot.png”.

\end{description}

\end{fulllineitems}

\subsubsection*{Methods}


\begin{savenotes}\sphinxattablestart
\sphinxthistablewithglobalstyle
\sphinxthistablewithnovlinesstyle
\centering
\begin{tabulary}{\linewidth}[t]{\X{1}{2}\X{1}{2}}
\sphinxtoprule
\sphinxtableatstartofbodyhook
\sphinxAtStartPar
{\hyperref[\detokenize{_autosummary/functionfinder.classes.idealdata:functionfinder.classes.idealdata.__init__}]{\sphinxcrossref{\sphinxcode{\sphinxupquote{\_\_init\_\_}}}}}({[}dataname, ylabel, xlabel, ...{]})
&
\sphinxAtStartPar
Construct attributes for this class.
\\
\sphinxhline
\sphinxAtStartPar
{\hyperref[\detokenize{_autosummary/functionfinder.classes.idealdata:functionfinder.classes.idealdata.draw}]{\sphinxcrossref{\sphinxcode{\sphinxupquote{draw}}}}}()
&
\sphinxAtStartPar
Plot data and save to png file.
\\
\sphinxhline
\sphinxAtStartPar
{\hyperref[\detokenize{_autosummary/functionfinder.classes.idealdata:functionfinder.classes.idealdata.getdata}]{\sphinxcrossref{\sphinxcode{\sphinxupquote{getdata}}}}}()
&
\sphinxAtStartPar
Query SQLite database to read and store data in \textquotesingle{}data\textquotesingle{} attribute.
\\
\sphinxhline
\sphinxAtStartPar
{\hyperref[\detokenize{_autosummary/functionfinder.classes.idealdata:functionfinder.classes.idealdata.matched_functions}]{\sphinxcrossref{\sphinxcode{\sphinxupquote{matched\_functions}}}}}({[}match\_result{]})
&
\sphinxAtStartPar
Assign dictionary to \textquotesingle{}matched\textquotesingle{} attribute.
\\
\sphinxbottomrule
\end{tabulary}
\sphinxtableafterendhook\par
\sphinxattableend\end{savenotes}
\index{draw() (functionfinder.classes.idealdata method)@\spxentry{draw()}\spxextra{functionfinder.classes.idealdata method}}

\begin{fulllineitems}
\phantomsection\label{\detokenize{_autosummary/functionfinder.classes.idealdata:functionfinder.classes.idealdata.draw}}
\pysigstartsignatures
\pysiglinewithargsret{\sphinxbfcode{\sphinxupquote{draw}}}{}{}
\pysigstopsignatures
\sphinxAtStartPar
Plot data and save to png file.

\end{fulllineitems}

\index{getdata() (functionfinder.classes.idealdata method)@\spxentry{getdata()}\spxextra{functionfinder.classes.idealdata method}}

\begin{fulllineitems}
\phantomsection\label{\detokenize{_autosummary/functionfinder.classes.idealdata:functionfinder.classes.idealdata.getdata}}
\pysigstartsignatures
\pysiglinewithargsret{\sphinxbfcode{\sphinxupquote{getdata}}}{}{}
\pysigstopsignatures
\sphinxAtStartPar
Query SQLite database to read and store data in ‘data’ attribute.

\end{fulllineitems}

\index{matched\_functions() (functionfinder.classes.idealdata method)@\spxentry{matched\_functions()}\spxextra{functionfinder.classes.idealdata method}}

\begin{fulllineitems}
\phantomsection\label{\detokenize{_autosummary/functionfinder.classes.idealdata:functionfinder.classes.idealdata.matched_functions}}
\pysigstartsignatures
\pysiglinewithargsret{\sphinxbfcode{\sphinxupquote{matched\_functions}}}{\emph{\DUrole{n}{match\_result}\DUrole{o}{=}\DUrole{default_value}{\{\}}}}{}
\pysigstopsignatures
\sphinxAtStartPar
Assign dictionary to ‘matched’ attribute.


\paragraph{Parameters}
\label{\detokenize{_autosummary/functionfinder.classes.idealdata:id1}}\begin{description}
\sphinxlineitem{match\_result}{[}dictionary, optional{]}
\sphinxAtStartPar
Provide dictionary to store in ‘matched’ attribute.
The default is dict().

\end{description}

\end{fulllineitems}


\end{fulllineitems}


\sphinxstepscope


\subsection{functionfinder.classes.projectdata}
\label{\detokenize{_autosummary/functionfinder.classes.projectdata:functionfinder-classes-projectdata}}\label{\detokenize{_autosummary/functionfinder.classes.projectdata::doc}}\index{projectdata (class in functionfinder.classes)@\spxentry{projectdata}\spxextra{class in functionfinder.classes}}

\begin{fulllineitems}
\phantomsection\label{\detokenize{_autosummary/functionfinder.classes.projectdata:functionfinder.classes.projectdata}}
\pysigstartsignatures
\pysiglinewithargsret{\sphinxbfcode{\sphinxupquote{class\DUrole{w}{  }}}\sphinxcode{\sphinxupquote{functionfinder.classes.}}\sphinxbfcode{\sphinxupquote{projectdata}}}{\emph{\DUrole{n}{dataname}\DUrole{o}{=}\DUrole{default_value}{\textquotesingle{}train\textquotesingle{}}}, \emph{\DUrole{n}{ylabel}\DUrole{o}{=}\DUrole{default_value}{\textquotesingle{}Y\textquotesingle{}}}, \emph{\DUrole{n}{xlabel}\DUrole{o}{=}\DUrole{default_value}{\textquotesingle{}X\textquotesingle{}}}, \emph{\DUrole{n}{plottitle}\DUrole{o}{=}\DUrole{default_value}{\textquotesingle{}no title\textquotesingle{}}}, \emph{\DUrole{n}{plotfile}\DUrole{o}{=}\DUrole{default_value}{\textquotesingle{}testplot.png\textquotesingle{}}}}{}
\pysigstopsignatures
\sphinxAtStartPar
Bases: \sphinxcode{\sphinxupquote{object}}

\sphinxAtStartPar
Storage defined to handle necessary data operations.

\sphinxAtStartPar
Contains recurring methods for the given types of datasets.


\subsubsection{Attributes}
\label{\detokenize{_autosummary/functionfinder.classes.projectdata:attributes}}\begin{description}
\sphinxlineitem{\_table}{[}string{]}
\sphinxAtStartPar
Refered table name in SQLite Database

\sphinxlineitem{\_style}{[}string{]}
\sphinxAtStartPar
Matplotlib plotting style

\sphinxlineitem{ylabel}{[}string{]}
\sphinxAtStartPar
Label of y\sphinxhyphen{}axis for plotting

\sphinxlineitem{xlabel}{[}string{]}
\sphinxAtStartPar
Label of x\sphinxhyphen{}axis for plotting

\sphinxlineitem{title}{[}string{]}
\sphinxAtStartPar
Title of plot

\sphinxlineitem{fname}{[}string{]}
\sphinxAtStartPar
Name of png file to save plots

\sphinxlineitem{\_dbcon}{[}string{]}
\sphinxAtStartPar
Name and location of SQLite Database

\sphinxlineitem{data}{[}pandas.DataFrame{]}
\sphinxAtStartPar
Data read in from SQLite Database

\end{description}


\subsubsection{Methods}
\label{\detokenize{_autosummary/functionfinder.classes.projectdata:methods}}\begin{description}
\sphinxlineitem{\_\_init\_\_(dataname=”train”, ylabel=”Y”, xlabel=”X”, plottitle = “no title”,}\begin{quote}

\sphinxAtStartPar
plotfile=”testplot.png”)
\end{quote}

\sphinxAtStartPar
Constructor method to specify relevant attributes.

\sphinxlineitem{getdata()}
\sphinxAtStartPar
Query SQLite database to read and store data in ‘data’ attribute.

\sphinxlineitem{draw()}
\sphinxAtStartPar
Plot data and save to png file.

\end{description}
\index{\_\_init\_\_() (functionfinder.classes.projectdata method)@\spxentry{\_\_init\_\_()}\spxextra{functionfinder.classes.projectdata method}}

\begin{fulllineitems}
\phantomsection\label{\detokenize{_autosummary/functionfinder.classes.projectdata:functionfinder.classes.projectdata.__init__}}
\pysigstartsignatures
\pysiglinewithargsret{\sphinxbfcode{\sphinxupquote{\_\_init\_\_}}}{\emph{\DUrole{n}{dataname}\DUrole{o}{=}\DUrole{default_value}{\textquotesingle{}train\textquotesingle{}}}, \emph{\DUrole{n}{ylabel}\DUrole{o}{=}\DUrole{default_value}{\textquotesingle{}Y\textquotesingle{}}}, \emph{\DUrole{n}{xlabel}\DUrole{o}{=}\DUrole{default_value}{\textquotesingle{}X\textquotesingle{}}}, \emph{\DUrole{n}{plottitle}\DUrole{o}{=}\DUrole{default_value}{\textquotesingle{}no title\textquotesingle{}}}, \emph{\DUrole{n}{plotfile}\DUrole{o}{=}\DUrole{default_value}{\textquotesingle{}testplot.png\textquotesingle{}}}}{}
\pysigstopsignatures
\sphinxAtStartPar
Construct attributes for this class.


\paragraph{Parameters}
\label{\detokenize{_autosummary/functionfinder.classes.projectdata:parameters}}\begin{description}
\sphinxlineitem{dataname}{[}string{]}
\sphinxAtStartPar
Provide SQLite table name to read data from.
The default is “train”.

\sphinxlineitem{ylabel}{[}string, optional{]}
\sphinxAtStartPar
Provide y\sphinxhyphen{}axis title for plotting. The default is “Y”.

\sphinxlineitem{xlabel}{[}string, optional{]}
\sphinxAtStartPar
Provide x\sphinxhyphen{}axis title for plotting. The default is “X”.

\sphinxlineitem{plottitle}{[}string{]}
\sphinxAtStartPar
Provide plot\sphinxhyphen{}title for contained data. The default is “no title”.

\sphinxlineitem{plotfile}{[}string{]}
\sphinxAtStartPar
Provide name of PNG to save plot. The default is “testplot.png”.

\end{description}

\end{fulllineitems}

\subsubsection*{Methods}


\begin{savenotes}\sphinxattablestart
\sphinxthistablewithglobalstyle
\sphinxthistablewithnovlinesstyle
\centering
\begin{tabulary}{\linewidth}[t]{\X{1}{2}\X{1}{2}}
\sphinxtoprule
\sphinxtableatstartofbodyhook
\sphinxAtStartPar
{\hyperref[\detokenize{_autosummary/functionfinder.classes.projectdata:functionfinder.classes.projectdata.__init__}]{\sphinxcrossref{\sphinxcode{\sphinxupquote{\_\_init\_\_}}}}}({[}dataname, ylabel, xlabel, ...{]})
&
\sphinxAtStartPar
Construct attributes for this class.
\\
\sphinxhline
\sphinxAtStartPar
{\hyperref[\detokenize{_autosummary/functionfinder.classes.projectdata:functionfinder.classes.projectdata.draw}]{\sphinxcrossref{\sphinxcode{\sphinxupquote{draw}}}}}()
&
\sphinxAtStartPar
Plot data and save to png file.
\\
\sphinxhline
\sphinxAtStartPar
{\hyperref[\detokenize{_autosummary/functionfinder.classes.projectdata:functionfinder.classes.projectdata.getdata}]{\sphinxcrossref{\sphinxcode{\sphinxupquote{getdata}}}}}()
&
\sphinxAtStartPar
Query SQLite database to read and store data in \textquotesingle{}data\textquotesingle{} attribute.
\\
\sphinxbottomrule
\end{tabulary}
\sphinxtableafterendhook\par
\sphinxattableend\end{savenotes}
\index{draw() (functionfinder.classes.projectdata method)@\spxentry{draw()}\spxextra{functionfinder.classes.projectdata method}}

\begin{fulllineitems}
\phantomsection\label{\detokenize{_autosummary/functionfinder.classes.projectdata:functionfinder.classes.projectdata.draw}}
\pysigstartsignatures
\pysiglinewithargsret{\sphinxbfcode{\sphinxupquote{draw}}}{}{}
\pysigstopsignatures
\sphinxAtStartPar
Plot data and save to png file.

\end{fulllineitems}

\index{getdata() (functionfinder.classes.projectdata method)@\spxentry{getdata()}\spxextra{functionfinder.classes.projectdata method}}

\begin{fulllineitems}
\phantomsection\label{\detokenize{_autosummary/functionfinder.classes.projectdata:functionfinder.classes.projectdata.getdata}}
\pysigstartsignatures
\pysiglinewithargsret{\sphinxbfcode{\sphinxupquote{getdata}}}{}{}
\pysigstopsignatures
\sphinxAtStartPar
Query SQLite database to read and store data in ‘data’ attribute.

\end{fulllineitems}


\end{fulllineitems}


\sphinxstepscope


\subsection{functionfinder.classes.testdata}
\label{\detokenize{_autosummary/functionfinder.classes.testdata:functionfinder-classes-testdata}}\label{\detokenize{_autosummary/functionfinder.classes.testdata::doc}}\index{testdata (class in functionfinder.classes)@\spxentry{testdata}\spxextra{class in functionfinder.classes}}

\begin{fulllineitems}
\phantomsection\label{\detokenize{_autosummary/functionfinder.classes.testdata:functionfinder.classes.testdata}}
\pysigstartsignatures
\pysiglinewithargsret{\sphinxbfcode{\sphinxupquote{class\DUrole{w}{  }}}\sphinxcode{\sphinxupquote{functionfinder.classes.}}\sphinxbfcode{\sphinxupquote{testdata}}}{\emph{\DUrole{n}{dataname}\DUrole{o}{=}\DUrole{default_value}{\textquotesingle{}train\textquotesingle{}}}, \emph{\DUrole{n}{ylabel}\DUrole{o}{=}\DUrole{default_value}{\textquotesingle{}Y\textquotesingle{}}}, \emph{\DUrole{n}{xlabel}\DUrole{o}{=}\DUrole{default_value}{\textquotesingle{}X\textquotesingle{}}}, \emph{\DUrole{n}{plottitle}\DUrole{o}{=}\DUrole{default_value}{\textquotesingle{}no title\textquotesingle{}}}, \emph{\DUrole{n}{plotfile}\DUrole{o}{=}\DUrole{default_value}{\textquotesingle{}testplot.png\textquotesingle{}}}}{}
\pysigstopsignatures
\sphinxAtStartPar
Bases: {\hyperref[\detokenize{_autosummary/functionfinder.classes.projectdata:functionfinder.classes.projectdata}]{\sphinxcrossref{\sphinxcode{\sphinxupquote{projectdata}}}}}

\sphinxAtStartPar
Subclass of class ‘projectdata’ to store and handle data operations.

\sphinxAtStartPar
Contains recurring methods for the test dataset.


\subsubsection{Attributes}
\label{\detokenize{_autosummary/functionfinder.classes.testdata:attributes}}
\sphinxAtStartPar
As in class ‘projectdata’ plus additionally:
off : pandas.DataFrame
\begin{quote}

\sphinxAtStartPar
Results of matching test data to ideal functions, which exceed the
predefined limit.
\end{quote}


\subsubsection{Methods}
\label{\detokenize{_autosummary/functionfinder.classes.testdata:methods}}
\sphinxAtStartPar
As in class ‘projectdata’ plus additionally:
off\_limit()
\begin{quote}

\sphinxAtStartPar
Query SQLite database to read and store entries, which exceeded the
predefined limit, to ‘off’ attribute.
\end{quote}
\index{\_\_init\_\_() (functionfinder.classes.testdata method)@\spxentry{\_\_init\_\_()}\spxextra{functionfinder.classes.testdata method}}

\begin{fulllineitems}
\phantomsection\label{\detokenize{_autosummary/functionfinder.classes.testdata:functionfinder.classes.testdata.__init__}}
\pysigstartsignatures
\pysiglinewithargsret{\sphinxbfcode{\sphinxupquote{\_\_init\_\_}}}{\emph{\DUrole{n}{dataname}\DUrole{o}{=}\DUrole{default_value}{\textquotesingle{}train\textquotesingle{}}}, \emph{\DUrole{n}{ylabel}\DUrole{o}{=}\DUrole{default_value}{\textquotesingle{}Y\textquotesingle{}}}, \emph{\DUrole{n}{xlabel}\DUrole{o}{=}\DUrole{default_value}{\textquotesingle{}X\textquotesingle{}}}, \emph{\DUrole{n}{plottitle}\DUrole{o}{=}\DUrole{default_value}{\textquotesingle{}no title\textquotesingle{}}}, \emph{\DUrole{n}{plotfile}\DUrole{o}{=}\DUrole{default_value}{\textquotesingle{}testplot.png\textquotesingle{}}}}{}
\pysigstopsignatures
\sphinxAtStartPar
Construct attributes for this class.


\paragraph{Parameters}
\label{\detokenize{_autosummary/functionfinder.classes.testdata:parameters}}\begin{description}
\sphinxlineitem{dataname}{[}string{]}
\sphinxAtStartPar
Provide SQLite table name to read data from.
The default is “train”.

\sphinxlineitem{ylabel}{[}string, optional{]}
\sphinxAtStartPar
Provide y\sphinxhyphen{}axis title for plotting. The default is “Y”.

\sphinxlineitem{xlabel}{[}string, optional{]}
\sphinxAtStartPar
Provide x\sphinxhyphen{}axis title for plotting. The default is “X”.

\sphinxlineitem{plottitle}{[}string{]}
\sphinxAtStartPar
Provide plot\sphinxhyphen{}title for contained data. The default is “no title”.

\sphinxlineitem{plotfile}{[}string{]}
\sphinxAtStartPar
Provide name of PNG to save plot. The default is “testplot.png”.

\end{description}

\end{fulllineitems}

\subsubsection*{Methods}


\begin{savenotes}\sphinxattablestart
\sphinxthistablewithglobalstyle
\sphinxthistablewithnovlinesstyle
\centering
\begin{tabulary}{\linewidth}[t]{\X{1}{2}\X{1}{2}}
\sphinxtoprule
\sphinxtableatstartofbodyhook
\sphinxAtStartPar
{\hyperref[\detokenize{_autosummary/functionfinder.classes.testdata:functionfinder.classes.testdata.__init__}]{\sphinxcrossref{\sphinxcode{\sphinxupquote{\_\_init\_\_}}}}}({[}dataname, ylabel, xlabel, ...{]})
&
\sphinxAtStartPar
Construct attributes for this class.
\\
\sphinxhline
\sphinxAtStartPar
{\hyperref[\detokenize{_autosummary/functionfinder.classes.testdata:functionfinder.classes.testdata.draw}]{\sphinxcrossref{\sphinxcode{\sphinxupquote{draw}}}}}()
&
\sphinxAtStartPar
Plot data and save to png file.
\\
\sphinxhline
\sphinxAtStartPar
{\hyperref[\detokenize{_autosummary/functionfinder.classes.testdata:functionfinder.classes.testdata.getdata}]{\sphinxcrossref{\sphinxcode{\sphinxupquote{getdata}}}}}()
&
\sphinxAtStartPar
Query SQLite database to read and store data in \textquotesingle{}data\textquotesingle{} attribute.
\\
\sphinxhline
\sphinxAtStartPar
{\hyperref[\detokenize{_autosummary/functionfinder.classes.testdata:functionfinder.classes.testdata.off_limit}]{\sphinxcrossref{\sphinxcode{\sphinxupquote{off\_limit}}}}}()
&
\sphinxAtStartPar
Read data, exceeding predifined deviation limit, from SQLite.
\\
\sphinxbottomrule
\end{tabulary}
\sphinxtableafterendhook\par
\sphinxattableend\end{savenotes}
\index{draw() (functionfinder.classes.testdata method)@\spxentry{draw()}\spxextra{functionfinder.classes.testdata method}}

\begin{fulllineitems}
\phantomsection\label{\detokenize{_autosummary/functionfinder.classes.testdata:functionfinder.classes.testdata.draw}}
\pysigstartsignatures
\pysiglinewithargsret{\sphinxbfcode{\sphinxupquote{draw}}}{}{}
\pysigstopsignatures
\sphinxAtStartPar
Plot data and save to png file.

\end{fulllineitems}

\index{getdata() (functionfinder.classes.testdata method)@\spxentry{getdata()}\spxextra{functionfinder.classes.testdata method}}

\begin{fulllineitems}
\phantomsection\label{\detokenize{_autosummary/functionfinder.classes.testdata:functionfinder.classes.testdata.getdata}}
\pysigstartsignatures
\pysiglinewithargsret{\sphinxbfcode{\sphinxupquote{getdata}}}{}{}
\pysigstopsignatures
\sphinxAtStartPar
Query SQLite database to read and store data in ‘data’ attribute.

\end{fulllineitems}

\index{off\_limit() (functionfinder.classes.testdata method)@\spxentry{off\_limit()}\spxextra{functionfinder.classes.testdata method}}

\begin{fulllineitems}
\phantomsection\label{\detokenize{_autosummary/functionfinder.classes.testdata:functionfinder.classes.testdata.off_limit}}
\pysigstartsignatures
\pysiglinewithargsret{\sphinxbfcode{\sphinxupquote{off\_limit}}}{}{}
\pysigstopsignatures
\sphinxAtStartPar
Read data, exceeding predifined deviation limit, from SQLite.

\end{fulllineitems}


\end{fulllineitems}


\sphinxstepscope


\section{functionfinder.config}
\label{\detokenize{_autosummary/functionfinder.config:module-functionfinder.config}}\label{\detokenize{_autosummary/functionfinder.config:functionfinder-config}}\label{\detokenize{_autosummary/functionfinder.config::doc}}\index{module@\spxentry{module}!functionfinder.config@\spxentry{functionfinder.config}}\index{functionfinder.config@\spxentry{functionfinder.config}!module@\spxentry{module}}
\sphinxAtStartPar
User configurable parameters.

\sphinxAtStartPar
This script contains definitions of configurable elements for the
functionfinder package and should be adjusted to your own needs before running
the program.
\begin{description}
\sphinxlineitem{It provides variables to configure}
\sphinxAtStartPar
output paths,
the name of the database,
logging configuration,
required files (checked at setup),
a dictionary to match datafiles to specified keys,
the functiondefinition for the errorcalculation of train and ideal data,
the factor to modify the calculated error when evaluating test data.

\end{description}


\subsection{Methods}
\label{\detokenize{_autosummary/functionfinder.config:methods}}\begin{description}
\sphinxlineitem{error\_calculation(trainvalue, idealvalue)}
\sphinxAtStartPar
Calculate error rate between to given pandas\sphinxhyphen{}series objects of same length.

\end{description}
\subsubsection*{Functions}


\begin{savenotes}\sphinxattablestart
\sphinxthistablewithglobalstyle
\sphinxthistablewithnovlinesstyle
\centering
\begin{tabulary}{\linewidth}[t]{\X{1}{2}\X{1}{2}}
\sphinxtoprule
\sphinxtableatstartofbodyhook
\sphinxAtStartPar
{\hyperref[\detokenize{_autosummary/functionfinder.config.error_calculation:functionfinder.config.error_calculation}]{\sphinxcrossref{\sphinxcode{\sphinxupquote{error\_calculation}}}}}(trainvalue, idealvalue)
&
\sphinxAtStartPar
Calculate an error\sphinxhyphen{}value between two pandas Series.
\\
\sphinxbottomrule
\end{tabulary}
\sphinxtableafterendhook\par
\sphinxattableend\end{savenotes}

\sphinxstepscope


\subsection{functionfinder.config.error\_calculation}
\label{\detokenize{_autosummary/functionfinder.config.error_calculation:functionfinder-config-error-calculation}}\label{\detokenize{_autosummary/functionfinder.config.error_calculation::doc}}\index{error\_calculation() (in module functionfinder.config)@\spxentry{error\_calculation()}\spxextra{in module functionfinder.config}}

\begin{fulllineitems}
\phantomsection\label{\detokenize{_autosummary/functionfinder.config.error_calculation:functionfinder.config.error_calculation}}
\pysigstartsignatures
\pysiglinewithargsret{\sphinxcode{\sphinxupquote{functionfinder.config.}}\sphinxbfcode{\sphinxupquote{error\_calculation}}}{\emph{\DUrole{n}{trainvalue}}, \emph{\DUrole{n}{idealvalue}}}{}
\pysigstopsignatures
\sphinxAtStartPar
Calculate an error\sphinxhyphen{}value between two pandas Series.

\sphinxAtStartPar
The method is called by datafunctions.min\_error method
and can modified to test different calculation algorithms.


\subsubsection{Parameters}
\label{\detokenize{_autosummary/functionfinder.config.error_calculation:parameters}}\begin{description}
\sphinxlineitem{trainvalue}{[}pandas Series{]}
\sphinxAtStartPar
Specific Y\sphinxhyphen{}Data column of a trainingdata DataFrame.
(has to have the same length as idealvalue!)

\sphinxlineitem{idealvalue}{[}pandas Series{]}
\sphinxAtStartPar
Specific Y\sphinxhyphen{}Data column of a idealdata DataFrame.
(has to have the same length as idealvalue!)

\end{description}


\subsubsection{Returns}
\label{\detokenize{_autosummary/functionfinder.config.error_calculation:returns}}\begin{description}
\sphinxlineitem{calcerror}{[}float{]}
\sphinxAtStartPar
Calculated error value of the provided two pandas Series

\end{description}


\subsubsection{Notes}
\label{\detokenize{_autosummary/functionfinder.config.error_calculation:notes}}
\sphinxAtStartPar
By default the provided algorithm is the summed squared error (SSE)
of the provided pandas Series.
\begin{equation*}
\begin{split}\sum_{i=1}^{n}(TrainingData_{i} - IdealFunction_{i})^2\end{split}
\end{equation*}
\end{fulllineitems}


\sphinxstepscope


\section{functionfinder.datafunctions}
\label{\detokenize{_autosummary/functionfinder.datafunctions:module-functionfinder.datafunctions}}\label{\detokenize{_autosummary/functionfinder.datafunctions:functionfinder-datafunctions}}\label{\detokenize{_autosummary/functionfinder.datafunctions::doc}}\index{module@\spxentry{module}!functionfinder.datafunctions@\spxentry{functionfinder.datafunctions}}\index{functionfinder.datafunctions@\spxentry{functionfinder.datafunctions}!module@\spxentry{module}}
\sphinxAtStartPar
Functions used for data handling.

\sphinxAtStartPar
This script contains definitions for data related functions of the
functionfinder package.


\subsection{Methods}
\label{\detokenize{_autosummary/functionfinder.datafunctions:methods}}\begin{description}
\sphinxlineitem{create\_empty\_sqlitedb(dbname)}
\sphinxAtStartPar
Create empty SQLite Database.

\sphinxlineitem{csv2sql\_directly(existing\_db, csv\_toadd, tablename)}
\sphinxAtStartPar
Import csv files into existing SQLite database using subprocess.

\sphinxlineitem{csv2sql\_pandas(existing\_db, csv\_toadd, tablename)}
\sphinxAtStartPar
Import csv files into existing SQLite database using pandas.

\sphinxlineitem{min\_error(train\_set, ideal\_df)}
\sphinxAtStartPar
Select column of a pandas DataFrame with minimal error to a Series.

\sphinxlineitem{calculate\_best\_ideal(test\_value, match\_against, index, idealdata, function)}
\sphinxAtStartPar
Select ideal function with minimal deviation to given point.

\end{description}
\subsubsection*{Functions}


\begin{savenotes}\sphinxattablestart
\sphinxthistablewithglobalstyle
\sphinxthistablewithnovlinesstyle
\centering
\begin{tabulary}{\linewidth}[t]{\X{1}{2}\X{1}{2}}
\sphinxtoprule
\sphinxtableatstartofbodyhook
\sphinxAtStartPar
{\hyperref[\detokenize{_autosummary/functionfinder.datafunctions.calculate_best_ideal:functionfinder.datafunctions.calculate_best_ideal}]{\sphinxcrossref{\sphinxcode{\sphinxupquote{calculate\_best\_ideal}}}}}(test\_value, ...{[}, function{]})
&
\sphinxAtStartPar
Select ideal function with minimal deviation to given point.
\\
\sphinxhline
\sphinxAtStartPar
{\hyperref[\detokenize{_autosummary/functionfinder.datafunctions.checktypes:functionfinder.datafunctions.checktypes}]{\sphinxcrossref{\sphinxcode{\sphinxupquote{checktypes}}}}}(functionname, typedict)
&
\sphinxAtStartPar
Check if types of data passed to a function upon call meet requirements.
\\
\sphinxhline
\sphinxAtStartPar
{\hyperref[\detokenize{_autosummary/functionfinder.datafunctions.create_empty_sqlitedb:functionfinder.datafunctions.create_empty_sqlitedb}]{\sphinxcrossref{\sphinxcode{\sphinxupquote{create\_empty\_sqlitedb}}}}}(dbname)
&
\sphinxAtStartPar
Create empty SQLite Database.
\\
\sphinxhline
\sphinxAtStartPar
{\hyperref[\detokenize{_autosummary/functionfinder.datafunctions.csv2sql_directly:functionfinder.datafunctions.csv2sql_directly}]{\sphinxcrossref{\sphinxcode{\sphinxupquote{csv2sql\_directly}}}}}(existing\_db, csv\_toadd, ...)
&
\sphinxAtStartPar
Import csv files into existing SQLite database using subprocess.
\\
\sphinxhline
\sphinxAtStartPar
{\hyperref[\detokenize{_autosummary/functionfinder.datafunctions.csv2sql_pandas:functionfinder.datafunctions.csv2sql_pandas}]{\sphinxcrossref{\sphinxcode{\sphinxupquote{csv2sql\_pandas}}}}}(existing\_db, csv\_toadd, tablename)
&
\sphinxAtStartPar
Import csv files into existing SQLite database using pandas.
\\
\sphinxhline
\sphinxAtStartPar
{\hyperref[\detokenize{_autosummary/functionfinder.datafunctions.min_error:functionfinder.datafunctions.min_error}]{\sphinxcrossref{\sphinxcode{\sphinxupquote{min\_error}}}}}(train\_set, ideal\_df)
&
\sphinxAtStartPar
Select column of a pandas DataFrame with minimal error to a Series.
\\
\sphinxbottomrule
\end{tabulary}
\sphinxtableafterendhook\par
\sphinxattableend\end{savenotes}

\sphinxstepscope


\subsection{functionfinder.datafunctions.calculate\_best\_ideal}
\label{\detokenize{_autosummary/functionfinder.datafunctions.calculate_best_ideal:functionfinder-datafunctions-calculate-best-ideal}}\label{\detokenize{_autosummary/functionfinder.datafunctions.calculate_best_ideal::doc}}\index{calculate\_best\_ideal() (in module functionfinder.datafunctions)@\spxentry{calculate\_best\_ideal()}\spxextra{in module functionfinder.datafunctions}}

\begin{fulllineitems}
\phantomsection\label{\detokenize{_autosummary/functionfinder.datafunctions.calculate_best_ideal:functionfinder.datafunctions.calculate_best_ideal}}
\pysigstartsignatures
\pysiglinewithargsret{\sphinxcode{\sphinxupquote{functionfinder.datafunctions.}}\sphinxbfcode{\sphinxupquote{calculate\_best\_ideal}}}{\emph{\DUrole{n}{test\_value}}, \emph{\DUrole{n}{match\_against}}, \emph{\DUrole{n}{index}}, \emph{\DUrole{n}{idealdata}}, \emph{\DUrole{n}{function=\textless{}built\sphinxhyphen{}in function min\textgreater{}}}}{}
\pysigstopsignatures
\sphinxAtStartPar
Select ideal function with minimal deviation to given point.


\subsubsection{Parameters}
\label{\detokenize{_autosummary/functionfinder.datafunctions.calculate_best_ideal:parameters}}\begin{description}
\sphinxlineitem{test\_value}{[}list{]}
\sphinxAtStartPar
Current row of test.csv file containing two values (x and y).

\sphinxlineitem{match\_against}{[}dictionary{]}
\sphinxAtStartPar
Matches of training data to selected ideal functions including
calculated deviation.

\sphinxlineitem{index}{[}float{]}
\sphinxAtStartPar
Read from x\sphinxhyphen{}value of test\_value parameter. Used to index resulting
DataFrame.

\sphinxlineitem{idealdata}{[}pandas.DataFrame{]}
\sphinxAtStartPar
Ideal functions from ideal dataset. Used to find function with closest
y\sphinxhyphen{}value at indexed position (x\sphinxhyphen{}value)

\sphinxlineitem{function}{[}method, optional{]}
\sphinxAtStartPar
Desired method to filter resulting deviation values.
The default is min (the observation with the lowest deviation will be
returned). When passing ‘raw’ all rows of the resulting DataFrame will
be returned (used to write all DataFrame lines to logfile for
evaluating certain values).

\end{description}


\subsubsection{Returns}
\label{\detokenize{_autosummary/functionfinder.datafunctions.calculate_best_ideal:returns}}
\sphinxAtStartPar
result : pandas.DataFrame
Containing the deviation (DeltaY), number of matched ideal function
(Idealfunktion) and a boolean whether the deviation exceeded the given
limit, filtered by passed ‘function’ method upon call.

\end{fulllineitems}


\sphinxstepscope


\subsection{functionfinder.datafunctions.checktypes}
\label{\detokenize{_autosummary/functionfinder.datafunctions.checktypes:functionfinder-datafunctions-checktypes}}\label{\detokenize{_autosummary/functionfinder.datafunctions.checktypes::doc}}\index{checktypes() (in module functionfinder.datafunctions)@\spxentry{checktypes()}\spxextra{in module functionfinder.datafunctions}}

\begin{fulllineitems}
\phantomsection\label{\detokenize{_autosummary/functionfinder.datafunctions.checktypes:functionfinder.datafunctions.checktypes}}
\pysigstartsignatures
\pysiglinewithargsret{\sphinxcode{\sphinxupquote{functionfinder.datafunctions.}}\sphinxbfcode{\sphinxupquote{checktypes}}}{\emph{\DUrole{n}{functionname}}, \emph{\DUrole{n}{typedict}}}{}
\pysigstopsignatures
\sphinxAtStartPar
Check if types of data passed to a function upon call meet requirements.


\subsubsection{Parameters}
\label{\detokenize{_autosummary/functionfinder.datafunctions.checktypes:parameters}}\begin{description}
\sphinxlineitem{functionname}{[}string{]}
\sphinxAtStartPar
Name of calling function, needed in case of error\sphinxhyphen{}logging.

\sphinxlineitem{typedict}{[}dictionary{]}
\sphinxAtStartPar
Required parameters of functions and according datatypes. Format has
to be \{object\sphinxhyphen{}name as string: (object\sphinxhyphen{}name, object\sphinxhyphen{}type)\}

\end{description}


\subsubsection{Returns}
\label{\detokenize{_autosummary/functionfinder.datafunctions.checktypes:returns}}
\sphinxAtStartPar
Raise Error on failure and quit program.

\end{fulllineitems}


\sphinxstepscope


\subsection{functionfinder.datafunctions.create\_empty\_sqlitedb}
\label{\detokenize{_autosummary/functionfinder.datafunctions.create_empty_sqlitedb:functionfinder-datafunctions-create-empty-sqlitedb}}\label{\detokenize{_autosummary/functionfinder.datafunctions.create_empty_sqlitedb::doc}}\index{create\_empty\_sqlitedb() (in module functionfinder.datafunctions)@\spxentry{create\_empty\_sqlitedb()}\spxextra{in module functionfinder.datafunctions}}

\begin{fulllineitems}
\phantomsection\label{\detokenize{_autosummary/functionfinder.datafunctions.create_empty_sqlitedb:functionfinder.datafunctions.create_empty_sqlitedb}}
\pysigstartsignatures
\pysiglinewithargsret{\sphinxcode{\sphinxupquote{functionfinder.datafunctions.}}\sphinxbfcode{\sphinxupquote{create\_empty\_sqlitedb}}}{\emph{\DUrole{n}{dbname}}}{}
\pysigstopsignatures
\sphinxAtStartPar
Create empty SQLite Database.

\sphinxAtStartPar
Name and location are specified by handed parameter and data output folder
in config.py. Existing databases in same location of the same name are
deleted before creation.


\subsubsection{Parameters}
\label{\detokenize{_autosummary/functionfinder.datafunctions.create_empty_sqlitedb:parameters}}\begin{description}
\sphinxlineitem{dbname}{[}string{]}
\sphinxAtStartPar
Name of database to create.

\end{description}

\end{fulllineitems}


\sphinxstepscope


\subsection{functionfinder.datafunctions.csv2sql\_directly}
\label{\detokenize{_autosummary/functionfinder.datafunctions.csv2sql_directly:functionfinder-datafunctions-csv2sql-directly}}\label{\detokenize{_autosummary/functionfinder.datafunctions.csv2sql_directly::doc}}\index{csv2sql\_directly() (in module functionfinder.datafunctions)@\spxentry{csv2sql\_directly()}\spxextra{in module functionfinder.datafunctions}}

\begin{fulllineitems}
\phantomsection\label{\detokenize{_autosummary/functionfinder.datafunctions.csv2sql_directly:functionfinder.datafunctions.csv2sql_directly}}
\pysigstartsignatures
\pysiglinewithargsret{\sphinxcode{\sphinxupquote{functionfinder.datafunctions.}}\sphinxbfcode{\sphinxupquote{csv2sql\_directly}}}{\emph{\DUrole{n}{existing\_db}}, \emph{\DUrole{n}{csv\_toadd}}, \emph{\DUrole{n}{tablename}}}{}
\pysigstopsignatures
\sphinxAtStartPar
Import csv files into existing SQLite database using subprocess.

\sphinxAtStartPar
Direct import using subprocess enhances importing large csv files.


\subsubsection{Parameters}
\label{\detokenize{_autosummary/functionfinder.datafunctions.csv2sql_directly:parameters}}\begin{description}
\sphinxlineitem{existing\_db}{[}string{]}
\sphinxAtStartPar
Name of existing SQLite database. Location/Directory is handled by
parameter set in config.py.

\sphinxlineitem{csv\_toadd}{[}string{]}
\sphinxAtStartPar
Location and name of csv\sphinxhyphen{}file to import.

\sphinxlineitem{tablename}{[}string{]}
\sphinxAtStartPar
Name of table in SQLite database in which to write the data.

\end{description}

\end{fulllineitems}


\sphinxstepscope


\subsection{functionfinder.datafunctions.csv2sql\_pandas}
\label{\detokenize{_autosummary/functionfinder.datafunctions.csv2sql_pandas:functionfinder-datafunctions-csv2sql-pandas}}\label{\detokenize{_autosummary/functionfinder.datafunctions.csv2sql_pandas::doc}}\index{csv2sql\_pandas() (in module functionfinder.datafunctions)@\spxentry{csv2sql\_pandas()}\spxextra{in module functionfinder.datafunctions}}

\begin{fulllineitems}
\phantomsection\label{\detokenize{_autosummary/functionfinder.datafunctions.csv2sql_pandas:functionfinder.datafunctions.csv2sql_pandas}}
\pysigstartsignatures
\pysiglinewithargsret{\sphinxcode{\sphinxupquote{functionfinder.datafunctions.}}\sphinxbfcode{\sphinxupquote{csv2sql\_pandas}}}{\emph{\DUrole{n}{existing\_db}}, \emph{\DUrole{n}{csv\_toadd}}, \emph{\DUrole{n}{tablename}}}{}
\pysigstopsignatures
\sphinxAtStartPar
Import csv files into existing SQLite database using pandas.

\sphinxAtStartPar
No direct import. csv files are first read into a pandas dataframe and
then fed into an existing SQLite database.


\subsubsection{Parameters}
\label{\detokenize{_autosummary/functionfinder.datafunctions.csv2sql_pandas:parameters}}\begin{description}
\sphinxlineitem{existing\_db}{[}string{]}
\sphinxAtStartPar
Name of existing SQLite database. Location/Directory is handled by
parameter set in config.py.

\sphinxlineitem{csv\_toadd}{[}string{]}
\sphinxAtStartPar
Location and name of csv\sphinxhyphen{}file to import.

\sphinxlineitem{tablename}{[}string{]}
\sphinxAtStartPar
Name of table in SQLite database in which to write the data.

\end{description}

\end{fulllineitems}


\sphinxstepscope


\subsection{functionfinder.datafunctions.min\_error}
\label{\detokenize{_autosummary/functionfinder.datafunctions.min_error:functionfinder-datafunctions-min-error}}\label{\detokenize{_autosummary/functionfinder.datafunctions.min_error::doc}}\index{min\_error() (in module functionfinder.datafunctions)@\spxentry{min\_error()}\spxextra{in module functionfinder.datafunctions}}

\begin{fulllineitems}
\phantomsection\label{\detokenize{_autosummary/functionfinder.datafunctions.min_error:functionfinder.datafunctions.min_error}}
\pysigstartsignatures
\pysiglinewithargsret{\sphinxcode{\sphinxupquote{functionfinder.datafunctions.}}\sphinxbfcode{\sphinxupquote{min\_error}}}{\emph{\DUrole{n}{train\_set}}, \emph{\DUrole{n}{ideal\_df}}}{}
\pysigstopsignatures
\sphinxAtStartPar
Select column of a pandas DataFrame with minimal error to a Series.


\subsubsection{Parameters}
\label{\detokenize{_autosummary/functionfinder.datafunctions.min_error:parameters}}\begin{description}
\sphinxlineitem{train\_set}{[}pandas Series{]}
\sphinxAtStartPar
Column of training dataset, that should be matched against DataFrame
of ideal functions.

\sphinxlineitem{ideal\_df}{[}pandas DataFrame{]}
\sphinxAtStartPar
DataFrame of ideal functions.

\end{description}


\subsubsection{Returns}
\label{\detokenize{_autosummary/functionfinder.datafunctions.min_error:returns}}
\sphinxAtStartPar
selected, last\_val : tupel
selected : string
\begin{quote}

\sphinxAtStartPar
Column name of matched ideal function for this train\_set.
\end{quote}
\begin{description}
\sphinxlineitem{last\_val}{[}float{]}
\sphinxAtStartPar
Deviation between train\_set and matched ideal function according to
the error\_calculation method.

\end{description}

\end{fulllineitems}


\sphinxstepscope


\section{functionfinder.exceptions}
\label{\detokenize{_autosummary/functionfinder.exceptions:module-functionfinder.exceptions}}\label{\detokenize{_autosummary/functionfinder.exceptions:functionfinder-exceptions}}\label{\detokenize{_autosummary/functionfinder.exceptions::doc}}\index{module@\spxentry{module}!functionfinder.exceptions@\spxentry{functionfinder.exceptions}}\index{functionfinder.exceptions@\spxentry{functionfinder.exceptions}!module@\spxentry{module}}
\sphinxAtStartPar
Userdefined exceptions.

\sphinxAtStartPar
This module contains userdefined exceptions.
\subsubsection*{Exceptions}


\begin{savenotes}\sphinxattablestart
\sphinxthistablewithglobalstyle
\sphinxthistablewithnovlinesstyle
\centering
\begin{tabulary}{\linewidth}[t]{\X{1}{2}\X{1}{2}}
\sphinxtoprule
\sphinxtableatstartofbodyhook
\sphinxAtStartPar
{\hyperref[\detokenize{_autosummary/functionfinder.exceptions.TypeError:functionfinder.exceptions.TypeError}]{\sphinxcrossref{\sphinxcode{\sphinxupquote{TypeError}}}}}
&
\sphinxAtStartPar
Exception raised during check of datatypes of function parameters.
\\
\sphinxbottomrule
\end{tabulary}
\sphinxtableafterendhook\par
\sphinxattableend\end{savenotes}

\sphinxstepscope


\subsection{functionfinder.exceptions.TypeError}
\label{\detokenize{_autosummary/functionfinder.exceptions.TypeError:functionfinder-exceptions-typeerror}}\label{\detokenize{_autosummary/functionfinder.exceptions.TypeError::doc}}\index{TypeError@\spxentry{TypeError}}

\begin{fulllineitems}
\phantomsection\label{\detokenize{_autosummary/functionfinder.exceptions.TypeError:functionfinder.exceptions.TypeError}}
\pysigstartsignatures
\pysigline{\sphinxbfcode{\sphinxupquote{exception\DUrole{w}{  }}}\sphinxcode{\sphinxupquote{functionfinder.exceptions.}}\sphinxbfcode{\sphinxupquote{TypeError}}}
\pysigstopsignatures
\sphinxAtStartPar
Exception raised during check of datatypes of function parameters.

\end{fulllineitems}


\sphinxstepscope


\section{functionfinder.ffrunner}
\label{\detokenize{_autosummary/functionfinder.ffrunner:module-functionfinder.ffrunner}}\label{\detokenize{_autosummary/functionfinder.ffrunner:functionfinder-ffrunner}}\label{\detokenize{_autosummary/functionfinder.ffrunner::doc}}\index{module@\spxentry{module}!functionfinder.ffrunner@\spxentry{functionfinder.ffrunner}}\index{functionfinder.ffrunner@\spxentry{functionfinder.ffrunner}!module@\spxentry{module}}
\sphinxAtStartPar
Main program.

\sphinxAtStartPar
This script contains the main part of the programm which orchestrates
calculations and can be called, after installation, by the CLI\sphinxhyphen{}command ff.


\subsection{Methods}
\label{\detokenize{_autosummary/functionfinder.ffrunner:methods}}\begin{description}
\sphinxlineitem{task()}
\sphinxAtStartPar
Run the calculations to solve the given task.

\end{description}
\subsubsection*{Functions}


\begin{savenotes}\sphinxattablestart
\sphinxthistablewithglobalstyle
\sphinxthistablewithnovlinesstyle
\centering
\begin{tabulary}{\linewidth}[t]{\X{1}{2}\X{1}{2}}
\sphinxtoprule
\sphinxtableatstartofbodyhook
\sphinxAtStartPar
{\hyperref[\detokenize{_autosummary/functionfinder.ffrunner.task:functionfinder.ffrunner.task}]{\sphinxcrossref{\sphinxcode{\sphinxupquote{task}}}}}()
&
\sphinxAtStartPar
Run the calculations to solve the given task.
\\
\sphinxbottomrule
\end{tabulary}
\sphinxtableafterendhook\par
\sphinxattableend\end{savenotes}

\sphinxstepscope


\subsection{functionfinder.ffrunner.task}
\label{\detokenize{_autosummary/functionfinder.ffrunner.task:functionfinder-ffrunner-task}}\label{\detokenize{_autosummary/functionfinder.ffrunner.task::doc}}\index{task() (in module functionfinder.ffrunner)@\spxentry{task()}\spxextra{in module functionfinder.ffrunner}}

\begin{fulllineitems}
\phantomsection\label{\detokenize{_autosummary/functionfinder.ffrunner.task:functionfinder.ffrunner.task}}
\pysigstartsignatures
\pysiglinewithargsret{\sphinxcode{\sphinxupquote{functionfinder.ffrunner.}}\sphinxbfcode{\sphinxupquote{task}}}{}{}
\pysigstopsignatures
\sphinxAtStartPar
Run the calculations to solve the given task.

\sphinxAtStartPar
Evaluate ideal functions for a set of training data (1) and assign
values of a test\sphinxhyphen{}dataset to those ideal functions (2)

\sphinxAtStartPar
Used criteria for evaluation:
(1) to match training data and ideal functions:
minimum SummedSquaredError (SSE)
(2) to match ideal functions and test data:
precalculated SSE (1) * SquareRoot(2)

\end{fulllineitems}


\sphinxstepscope


\section{functionfinder.log}
\label{\detokenize{_autosummary/functionfinder.log:module-functionfinder.log}}\label{\detokenize{_autosummary/functionfinder.log:functionfinder-log}}\label{\detokenize{_autosummary/functionfinder.log::doc}}\index{module@\spxentry{module}!functionfinder.log@\spxentry{functionfinder.log}}\index{functionfinder.log@\spxentry{functionfinder.log}!module@\spxentry{module}}
\sphinxAtStartPar
Definition of logging parameters.

\sphinxAtStartPar
This script defines the basic logging configuration for the main program.
User configurable variables are read from ‘config’ module.
\subsubsection*{Functions}


\begin{savenotes}\sphinxattablestart
\sphinxthistablewithglobalstyle
\sphinxthistablewithnovlinesstyle
\centering
\begin{tabulary}{\linewidth}[t]{\X{1}{2}\X{1}{2}}
\sphinxtoprule
\sphinxtableatstartofbodyhook
\sphinxAtStartPar
{\hyperref[\detokenize{_autosummary/functionfinder.log.setlogging:functionfinder.log.setlogging}]{\sphinxcrossref{\sphinxcode{\sphinxupquote{setlogging}}}}}()
&
\sphinxAtStartPar
Set predefined parameters for logging.
\\
\sphinxbottomrule
\end{tabulary}
\sphinxtableafterendhook\par
\sphinxattableend\end{savenotes}

\sphinxstepscope


\subsection{functionfinder.log.setlogging}
\label{\detokenize{_autosummary/functionfinder.log.setlogging:functionfinder-log-setlogging}}\label{\detokenize{_autosummary/functionfinder.log.setlogging::doc}}\index{setlogging() (in module functionfinder.log)@\spxentry{setlogging()}\spxextra{in module functionfinder.log}}

\begin{fulllineitems}
\phantomsection\label{\detokenize{_autosummary/functionfinder.log.setlogging:functionfinder.log.setlogging}}
\pysigstartsignatures
\pysiglinewithargsret{\sphinxcode{\sphinxupquote{functionfinder.log.}}\sphinxbfcode{\sphinxupquote{setlogging}}}{}{}
\pysigstopsignatures
\sphinxAtStartPar
Set predefined parameters for logging.

\end{fulllineitems}


\sphinxstepscope


\section{functionfinder.setuplog}
\label{\detokenize{_autosummary/functionfinder.setuplog:module-functionfinder.setuplog}}\label{\detokenize{_autosummary/functionfinder.setuplog:functionfinder-setuplog}}\label{\detokenize{_autosummary/functionfinder.setuplog::doc}}\index{module@\spxentry{module}!functionfinder.setuplog@\spxentry{functionfinder.setuplog}}\index{functionfinder.setuplog@\spxentry{functionfinder.setuplog}!module@\spxentry{module}}
\sphinxAtStartPar
Definition of logging parameters for the installation process.

\sphinxAtStartPar
This script defines the basic logging configuration for the setup process of
this program. User configurable variables are read from ‘config’ module.
\subsubsection*{Functions}


\begin{savenotes}\sphinxattablestart
\sphinxthistablewithglobalstyle
\sphinxthistablewithnovlinesstyle
\centering
\begin{tabulary}{\linewidth}[t]{\X{1}{2}\X{1}{2}}
\sphinxtoprule
\sphinxtableatstartofbodyhook
\sphinxAtStartPar
{\hyperref[\detokenize{_autosummary/functionfinder.setuplog.set_setuplogging:functionfinder.setuplog.set_setuplogging}]{\sphinxcrossref{\sphinxcode{\sphinxupquote{set\_setuplogging}}}}}()
&
\sphinxAtStartPar
Set predefined parameters for logging setup process.
\\
\sphinxbottomrule
\end{tabulary}
\sphinxtableafterendhook\par
\sphinxattableend\end{savenotes}

\sphinxstepscope


\subsection{functionfinder.setuplog.set\_setuplogging}
\label{\detokenize{_autosummary/functionfinder.setuplog.set_setuplogging:functionfinder-setuplog-set-setuplogging}}\label{\detokenize{_autosummary/functionfinder.setuplog.set_setuplogging::doc}}\index{set\_setuplogging() (in module functionfinder.setuplog)@\spxentry{set\_setuplogging()}\spxextra{in module functionfinder.setuplog}}

\begin{fulllineitems}
\phantomsection\label{\detokenize{_autosummary/functionfinder.setuplog.set_setuplogging:functionfinder.setuplog.set_setuplogging}}
\pysigstartsignatures
\pysiglinewithargsret{\sphinxcode{\sphinxupquote{functionfinder.setuplog.}}\sphinxbfcode{\sphinxupquote{set\_setuplogging}}}{}{}
\pysigstopsignatures
\sphinxAtStartPar
Set predefined parameters for logging setup process.

\end{fulllineitems}


\sphinxstepscope


\chapter{tests}
\label{\detokenize{_autosummary/tests:module-tests}}\label{\detokenize{_autosummary/tests:tests}}\label{\detokenize{_autosummary/tests::doc}}\index{module@\spxentry{module}!tests@\spxentry{tests}}\index{tests@\spxentry{tests}!module@\spxentry{module}}\subsubsection*{Modules}


\begin{savenotes}\sphinxattablestart
\sphinxthistablewithglobalstyle
\sphinxthistablewithnovlinesstyle
\centering
\begin{tabulary}{\linewidth}[t]{\X{1}{2}\X{1}{2}}
\sphinxtoprule
\sphinxtableatstartofbodyhook
\sphinxAtStartPar
{\hyperref[\detokenize{_autosummary/tests.test_unit:module-tests.test_unit}]{\sphinxcrossref{\sphinxcode{\sphinxupquote{tests.test\_unit}}}}}
&
\sphinxAtStartPar
Definition of UnitTest\sphinxhyphen{}Sets.
\\
\sphinxbottomrule
\end{tabulary}
\sphinxtableafterendhook\par
\sphinxattableend\end{savenotes}

\sphinxstepscope


\section{tests.test\_unit}
\label{\detokenize{_autosummary/tests.test_unit:module-tests.test_unit}}\label{\detokenize{_autosummary/tests.test_unit:tests-test-unit}}\label{\detokenize{_autosummary/tests.test_unit::doc}}\index{module@\spxentry{module}!tests.test\_unit@\spxentry{tests.test\_unit}}\index{tests.test\_unit@\spxentry{tests.test\_unit}!module@\spxentry{module}}
\sphinxAtStartPar
Definition of UnitTest\sphinxhyphen{}Sets.

\sphinxAtStartPar
This script contains sets of UnitTests to check the functionality of the
functionfinder package and is called during the setup process by setup.py.


\subsection{Notes}
\label{\detokenize{_autosummary/tests.test_unit:notes}}
\sphinxAtStartPar
Tests of the calculation functions have not been written, due to the fact that
calculation functions are, by the author, meant to be open for modification and
therefore cannot be tested with specific values.


\subsection{TestSets}
\label{\detokenize{_autosummary/tests.test_unit:testsets}}\begin{description}
\sphinxlineitem{test\_sqlite}
\sphinxAtStartPar
Test data functions with regards to SQLite operations.

\sphinxlineitem{test\_df}
\sphinxAtStartPar
Test basic data properties such as the structure of provided data files.

\end{description}


\subsection{Methods}
\label{\detokenize{_autosummary/tests.test_unit:methods}}\begin{description}
\sphinxlineitem{check\_struct(structdata)}
\sphinxAtStartPar
Check structure of a passed DataFrame to certain needs. Used in test\_df
TestSet.

\end{description}
\subsubsection*{Functions}


\begin{savenotes}\sphinxattablestart
\sphinxthistablewithglobalstyle
\sphinxthistablewithnovlinesstyle
\centering
\begin{tabulary}{\linewidth}[t]{\X{1}{2}\X{1}{2}}
\sphinxtoprule
\sphinxtableatstartofbodyhook
\sphinxAtStartPar
{\hyperref[\detokenize{_autosummary/tests.test_unit.check_struct:tests.test_unit.check_struct}]{\sphinxcrossref{\sphinxcode{\sphinxupquote{check\_struct}}}}}(structdata)
&
\sphinxAtStartPar
Check structure of passed DataFrame to certain needs.
\\
\sphinxbottomrule
\end{tabulary}
\sphinxtableafterendhook\par
\sphinxattableend\end{savenotes}

\sphinxstepscope


\subsection{tests.test\_unit.check\_struct}
\label{\detokenize{_autosummary/tests.test_unit.check_struct:tests-test-unit-check-struct}}\label{\detokenize{_autosummary/tests.test_unit.check_struct::doc}}\index{check\_struct() (in module tests.test\_unit)@\spxentry{check\_struct()}\spxextra{in module tests.test\_unit}}

\begin{fulllineitems}
\phantomsection\label{\detokenize{_autosummary/tests.test_unit.check_struct:tests.test_unit.check_struct}}
\pysigstartsignatures
\pysiglinewithargsret{\sphinxcode{\sphinxupquote{tests.test\_unit.}}\sphinxbfcode{\sphinxupquote{check\_struct}}}{\emph{\DUrole{n}{structdata}}}{}
\pysigstopsignatures
\sphinxAtStartPar
Check structure of passed DataFrame to certain needs.

\sphinxAtStartPar
Check if column of name ‘x’ exists and if the number of columns is at
least two.


\subsubsection{Parameters}
\label{\detokenize{_autosummary/tests.test_unit.check_struct:parameters}}\begin{description}
\sphinxlineitem{structdata}{[}pandas.DataFrame{]}
\sphinxAtStartPar
DataFrame to check for above criteria.

\end{description}


\subsubsection{Returns}
\label{\detokenize{_autosummary/tests.test_unit.check_struct:returns}}\begin{description}
\sphinxlineitem{checkresult}{[}bool{]}
\sphinxAtStartPar
If True, all checks were passed.

\end{description}

\end{fulllineitems}

\subsubsection*{Classes}


\begin{savenotes}\sphinxattablestart
\sphinxthistablewithglobalstyle
\sphinxthistablewithnovlinesstyle
\centering
\begin{tabulary}{\linewidth}[t]{\X{1}{2}\X{1}{2}}
\sphinxtoprule
\sphinxtableatstartofbodyhook
\sphinxAtStartPar
{\hyperref[\detokenize{_autosummary/tests.test_unit.test_df:tests.test_unit.test_df}]{\sphinxcrossref{\sphinxcode{\sphinxupquote{test\_df}}}}}({[}methodName{]})
&
\sphinxAtStartPar

\\
\sphinxhline
\sphinxAtStartPar
{\hyperref[\detokenize{_autosummary/tests.test_unit.test_sqlite:tests.test_unit.test_sqlite}]{\sphinxcrossref{\sphinxcode{\sphinxupquote{test\_sqlite}}}}}({[}methodName{]})
&
\sphinxAtStartPar

\\
\sphinxbottomrule
\end{tabulary}
\sphinxtableafterendhook\par
\sphinxattableend\end{savenotes}

\sphinxstepscope


\subsection{tests.test\_unit.test\_df}
\label{\detokenize{_autosummary/tests.test_unit.test_df:tests-test-unit-test-df}}\label{\detokenize{_autosummary/tests.test_unit.test_df::doc}}\index{test\_df (class in tests.test\_unit)@\spxentry{test\_df}\spxextra{class in tests.test\_unit}}

\begin{fulllineitems}
\phantomsection\label{\detokenize{_autosummary/tests.test_unit.test_df:tests.test_unit.test_df}}
\pysigstartsignatures
\pysiglinewithargsret{\sphinxbfcode{\sphinxupquote{class\DUrole{w}{  }}}\sphinxcode{\sphinxupquote{tests.test\_unit.}}\sphinxbfcode{\sphinxupquote{test\_df}}}{\emph{\DUrole{n}{methodName}\DUrole{o}{=}\DUrole{default_value}{\textquotesingle{}runTest\textquotesingle{}}}}{}
\pysigstopsignatures
\sphinxAtStartPar
Bases: \sphinxcode{\sphinxupquote{TestCase}}
\index{\_\_init\_\_() (tests.test\_unit.test\_df method)@\spxentry{\_\_init\_\_()}\spxextra{tests.test\_unit.test\_df method}}

\begin{fulllineitems}
\phantomsection\label{\detokenize{_autosummary/tests.test_unit.test_df:tests.test_unit.test_df.__init__}}
\pysigstartsignatures
\pysiglinewithargsret{\sphinxbfcode{\sphinxupquote{\_\_init\_\_}}}{\emph{\DUrole{n}{methodName}\DUrole{o}{=}\DUrole{default_value}{\textquotesingle{}runTest\textquotesingle{}}}}{}
\pysigstopsignatures
\sphinxAtStartPar
Create an instance of the class that will use the named test
method when executed. Raises a ValueError if the instance does
not have a method with the specified name.

\end{fulllineitems}

\subsubsection*{Methods}


\begin{savenotes}
\sphinxatlongtablestart
\sphinxthistablewithglobalstyle
\sphinxthistablewithnovlinesstyle
\begin{longtable}[c]{\X{1}{2}\X{1}{2}}
\sphinxtoprule
\endfirsthead

\multicolumn{2}{c}{\sphinxnorowcolor
    \makebox[0pt]{\sphinxtablecontinued{\tablename\ \thetable{} \textendash{} continued from previous page}}%
}\\
\sphinxtoprule
\endhead

\sphinxbottomrule
\multicolumn{2}{r}{\sphinxnorowcolor
    \makebox[0pt][r]{\sphinxtablecontinued{continues on next page}}%
}\\
\endfoot

\endlastfoot
\sphinxtableatstartofbodyhook

\sphinxAtStartPar
{\hyperref[\detokenize{_autosummary/tests.test_unit.test_df:tests.test_unit.test_df.__init__}]{\sphinxcrossref{\sphinxcode{\sphinxupquote{\_\_init\_\_}}}}}({[}methodName{]})
&
\sphinxAtStartPar
Create an instance of the class that will use the named test method when executed.
\\
\sphinxhline
\sphinxAtStartPar
{\hyperref[\detokenize{_autosummary/tests.test_unit.test_df:tests.test_unit.test_df.addClassCleanup}]{\sphinxcrossref{\sphinxcode{\sphinxupquote{addClassCleanup}}}}}(function, /, *args, **kwargs)
&
\sphinxAtStartPar
Same as addCleanup, except the cleanup items are called even if setUpClass fails (unlike tearDownClass).
\\
\sphinxhline
\sphinxAtStartPar
{\hyperref[\detokenize{_autosummary/tests.test_unit.test_df:tests.test_unit.test_df.addCleanup}]{\sphinxcrossref{\sphinxcode{\sphinxupquote{addCleanup}}}}}(function, /, *args, **kwargs)
&
\sphinxAtStartPar
Add a function, with arguments, to be called when the test is completed.
\\
\sphinxhline
\sphinxAtStartPar
{\hyperref[\detokenize{_autosummary/tests.test_unit.test_df:tests.test_unit.test_df.addTypeEqualityFunc}]{\sphinxcrossref{\sphinxcode{\sphinxupquote{addTypeEqualityFunc}}}}}(typeobj, function)
&
\sphinxAtStartPar
Add a type specific assertEqual style function to compare a type.
\\
\sphinxhline
\sphinxAtStartPar
{\hyperref[\detokenize{_autosummary/tests.test_unit.test_df:tests.test_unit.test_df.assertAlmostEqual}]{\sphinxcrossref{\sphinxcode{\sphinxupquote{assertAlmostEqual}}}}}(first, second{[}, places, ...{]})
&
\sphinxAtStartPar
Fail if the two objects are unequal as determined by their difference rounded to the given number of decimal places (default 7) and comparing to zero, or by comparing that the difference between the two objects is more than the given delta.
\\
\sphinxhline
\sphinxAtStartPar
\sphinxcode{\sphinxupquote{assertAlmostEquals}}(**kwargs)
&
\sphinxAtStartPar

\\
\sphinxhline
\sphinxAtStartPar
{\hyperref[\detokenize{_autosummary/tests.test_unit.test_df:tests.test_unit.test_df.assertCountEqual}]{\sphinxcrossref{\sphinxcode{\sphinxupquote{assertCountEqual}}}}}(first, second{[}, msg{]})
&
\sphinxAtStartPar
Asserts that two iterables have the same elements, the same number of times, without regard to order.
\\
\sphinxhline
\sphinxAtStartPar
{\hyperref[\detokenize{_autosummary/tests.test_unit.test_df:tests.test_unit.test_df.assertDictContainsSubset}]{\sphinxcrossref{\sphinxcode{\sphinxupquote{assertDictContainsSubset}}}}}(subset, dictionary)
&
\sphinxAtStartPar
Checks whether dictionary is a superset of subset.
\\
\sphinxhline
\sphinxAtStartPar
\sphinxcode{\sphinxupquote{assertDictEqual}}(d1, d2{[}, msg{]})
&
\sphinxAtStartPar

\\
\sphinxhline
\sphinxAtStartPar
{\hyperref[\detokenize{_autosummary/tests.test_unit.test_df:tests.test_unit.test_df.assertEqual}]{\sphinxcrossref{\sphinxcode{\sphinxupquote{assertEqual}}}}}(first, second{[}, msg{]})
&
\sphinxAtStartPar
Fail if the two objects are unequal as determined by the \textquotesingle{}==\textquotesingle{} operator.
\\
\sphinxhline
\sphinxAtStartPar
\sphinxcode{\sphinxupquote{assertEquals}}(**kwargs)
&
\sphinxAtStartPar

\\
\sphinxhline
\sphinxAtStartPar
{\hyperref[\detokenize{_autosummary/tests.test_unit.test_df:tests.test_unit.test_df.assertFalse}]{\sphinxcrossref{\sphinxcode{\sphinxupquote{assertFalse}}}}}(expr{[}, msg{]})
&
\sphinxAtStartPar
Check that the expression is false.
\\
\sphinxhline
\sphinxAtStartPar
{\hyperref[\detokenize{_autosummary/tests.test_unit.test_df:tests.test_unit.test_df.assertGreater}]{\sphinxcrossref{\sphinxcode{\sphinxupquote{assertGreater}}}}}(a, b{[}, msg{]})
&
\sphinxAtStartPar
Just like self.assertTrue(a \textgreater{} b), but with a nicer default message.
\\
\sphinxhline
\sphinxAtStartPar
{\hyperref[\detokenize{_autosummary/tests.test_unit.test_df:tests.test_unit.test_df.assertGreaterEqual}]{\sphinxcrossref{\sphinxcode{\sphinxupquote{assertGreaterEqual}}}}}(a, b{[}, msg{]})
&
\sphinxAtStartPar
Just like self.assertTrue(a \textgreater{}= b), but with a nicer default message.
\\
\sphinxhline
\sphinxAtStartPar
{\hyperref[\detokenize{_autosummary/tests.test_unit.test_df:tests.test_unit.test_df.assertIn}]{\sphinxcrossref{\sphinxcode{\sphinxupquote{assertIn}}}}}(member, container{[}, msg{]})
&
\sphinxAtStartPar
Just like self.assertTrue(a in b), but with a nicer default message.
\\
\sphinxhline
\sphinxAtStartPar
{\hyperref[\detokenize{_autosummary/tests.test_unit.test_df:tests.test_unit.test_df.assertIs}]{\sphinxcrossref{\sphinxcode{\sphinxupquote{assertIs}}}}}(expr1, expr2{[}, msg{]})
&
\sphinxAtStartPar
Just like self.assertTrue(a is b), but with a nicer default message.
\\
\sphinxhline
\sphinxAtStartPar
{\hyperref[\detokenize{_autosummary/tests.test_unit.test_df:tests.test_unit.test_df.assertIsInstance}]{\sphinxcrossref{\sphinxcode{\sphinxupquote{assertIsInstance}}}}}(obj, cls{[}, msg{]})
&
\sphinxAtStartPar
Same as self.assertTrue(isinstance(obj, cls)), with a nicer default message.
\\
\sphinxhline
\sphinxAtStartPar
{\hyperref[\detokenize{_autosummary/tests.test_unit.test_df:tests.test_unit.test_df.assertIsNone}]{\sphinxcrossref{\sphinxcode{\sphinxupquote{assertIsNone}}}}}(obj{[}, msg{]})
&
\sphinxAtStartPar
Same as self.assertTrue(obj is None), with a nicer default message.
\\
\sphinxhline
\sphinxAtStartPar
{\hyperref[\detokenize{_autosummary/tests.test_unit.test_df:tests.test_unit.test_df.assertIsNot}]{\sphinxcrossref{\sphinxcode{\sphinxupquote{assertIsNot}}}}}(expr1, expr2{[}, msg{]})
&
\sphinxAtStartPar
Just like self.assertTrue(a is not b), but with a nicer default message.
\\
\sphinxhline
\sphinxAtStartPar
{\hyperref[\detokenize{_autosummary/tests.test_unit.test_df:tests.test_unit.test_df.assertIsNotNone}]{\sphinxcrossref{\sphinxcode{\sphinxupquote{assertIsNotNone}}}}}(obj{[}, msg{]})
&
\sphinxAtStartPar
Included for symmetry with assertIsNone.
\\
\sphinxhline
\sphinxAtStartPar
{\hyperref[\detokenize{_autosummary/tests.test_unit.test_df:tests.test_unit.test_df.assertLess}]{\sphinxcrossref{\sphinxcode{\sphinxupquote{assertLess}}}}}(a, b{[}, msg{]})
&
\sphinxAtStartPar
Just like self.assertTrue(a \textless{} b), but with a nicer default message.
\\
\sphinxhline
\sphinxAtStartPar
{\hyperref[\detokenize{_autosummary/tests.test_unit.test_df:tests.test_unit.test_df.assertLessEqual}]{\sphinxcrossref{\sphinxcode{\sphinxupquote{assertLessEqual}}}}}(a, b{[}, msg{]})
&
\sphinxAtStartPar
Just like self.assertTrue(a \textless{}= b), but with a nicer default message.
\\
\sphinxhline
\sphinxAtStartPar
{\hyperref[\detokenize{_autosummary/tests.test_unit.test_df:tests.test_unit.test_df.assertListEqual}]{\sphinxcrossref{\sphinxcode{\sphinxupquote{assertListEqual}}}}}(list1, list2{[}, msg{]})
&
\sphinxAtStartPar
A list\sphinxhyphen{}specific equality assertion.
\\
\sphinxhline
\sphinxAtStartPar
{\hyperref[\detokenize{_autosummary/tests.test_unit.test_df:tests.test_unit.test_df.assertLogs}]{\sphinxcrossref{\sphinxcode{\sphinxupquote{assertLogs}}}}}({[}logger, level{]})
&
\sphinxAtStartPar
Fail unless a log message of level \sphinxstyleemphasis{level} or higher is emitted on \sphinxstyleemphasis{logger\_name} or its children.
\\
\sphinxhline
\sphinxAtStartPar
{\hyperref[\detokenize{_autosummary/tests.test_unit.test_df:tests.test_unit.test_df.assertMultiLineEqual}]{\sphinxcrossref{\sphinxcode{\sphinxupquote{assertMultiLineEqual}}}}}(first, second{[}, msg{]})
&
\sphinxAtStartPar
Assert that two multi\sphinxhyphen{}line strings are equal.
\\
\sphinxhline
\sphinxAtStartPar
{\hyperref[\detokenize{_autosummary/tests.test_unit.test_df:tests.test_unit.test_df.assertNoLogs}]{\sphinxcrossref{\sphinxcode{\sphinxupquote{assertNoLogs}}}}}({[}logger, level{]})
&
\sphinxAtStartPar
Fail unless no log messages of level \sphinxstyleemphasis{level} or higher are emitted on \sphinxstyleemphasis{logger\_name} or its children.
\\
\sphinxhline
\sphinxAtStartPar
{\hyperref[\detokenize{_autosummary/tests.test_unit.test_df:tests.test_unit.test_df.assertNotAlmostEqual}]{\sphinxcrossref{\sphinxcode{\sphinxupquote{assertNotAlmostEqual}}}}}(first, second{[}, ...{]})
&
\sphinxAtStartPar
Fail if the two objects are equal as determined by their difference rounded to the given number of decimal places (default 7) and comparing to zero, or by comparing that the difference between the two objects is less than the given delta.
\\
\sphinxhline
\sphinxAtStartPar
\sphinxcode{\sphinxupquote{assertNotAlmostEquals}}(**kwargs)
&
\sphinxAtStartPar

\\
\sphinxhline
\sphinxAtStartPar
{\hyperref[\detokenize{_autosummary/tests.test_unit.test_df:tests.test_unit.test_df.assertNotEqual}]{\sphinxcrossref{\sphinxcode{\sphinxupquote{assertNotEqual}}}}}(first, second{[}, msg{]})
&
\sphinxAtStartPar
Fail if the two objects are equal as determined by the \textquotesingle{}!=\textquotesingle{} operator.
\\
\sphinxhline
\sphinxAtStartPar
\sphinxcode{\sphinxupquote{assertNotEquals}}(**kwargs)
&
\sphinxAtStartPar

\\
\sphinxhline
\sphinxAtStartPar
{\hyperref[\detokenize{_autosummary/tests.test_unit.test_df:tests.test_unit.test_df.assertNotIn}]{\sphinxcrossref{\sphinxcode{\sphinxupquote{assertNotIn}}}}}(member, container{[}, msg{]})
&
\sphinxAtStartPar
Just like self.assertTrue(a not in b), but with a nicer default message.
\\
\sphinxhline
\sphinxAtStartPar
{\hyperref[\detokenize{_autosummary/tests.test_unit.test_df:tests.test_unit.test_df.assertNotIsInstance}]{\sphinxcrossref{\sphinxcode{\sphinxupquote{assertNotIsInstance}}}}}(obj, cls{[}, msg{]})
&
\sphinxAtStartPar
Included for symmetry with assertIsInstance.
\\
\sphinxhline
\sphinxAtStartPar
{\hyperref[\detokenize{_autosummary/tests.test_unit.test_df:tests.test_unit.test_df.assertNotRegex}]{\sphinxcrossref{\sphinxcode{\sphinxupquote{assertNotRegex}}}}}(text, unexpected\_regex{[}, msg{]})
&
\sphinxAtStartPar
Fail the test if the text matches the regular expression.
\\
\sphinxhline
\sphinxAtStartPar
\sphinxcode{\sphinxupquote{assertNotRegexpMatches}}(**kwargs)
&
\sphinxAtStartPar

\\
\sphinxhline
\sphinxAtStartPar
{\hyperref[\detokenize{_autosummary/tests.test_unit.test_df:tests.test_unit.test_df.assertRaises}]{\sphinxcrossref{\sphinxcode{\sphinxupquote{assertRaises}}}}}(expected\_exception, *args, **kwargs)
&
\sphinxAtStartPar
Fail unless an exception of class expected\_exception is raised by the callable when invoked with specified positional and keyword arguments.
\\
\sphinxhline
\sphinxAtStartPar
{\hyperref[\detokenize{_autosummary/tests.test_unit.test_df:tests.test_unit.test_df.assertRaisesRegex}]{\sphinxcrossref{\sphinxcode{\sphinxupquote{assertRaisesRegex}}}}}(expected\_exception, ...)
&
\sphinxAtStartPar
Asserts that the message in a raised exception matches a regex.
\\
\sphinxhline
\sphinxAtStartPar
\sphinxcode{\sphinxupquote{assertRaisesRegexp}}(**kwargs)
&
\sphinxAtStartPar

\\
\sphinxhline
\sphinxAtStartPar
{\hyperref[\detokenize{_autosummary/tests.test_unit.test_df:tests.test_unit.test_df.assertRegex}]{\sphinxcrossref{\sphinxcode{\sphinxupquote{assertRegex}}}}}(text, expected\_regex{[}, msg{]})
&
\sphinxAtStartPar
Fail the test unless the text matches the regular expression.
\\
\sphinxhline
\sphinxAtStartPar
\sphinxcode{\sphinxupquote{assertRegexpMatches}}(**kwargs)
&
\sphinxAtStartPar

\\
\sphinxhline
\sphinxAtStartPar
{\hyperref[\detokenize{_autosummary/tests.test_unit.test_df:tests.test_unit.test_df.assertSequenceEqual}]{\sphinxcrossref{\sphinxcode{\sphinxupquote{assertSequenceEqual}}}}}(seq1, seq2{[}, msg, seq\_type{]})
&
\sphinxAtStartPar
An equality assertion for ordered sequences (like lists and tuples).
\\
\sphinxhline
\sphinxAtStartPar
{\hyperref[\detokenize{_autosummary/tests.test_unit.test_df:tests.test_unit.test_df.assertSetEqual}]{\sphinxcrossref{\sphinxcode{\sphinxupquote{assertSetEqual}}}}}(set1, set2{[}, msg{]})
&
\sphinxAtStartPar
A set\sphinxhyphen{}specific equality assertion.
\\
\sphinxhline
\sphinxAtStartPar
{\hyperref[\detokenize{_autosummary/tests.test_unit.test_df:tests.test_unit.test_df.assertTrue}]{\sphinxcrossref{\sphinxcode{\sphinxupquote{assertTrue}}}}}(expr{[}, msg{]})
&
\sphinxAtStartPar
Check that the expression is true.
\\
\sphinxhline
\sphinxAtStartPar
{\hyperref[\detokenize{_autosummary/tests.test_unit.test_df:tests.test_unit.test_df.assertTupleEqual}]{\sphinxcrossref{\sphinxcode{\sphinxupquote{assertTupleEqual}}}}}(tuple1, tuple2{[}, msg{]})
&
\sphinxAtStartPar
A tuple\sphinxhyphen{}specific equality assertion.
\\
\sphinxhline
\sphinxAtStartPar
{\hyperref[\detokenize{_autosummary/tests.test_unit.test_df:tests.test_unit.test_df.assertWarns}]{\sphinxcrossref{\sphinxcode{\sphinxupquote{assertWarns}}}}}(expected\_warning, *args, **kwargs)
&
\sphinxAtStartPar
Fail unless a warning of class warnClass is triggered by the callable when invoked with specified positional and keyword arguments.
\\
\sphinxhline
\sphinxAtStartPar
{\hyperref[\detokenize{_autosummary/tests.test_unit.test_df:tests.test_unit.test_df.assertWarnsRegex}]{\sphinxcrossref{\sphinxcode{\sphinxupquote{assertWarnsRegex}}}}}(expected\_warning, ...)
&
\sphinxAtStartPar
Asserts that the message in a triggered warning matches a regexp.
\\
\sphinxhline
\sphinxAtStartPar
\sphinxcode{\sphinxupquote{assert\_}}(**kwargs)
&
\sphinxAtStartPar

\\
\sphinxhline
\sphinxAtStartPar
\sphinxcode{\sphinxupquote{countTestCases}}()
&
\sphinxAtStartPar

\\
\sphinxhline
\sphinxAtStartPar
{\hyperref[\detokenize{_autosummary/tests.test_unit.test_df:tests.test_unit.test_df.debug}]{\sphinxcrossref{\sphinxcode{\sphinxupquote{debug}}}}}()
&
\sphinxAtStartPar
Run the test without collecting errors in a TestResult
\\
\sphinxhline
\sphinxAtStartPar
\sphinxcode{\sphinxupquote{defaultTestResult}}()
&
\sphinxAtStartPar

\\
\sphinxhline
\sphinxAtStartPar
{\hyperref[\detokenize{_autosummary/tests.test_unit.test_df:tests.test_unit.test_df.doClassCleanups}]{\sphinxcrossref{\sphinxcode{\sphinxupquote{doClassCleanups}}}}}()
&
\sphinxAtStartPar
Execute all class cleanup functions.
\\
\sphinxhline
\sphinxAtStartPar
{\hyperref[\detokenize{_autosummary/tests.test_unit.test_df:tests.test_unit.test_df.doCleanups}]{\sphinxcrossref{\sphinxcode{\sphinxupquote{doCleanups}}}}}()
&
\sphinxAtStartPar
Execute all cleanup functions.
\\
\sphinxhline
\sphinxAtStartPar
{\hyperref[\detokenize{_autosummary/tests.test_unit.test_df:tests.test_unit.test_df.enterClassContext}]{\sphinxcrossref{\sphinxcode{\sphinxupquote{enterClassContext}}}}}(cm)
&
\sphinxAtStartPar
Same as enterContext, but class\sphinxhyphen{}wide.
\\
\sphinxhline
\sphinxAtStartPar
{\hyperref[\detokenize{_autosummary/tests.test_unit.test_df:tests.test_unit.test_df.enterContext}]{\sphinxcrossref{\sphinxcode{\sphinxupquote{enterContext}}}}}(cm)
&
\sphinxAtStartPar
Enters the supplied context manager.
\\
\sphinxhline
\sphinxAtStartPar
{\hyperref[\detokenize{_autosummary/tests.test_unit.test_df:tests.test_unit.test_df.fail}]{\sphinxcrossref{\sphinxcode{\sphinxupquote{fail}}}}}({[}msg{]})
&
\sphinxAtStartPar
Fail immediately, with the given message.
\\
\sphinxhline
\sphinxAtStartPar
\sphinxcode{\sphinxupquote{failIf}}(**kwargs)
&
\sphinxAtStartPar

\\
\sphinxhline
\sphinxAtStartPar
\sphinxcode{\sphinxupquote{failIfAlmostEqual}}(**kwargs)
&
\sphinxAtStartPar

\\
\sphinxhline
\sphinxAtStartPar
\sphinxcode{\sphinxupquote{failIfEqual}}(**kwargs)
&
\sphinxAtStartPar

\\
\sphinxhline
\sphinxAtStartPar
\sphinxcode{\sphinxupquote{failUnless}}(**kwargs)
&
\sphinxAtStartPar

\\
\sphinxhline
\sphinxAtStartPar
\sphinxcode{\sphinxupquote{failUnlessAlmostEqual}}(**kwargs)
&
\sphinxAtStartPar

\\
\sphinxhline
\sphinxAtStartPar
\sphinxcode{\sphinxupquote{failUnlessEqual}}(**kwargs)
&
\sphinxAtStartPar

\\
\sphinxhline
\sphinxAtStartPar
\sphinxcode{\sphinxupquote{failUnlessRaises}}(**kwargs)
&
\sphinxAtStartPar

\\
\sphinxhline
\sphinxAtStartPar
\sphinxcode{\sphinxupquote{id}}()
&
\sphinxAtStartPar

\\
\sphinxhline
\sphinxAtStartPar
\sphinxcode{\sphinxupquote{run}}({[}result{]})
&
\sphinxAtStartPar

\\
\sphinxhline
\sphinxAtStartPar
{\hyperref[\detokenize{_autosummary/tests.test_unit.test_df:tests.test_unit.test_df.setUp}]{\sphinxcrossref{\sphinxcode{\sphinxupquote{setUp}}}}}()
&
\sphinxAtStartPar
Execute everytime before running a test of this class.
\\
\sphinxhline
\sphinxAtStartPar
{\hyperref[\detokenize{_autosummary/tests.test_unit.test_df:tests.test_unit.test_df.setUpClass}]{\sphinxcrossref{\sphinxcode{\sphinxupquote{setUpClass}}}}}()
&
\sphinxAtStartPar
Execute before running tests of this class.
\\
\sphinxhline
\sphinxAtStartPar
{\hyperref[\detokenize{_autosummary/tests.test_unit.test_df:tests.test_unit.test_df.shortDescription}]{\sphinxcrossref{\sphinxcode{\sphinxupquote{shortDescription}}}}}()
&
\sphinxAtStartPar
Returns a one\sphinxhyphen{}line description of the test, or None if no description has been provided.
\\
\sphinxhline
\sphinxAtStartPar
{\hyperref[\detokenize{_autosummary/tests.test_unit.test_df:tests.test_unit.test_df.skipTest}]{\sphinxcrossref{\sphinxcode{\sphinxupquote{skipTest}}}}}(reason)
&
\sphinxAtStartPar
Skip this test.
\\
\sphinxhline
\sphinxAtStartPar
{\hyperref[\detokenize{_autosummary/tests.test_unit.test_df:tests.test_unit.test_df.subTest}]{\sphinxcrossref{\sphinxcode{\sphinxupquote{subTest}}}}}({[}msg{]})
&
\sphinxAtStartPar
Return a context manager that will return the enclosed block of code in a subtest identified by the optional message and keyword parameters.
\\
\sphinxhline
\sphinxAtStartPar
{\hyperref[\detokenize{_autosummary/tests.test_unit.test_df:tests.test_unit.test_df.tearDown}]{\sphinxcrossref{\sphinxcode{\sphinxupquote{tearDown}}}}}()
&
\sphinxAtStartPar
Execute everytime after running a test of this class.
\\
\sphinxhline
\sphinxAtStartPar
{\hyperref[\detokenize{_autosummary/tests.test_unit.test_df:tests.test_unit.test_df.tearDownClass}]{\sphinxcrossref{\sphinxcode{\sphinxupquote{tearDownClass}}}}}()
&
\sphinxAtStartPar
Execute after running all tests of this class.
\\
\sphinxhline
\sphinxAtStartPar
{\hyperref[\detokenize{_autosummary/tests.test_unit.test_df:tests.test_unit.test_df.test_idealstruct}]{\sphinxcrossref{\sphinxcode{\sphinxupquote{test\_idealstruct}}}}}()
&
\sphinxAtStartPar
Test structure of ideal dataset.
\\
\sphinxhline
\sphinxAtStartPar
{\hyperref[\detokenize{_autosummary/tests.test_unit.test_df:tests.test_unit.test_df.test_instancecheck}]{\sphinxcrossref{\sphinxcode{\sphinxupquote{test\_instancecheck}}}}}()
&
\sphinxAtStartPar
Test functionality of datafunctions checktypes
\\
\sphinxhline
\sphinxAtStartPar
{\hyperref[\detokenize{_autosummary/tests.test_unit.test_df:tests.test_unit.test_df.test_teststruct}]{\sphinxcrossref{\sphinxcode{\sphinxupquote{test\_teststruct}}}}}()
&
\sphinxAtStartPar
Test structure of test dataset.
\\
\sphinxhline
\sphinxAtStartPar
{\hyperref[\detokenize{_autosummary/tests.test_unit.test_df:tests.test_unit.test_df.test_trainideal_length}]{\sphinxcrossref{\sphinxcode{\sphinxupquote{test\_trainideal\_length}}}}}()
&
\sphinxAtStartPar
Test if train dataset and ideal dataset are of same length.
\\
\sphinxhline
\sphinxAtStartPar
{\hyperref[\detokenize{_autosummary/tests.test_unit.test_df:tests.test_unit.test_df.test_trainstruct}]{\sphinxcrossref{\sphinxcode{\sphinxupquote{test\_trainstruct}}}}}()
&
\sphinxAtStartPar
Test structure of train dataset.
\\
\sphinxbottomrule
\end{longtable}
\sphinxtableafterendhook
\sphinxatlongtableend
\end{savenotes}
\subsubsection*{Attributes}


\begin{savenotes}\sphinxattablestart
\sphinxthistablewithglobalstyle
\sphinxthistablewithnovlinesstyle
\centering
\begin{tabulary}{\linewidth}[t]{\X{1}{2}\X{1}{2}}
\sphinxtoprule
\sphinxtableatstartofbodyhook
\sphinxAtStartPar
\sphinxcode{\sphinxupquote{longMessage}}
&
\sphinxAtStartPar

\\
\sphinxhline
\sphinxAtStartPar
\sphinxcode{\sphinxupquote{maxDiff}}
&
\sphinxAtStartPar

\\
\sphinxbottomrule
\end{tabulary}
\sphinxtableafterendhook\par
\sphinxattableend\end{savenotes}
\index{addClassCleanup() (tests.test\_unit.test\_df class method)@\spxentry{addClassCleanup()}\spxextra{tests.test\_unit.test\_df class method}}

\begin{fulllineitems}
\phantomsection\label{\detokenize{_autosummary/tests.test_unit.test_df:tests.test_unit.test_df.addClassCleanup}}
\pysigstartsignatures
\pysiglinewithargsret{\sphinxbfcode{\sphinxupquote{classmethod\DUrole{w}{  }}}\sphinxbfcode{\sphinxupquote{addClassCleanup}}}{\emph{\DUrole{n}{function}}, \emph{\DUrole{o}{/}}, \emph{\DUrole{o}{*}\DUrole{n}{args}}, \emph{\DUrole{o}{**}\DUrole{n}{kwargs}}}{}
\pysigstopsignatures
\sphinxAtStartPar
Same as addCleanup, except the cleanup items are called even if
setUpClass fails (unlike tearDownClass).

\end{fulllineitems}

\index{addCleanup() (tests.test\_unit.test\_df method)@\spxentry{addCleanup()}\spxextra{tests.test\_unit.test\_df method}}

\begin{fulllineitems}
\phantomsection\label{\detokenize{_autosummary/tests.test_unit.test_df:tests.test_unit.test_df.addCleanup}}
\pysigstartsignatures
\pysiglinewithargsret{\sphinxbfcode{\sphinxupquote{addCleanup}}}{\emph{\DUrole{n}{function}}, \emph{\DUrole{o}{/}}, \emph{\DUrole{o}{*}\DUrole{n}{args}}, \emph{\DUrole{o}{**}\DUrole{n}{kwargs}}}{}
\pysigstopsignatures
\sphinxAtStartPar
Add a function, with arguments, to be called when the test is
completed. Functions added are called on a LIFO basis and are
called after tearDown on test failure or success.

\sphinxAtStartPar
Cleanup items are called even if setUp fails (unlike tearDown).

\end{fulllineitems}

\index{addTypeEqualityFunc() (tests.test\_unit.test\_df method)@\spxentry{addTypeEqualityFunc()}\spxextra{tests.test\_unit.test\_df method}}

\begin{fulllineitems}
\phantomsection\label{\detokenize{_autosummary/tests.test_unit.test_df:tests.test_unit.test_df.addTypeEqualityFunc}}
\pysigstartsignatures
\pysiglinewithargsret{\sphinxbfcode{\sphinxupquote{addTypeEqualityFunc}}}{\emph{\DUrole{n}{typeobj}}, \emph{\DUrole{n}{function}}}{}
\pysigstopsignatures
\sphinxAtStartPar
Add a type specific assertEqual style function to compare a type.

\sphinxAtStartPar
This method is for use by TestCase subclasses that need to register
their own type equality functions to provide nicer error messages.
\begin{description}
\sphinxlineitem{Args:}\begin{description}
\sphinxlineitem{typeobj: The data type to call this function on when both values}
\sphinxAtStartPar
are of the same type in assertEqual().

\sphinxlineitem{function: The callable taking two arguments and an optional}
\sphinxAtStartPar
msg= argument that raises self.failureException with a
useful error message when the two arguments are not equal.

\end{description}

\end{description}

\end{fulllineitems}

\index{assertAlmostEqual() (tests.test\_unit.test\_df method)@\spxentry{assertAlmostEqual()}\spxextra{tests.test\_unit.test\_df method}}

\begin{fulllineitems}
\phantomsection\label{\detokenize{_autosummary/tests.test_unit.test_df:tests.test_unit.test_df.assertAlmostEqual}}
\pysigstartsignatures
\pysiglinewithargsret{\sphinxbfcode{\sphinxupquote{assertAlmostEqual}}}{\emph{\DUrole{n}{first}}, \emph{\DUrole{n}{second}}, \emph{\DUrole{n}{places}\DUrole{o}{=}\DUrole{default_value}{None}}, \emph{\DUrole{n}{msg}\DUrole{o}{=}\DUrole{default_value}{None}}, \emph{\DUrole{n}{delta}\DUrole{o}{=}\DUrole{default_value}{None}}}{}
\pysigstopsignatures
\sphinxAtStartPar
Fail if the two objects are unequal as determined by their
difference rounded to the given number of decimal places
(default 7) and comparing to zero, or by comparing that the
difference between the two objects is more than the given
delta.

\sphinxAtStartPar
Note that decimal places (from zero) are usually not the same
as significant digits (measured from the most significant digit).

\sphinxAtStartPar
If the two objects compare equal then they will automatically
compare almost equal.

\end{fulllineitems}

\index{assertCountEqual() (tests.test\_unit.test\_df method)@\spxentry{assertCountEqual()}\spxextra{tests.test\_unit.test\_df method}}

\begin{fulllineitems}
\phantomsection\label{\detokenize{_autosummary/tests.test_unit.test_df:tests.test_unit.test_df.assertCountEqual}}
\pysigstartsignatures
\pysiglinewithargsret{\sphinxbfcode{\sphinxupquote{assertCountEqual}}}{\emph{\DUrole{n}{first}}, \emph{\DUrole{n}{second}}, \emph{\DUrole{n}{msg}\DUrole{o}{=}\DUrole{default_value}{None}}}{}
\pysigstopsignatures
\sphinxAtStartPar
Asserts that two iterables have the same elements, the same number of
times, without regard to order.
\begin{quote}
\begin{quote}
\begin{description}
\sphinxlineitem{self.assertEqual(Counter(list(first)),}
\sphinxAtStartPar
Counter(list(second)))

\end{description}
\end{quote}
\begin{description}
\sphinxlineitem{Example:}\begin{itemize}
\item {} 
\sphinxAtStartPar
{[}0, 1, 1{]} and {[}1, 0, 1{]} compare equal.

\item {} 
\sphinxAtStartPar
{[}0, 0, 1{]} and {[}0, 1{]} compare unequal.

\end{itemize}

\end{description}
\end{quote}

\end{fulllineitems}

\index{assertDictContainsSubset() (tests.test\_unit.test\_df method)@\spxentry{assertDictContainsSubset()}\spxextra{tests.test\_unit.test\_df method}}

\begin{fulllineitems}
\phantomsection\label{\detokenize{_autosummary/tests.test_unit.test_df:tests.test_unit.test_df.assertDictContainsSubset}}
\pysigstartsignatures
\pysiglinewithargsret{\sphinxbfcode{\sphinxupquote{assertDictContainsSubset}}}{\emph{\DUrole{n}{subset}}, \emph{\DUrole{n}{dictionary}}, \emph{\DUrole{n}{msg}\DUrole{o}{=}\DUrole{default_value}{None}}}{}
\pysigstopsignatures
\sphinxAtStartPar
Checks whether dictionary is a superset of subset.

\end{fulllineitems}

\index{assertEqual() (tests.test\_unit.test\_df method)@\spxentry{assertEqual()}\spxextra{tests.test\_unit.test\_df method}}

\begin{fulllineitems}
\phantomsection\label{\detokenize{_autosummary/tests.test_unit.test_df:tests.test_unit.test_df.assertEqual}}
\pysigstartsignatures
\pysiglinewithargsret{\sphinxbfcode{\sphinxupquote{assertEqual}}}{\emph{\DUrole{n}{first}}, \emph{\DUrole{n}{second}}, \emph{\DUrole{n}{msg}\DUrole{o}{=}\DUrole{default_value}{None}}}{}
\pysigstopsignatures
\sphinxAtStartPar
Fail if the two objects are unequal as determined by the ‘==’
operator.

\end{fulllineitems}

\index{assertFalse() (tests.test\_unit.test\_df method)@\spxentry{assertFalse()}\spxextra{tests.test\_unit.test\_df method}}

\begin{fulllineitems}
\phantomsection\label{\detokenize{_autosummary/tests.test_unit.test_df:tests.test_unit.test_df.assertFalse}}
\pysigstartsignatures
\pysiglinewithargsret{\sphinxbfcode{\sphinxupquote{assertFalse}}}{\emph{\DUrole{n}{expr}}, \emph{\DUrole{n}{msg}\DUrole{o}{=}\DUrole{default_value}{None}}}{}
\pysigstopsignatures
\sphinxAtStartPar
Check that the expression is false.

\end{fulllineitems}

\index{assertGreater() (tests.test\_unit.test\_df method)@\spxentry{assertGreater()}\spxextra{tests.test\_unit.test\_df method}}

\begin{fulllineitems}
\phantomsection\label{\detokenize{_autosummary/tests.test_unit.test_df:tests.test_unit.test_df.assertGreater}}
\pysigstartsignatures
\pysiglinewithargsret{\sphinxbfcode{\sphinxupquote{assertGreater}}}{\emph{\DUrole{n}{a}}, \emph{\DUrole{n}{b}}, \emph{\DUrole{n}{msg}\DUrole{o}{=}\DUrole{default_value}{None}}}{}
\pysigstopsignatures
\sphinxAtStartPar
Just like self.assertTrue(a \textgreater{} b), but with a nicer default message.

\end{fulllineitems}

\index{assertGreaterEqual() (tests.test\_unit.test\_df method)@\spxentry{assertGreaterEqual()}\spxextra{tests.test\_unit.test\_df method}}

\begin{fulllineitems}
\phantomsection\label{\detokenize{_autosummary/tests.test_unit.test_df:tests.test_unit.test_df.assertGreaterEqual}}
\pysigstartsignatures
\pysiglinewithargsret{\sphinxbfcode{\sphinxupquote{assertGreaterEqual}}}{\emph{\DUrole{n}{a}}, \emph{\DUrole{n}{b}}, \emph{\DUrole{n}{msg}\DUrole{o}{=}\DUrole{default_value}{None}}}{}
\pysigstopsignatures
\sphinxAtStartPar
Just like self.assertTrue(a \textgreater{}= b), but with a nicer default message.

\end{fulllineitems}

\index{assertIn() (tests.test\_unit.test\_df method)@\spxentry{assertIn()}\spxextra{tests.test\_unit.test\_df method}}

\begin{fulllineitems}
\phantomsection\label{\detokenize{_autosummary/tests.test_unit.test_df:tests.test_unit.test_df.assertIn}}
\pysigstartsignatures
\pysiglinewithargsret{\sphinxbfcode{\sphinxupquote{assertIn}}}{\emph{\DUrole{n}{member}}, \emph{\DUrole{n}{container}}, \emph{\DUrole{n}{msg}\DUrole{o}{=}\DUrole{default_value}{None}}}{}
\pysigstopsignatures
\sphinxAtStartPar
Just like self.assertTrue(a in b), but with a nicer default message.

\end{fulllineitems}

\index{assertIs() (tests.test\_unit.test\_df method)@\spxentry{assertIs()}\spxextra{tests.test\_unit.test\_df method}}

\begin{fulllineitems}
\phantomsection\label{\detokenize{_autosummary/tests.test_unit.test_df:tests.test_unit.test_df.assertIs}}
\pysigstartsignatures
\pysiglinewithargsret{\sphinxbfcode{\sphinxupquote{assertIs}}}{\emph{\DUrole{n}{expr1}}, \emph{\DUrole{n}{expr2}}, \emph{\DUrole{n}{msg}\DUrole{o}{=}\DUrole{default_value}{None}}}{}
\pysigstopsignatures
\sphinxAtStartPar
Just like self.assertTrue(a is b), but with a nicer default message.

\end{fulllineitems}

\index{assertIsInstance() (tests.test\_unit.test\_df method)@\spxentry{assertIsInstance()}\spxextra{tests.test\_unit.test\_df method}}

\begin{fulllineitems}
\phantomsection\label{\detokenize{_autosummary/tests.test_unit.test_df:tests.test_unit.test_df.assertIsInstance}}
\pysigstartsignatures
\pysiglinewithargsret{\sphinxbfcode{\sphinxupquote{assertIsInstance}}}{\emph{\DUrole{n}{obj}}, \emph{\DUrole{n}{cls}}, \emph{\DUrole{n}{msg}\DUrole{o}{=}\DUrole{default_value}{None}}}{}
\pysigstopsignatures
\sphinxAtStartPar
Same as self.assertTrue(isinstance(obj, cls)), with a nicer
default message.

\end{fulllineitems}

\index{assertIsNone() (tests.test\_unit.test\_df method)@\spxentry{assertIsNone()}\spxextra{tests.test\_unit.test\_df method}}

\begin{fulllineitems}
\phantomsection\label{\detokenize{_autosummary/tests.test_unit.test_df:tests.test_unit.test_df.assertIsNone}}
\pysigstartsignatures
\pysiglinewithargsret{\sphinxbfcode{\sphinxupquote{assertIsNone}}}{\emph{\DUrole{n}{obj}}, \emph{\DUrole{n}{msg}\DUrole{o}{=}\DUrole{default_value}{None}}}{}
\pysigstopsignatures
\sphinxAtStartPar
Same as self.assertTrue(obj is None), with a nicer default message.

\end{fulllineitems}

\index{assertIsNot() (tests.test\_unit.test\_df method)@\spxentry{assertIsNot()}\spxextra{tests.test\_unit.test\_df method}}

\begin{fulllineitems}
\phantomsection\label{\detokenize{_autosummary/tests.test_unit.test_df:tests.test_unit.test_df.assertIsNot}}
\pysigstartsignatures
\pysiglinewithargsret{\sphinxbfcode{\sphinxupquote{assertIsNot}}}{\emph{\DUrole{n}{expr1}}, \emph{\DUrole{n}{expr2}}, \emph{\DUrole{n}{msg}\DUrole{o}{=}\DUrole{default_value}{None}}}{}
\pysigstopsignatures
\sphinxAtStartPar
Just like self.assertTrue(a is not b), but with a nicer default message.

\end{fulllineitems}

\index{assertIsNotNone() (tests.test\_unit.test\_df method)@\spxentry{assertIsNotNone()}\spxextra{tests.test\_unit.test\_df method}}

\begin{fulllineitems}
\phantomsection\label{\detokenize{_autosummary/tests.test_unit.test_df:tests.test_unit.test_df.assertIsNotNone}}
\pysigstartsignatures
\pysiglinewithargsret{\sphinxbfcode{\sphinxupquote{assertIsNotNone}}}{\emph{\DUrole{n}{obj}}, \emph{\DUrole{n}{msg}\DUrole{o}{=}\DUrole{default_value}{None}}}{}
\pysigstopsignatures
\sphinxAtStartPar
Included for symmetry with assertIsNone.

\end{fulllineitems}

\index{assertLess() (tests.test\_unit.test\_df method)@\spxentry{assertLess()}\spxextra{tests.test\_unit.test\_df method}}

\begin{fulllineitems}
\phantomsection\label{\detokenize{_autosummary/tests.test_unit.test_df:tests.test_unit.test_df.assertLess}}
\pysigstartsignatures
\pysiglinewithargsret{\sphinxbfcode{\sphinxupquote{assertLess}}}{\emph{\DUrole{n}{a}}, \emph{\DUrole{n}{b}}, \emph{\DUrole{n}{msg}\DUrole{o}{=}\DUrole{default_value}{None}}}{}
\pysigstopsignatures
\sphinxAtStartPar
Just like self.assertTrue(a \textless{} b), but with a nicer default message.

\end{fulllineitems}

\index{assertLessEqual() (tests.test\_unit.test\_df method)@\spxentry{assertLessEqual()}\spxextra{tests.test\_unit.test\_df method}}

\begin{fulllineitems}
\phantomsection\label{\detokenize{_autosummary/tests.test_unit.test_df:tests.test_unit.test_df.assertLessEqual}}
\pysigstartsignatures
\pysiglinewithargsret{\sphinxbfcode{\sphinxupquote{assertLessEqual}}}{\emph{\DUrole{n}{a}}, \emph{\DUrole{n}{b}}, \emph{\DUrole{n}{msg}\DUrole{o}{=}\DUrole{default_value}{None}}}{}
\pysigstopsignatures
\sphinxAtStartPar
Just like self.assertTrue(a \textless{}= b), but with a nicer default message.

\end{fulllineitems}

\index{assertListEqual() (tests.test\_unit.test\_df method)@\spxentry{assertListEqual()}\spxextra{tests.test\_unit.test\_df method}}

\begin{fulllineitems}
\phantomsection\label{\detokenize{_autosummary/tests.test_unit.test_df:tests.test_unit.test_df.assertListEqual}}
\pysigstartsignatures
\pysiglinewithargsret{\sphinxbfcode{\sphinxupquote{assertListEqual}}}{\emph{\DUrole{n}{list1}}, \emph{\DUrole{n}{list2}}, \emph{\DUrole{n}{msg}\DUrole{o}{=}\DUrole{default_value}{None}}}{}
\pysigstopsignatures
\sphinxAtStartPar
A list\sphinxhyphen{}specific equality assertion.
\begin{description}
\sphinxlineitem{Args:}
\sphinxAtStartPar
list1: The first list to compare.
list2: The second list to compare.
msg: Optional message to use on failure instead of a list of
\begin{quote}

\sphinxAtStartPar
differences.
\end{quote}

\end{description}

\end{fulllineitems}

\index{assertLogs() (tests.test\_unit.test\_df method)@\spxentry{assertLogs()}\spxextra{tests.test\_unit.test\_df method}}

\begin{fulllineitems}
\phantomsection\label{\detokenize{_autosummary/tests.test_unit.test_df:tests.test_unit.test_df.assertLogs}}
\pysigstartsignatures
\pysiglinewithargsret{\sphinxbfcode{\sphinxupquote{assertLogs}}}{\emph{\DUrole{n}{logger}\DUrole{o}{=}\DUrole{default_value}{None}}, \emph{\DUrole{n}{level}\DUrole{o}{=}\DUrole{default_value}{None}}}{}
\pysigstopsignatures
\sphinxAtStartPar
Fail unless a log message of level \sphinxstyleemphasis{level} or higher is emitted
on \sphinxstyleemphasis{logger\_name} or its children.  If omitted, \sphinxstyleemphasis{level} defaults to
INFO and \sphinxstyleemphasis{logger} defaults to the root logger.

\sphinxAtStartPar
This method must be used as a context manager, and will yield
a recording object with two attributes: \sphinxtitleref{output} and \sphinxtitleref{records}.
At the end of the context manager, the \sphinxtitleref{output} attribute will
be a list of the matching formatted log messages and the
\sphinxtitleref{records} attribute will be a list of the corresponding LogRecord
objects.

\sphinxAtStartPar
Example:

\begin{sphinxVerbatim}[commandchars=\\\{\}]
\PYG{k}{with} \PYG{n+nb+bp}{self}\PYG{o}{.}\PYG{n}{assertLogs}\PYG{p}{(}\PYG{l+s+s1}{\PYGZsq{}}\PYG{l+s+s1}{foo}\PYG{l+s+s1}{\PYGZsq{}}\PYG{p}{,} \PYG{n}{level}\PYG{o}{=}\PYG{l+s+s1}{\PYGZsq{}}\PYG{l+s+s1}{INFO}\PYG{l+s+s1}{\PYGZsq{}}\PYG{p}{)} \PYG{k}{as} \PYG{n}{cm}\PYG{p}{:}
    \PYG{n}{logging}\PYG{o}{.}\PYG{n}{getLogger}\PYG{p}{(}\PYG{l+s+s1}{\PYGZsq{}}\PYG{l+s+s1}{foo}\PYG{l+s+s1}{\PYGZsq{}}\PYG{p}{)}\PYG{o}{.}\PYG{n}{info}\PYG{p}{(}\PYG{l+s+s1}{\PYGZsq{}}\PYG{l+s+s1}{first message}\PYG{l+s+s1}{\PYGZsq{}}\PYG{p}{)}
    \PYG{n}{logging}\PYG{o}{.}\PYG{n}{getLogger}\PYG{p}{(}\PYG{l+s+s1}{\PYGZsq{}}\PYG{l+s+s1}{foo.bar}\PYG{l+s+s1}{\PYGZsq{}}\PYG{p}{)}\PYG{o}{.}\PYG{n}{error}\PYG{p}{(}\PYG{l+s+s1}{\PYGZsq{}}\PYG{l+s+s1}{second message}\PYG{l+s+s1}{\PYGZsq{}}\PYG{p}{)}
\PYG{n+nb+bp}{self}\PYG{o}{.}\PYG{n}{assertEqual}\PYG{p}{(}\PYG{n}{cm}\PYG{o}{.}\PYG{n}{output}\PYG{p}{,} \PYG{p}{[}\PYG{l+s+s1}{\PYGZsq{}}\PYG{l+s+s1}{INFO:foo:first message}\PYG{l+s+s1}{\PYGZsq{}}\PYG{p}{,}
                             \PYG{l+s+s1}{\PYGZsq{}}\PYG{l+s+s1}{ERROR:foo.bar:second message}\PYG{l+s+s1}{\PYGZsq{}}\PYG{p}{]}\PYG{p}{)}
\end{sphinxVerbatim}

\end{fulllineitems}

\index{assertMultiLineEqual() (tests.test\_unit.test\_df method)@\spxentry{assertMultiLineEqual()}\spxextra{tests.test\_unit.test\_df method}}

\begin{fulllineitems}
\phantomsection\label{\detokenize{_autosummary/tests.test_unit.test_df:tests.test_unit.test_df.assertMultiLineEqual}}
\pysigstartsignatures
\pysiglinewithargsret{\sphinxbfcode{\sphinxupquote{assertMultiLineEqual}}}{\emph{\DUrole{n}{first}}, \emph{\DUrole{n}{second}}, \emph{\DUrole{n}{msg}\DUrole{o}{=}\DUrole{default_value}{None}}}{}
\pysigstopsignatures
\sphinxAtStartPar
Assert that two multi\sphinxhyphen{}line strings are equal.

\end{fulllineitems}

\index{assertNoLogs() (tests.test\_unit.test\_df method)@\spxentry{assertNoLogs()}\spxextra{tests.test\_unit.test\_df method}}

\begin{fulllineitems}
\phantomsection\label{\detokenize{_autosummary/tests.test_unit.test_df:tests.test_unit.test_df.assertNoLogs}}
\pysigstartsignatures
\pysiglinewithargsret{\sphinxbfcode{\sphinxupquote{assertNoLogs}}}{\emph{\DUrole{n}{logger}\DUrole{o}{=}\DUrole{default_value}{None}}, \emph{\DUrole{n}{level}\DUrole{o}{=}\DUrole{default_value}{None}}}{}
\pysigstopsignatures
\sphinxAtStartPar
Fail unless no log messages of level \sphinxstyleemphasis{level} or higher are emitted
on \sphinxstyleemphasis{logger\_name} or its children.

\sphinxAtStartPar
This method must be used as a context manager.

\end{fulllineitems}

\index{assertNotAlmostEqual() (tests.test\_unit.test\_df method)@\spxentry{assertNotAlmostEqual()}\spxextra{tests.test\_unit.test\_df method}}

\begin{fulllineitems}
\phantomsection\label{\detokenize{_autosummary/tests.test_unit.test_df:tests.test_unit.test_df.assertNotAlmostEqual}}
\pysigstartsignatures
\pysiglinewithargsret{\sphinxbfcode{\sphinxupquote{assertNotAlmostEqual}}}{\emph{\DUrole{n}{first}}, \emph{\DUrole{n}{second}}, \emph{\DUrole{n}{places}\DUrole{o}{=}\DUrole{default_value}{None}}, \emph{\DUrole{n}{msg}\DUrole{o}{=}\DUrole{default_value}{None}}, \emph{\DUrole{n}{delta}\DUrole{o}{=}\DUrole{default_value}{None}}}{}
\pysigstopsignatures
\sphinxAtStartPar
Fail if the two objects are equal as determined by their
difference rounded to the given number of decimal places
(default 7) and comparing to zero, or by comparing that the
difference between the two objects is less than the given delta.

\sphinxAtStartPar
Note that decimal places (from zero) are usually not the same
as significant digits (measured from the most significant digit).

\sphinxAtStartPar
Objects that are equal automatically fail.

\end{fulllineitems}

\index{assertNotEqual() (tests.test\_unit.test\_df method)@\spxentry{assertNotEqual()}\spxextra{tests.test\_unit.test\_df method}}

\begin{fulllineitems}
\phantomsection\label{\detokenize{_autosummary/tests.test_unit.test_df:tests.test_unit.test_df.assertNotEqual}}
\pysigstartsignatures
\pysiglinewithargsret{\sphinxbfcode{\sphinxupquote{assertNotEqual}}}{\emph{\DUrole{n}{first}}, \emph{\DUrole{n}{second}}, \emph{\DUrole{n}{msg}\DUrole{o}{=}\DUrole{default_value}{None}}}{}
\pysigstopsignatures
\sphinxAtStartPar
Fail if the two objects are equal as determined by the ‘!=’
operator.

\end{fulllineitems}

\index{assertNotIn() (tests.test\_unit.test\_df method)@\spxentry{assertNotIn()}\spxextra{tests.test\_unit.test\_df method}}

\begin{fulllineitems}
\phantomsection\label{\detokenize{_autosummary/tests.test_unit.test_df:tests.test_unit.test_df.assertNotIn}}
\pysigstartsignatures
\pysiglinewithargsret{\sphinxbfcode{\sphinxupquote{assertNotIn}}}{\emph{\DUrole{n}{member}}, \emph{\DUrole{n}{container}}, \emph{\DUrole{n}{msg}\DUrole{o}{=}\DUrole{default_value}{None}}}{}
\pysigstopsignatures
\sphinxAtStartPar
Just like self.assertTrue(a not in b), but with a nicer default message.

\end{fulllineitems}

\index{assertNotIsInstance() (tests.test\_unit.test\_df method)@\spxentry{assertNotIsInstance()}\spxextra{tests.test\_unit.test\_df method}}

\begin{fulllineitems}
\phantomsection\label{\detokenize{_autosummary/tests.test_unit.test_df:tests.test_unit.test_df.assertNotIsInstance}}
\pysigstartsignatures
\pysiglinewithargsret{\sphinxbfcode{\sphinxupquote{assertNotIsInstance}}}{\emph{\DUrole{n}{obj}}, \emph{\DUrole{n}{cls}}, \emph{\DUrole{n}{msg}\DUrole{o}{=}\DUrole{default_value}{None}}}{}
\pysigstopsignatures
\sphinxAtStartPar
Included for symmetry with assertIsInstance.

\end{fulllineitems}

\index{assertNotRegex() (tests.test\_unit.test\_df method)@\spxentry{assertNotRegex()}\spxextra{tests.test\_unit.test\_df method}}

\begin{fulllineitems}
\phantomsection\label{\detokenize{_autosummary/tests.test_unit.test_df:tests.test_unit.test_df.assertNotRegex}}
\pysigstartsignatures
\pysiglinewithargsret{\sphinxbfcode{\sphinxupquote{assertNotRegex}}}{\emph{\DUrole{n}{text}}, \emph{\DUrole{n}{unexpected\_regex}}, \emph{\DUrole{n}{msg}\DUrole{o}{=}\DUrole{default_value}{None}}}{}
\pysigstopsignatures
\sphinxAtStartPar
Fail the test if the text matches the regular expression.

\end{fulllineitems}

\index{assertRaises() (tests.test\_unit.test\_df method)@\spxentry{assertRaises()}\spxextra{tests.test\_unit.test\_df method}}

\begin{fulllineitems}
\phantomsection\label{\detokenize{_autosummary/tests.test_unit.test_df:tests.test_unit.test_df.assertRaises}}
\pysigstartsignatures
\pysiglinewithargsret{\sphinxbfcode{\sphinxupquote{assertRaises}}}{\emph{\DUrole{n}{expected\_exception}}, \emph{\DUrole{o}{*}\DUrole{n}{args}}, \emph{\DUrole{o}{**}\DUrole{n}{kwargs}}}{}
\pysigstopsignatures
\sphinxAtStartPar
Fail unless an exception of class expected\_exception is raised
by the callable when invoked with specified positional and
keyword arguments. If a different type of exception is
raised, it will not be caught, and the test case will be
deemed to have suffered an error, exactly as for an
unexpected exception.

\sphinxAtStartPar
If called with the callable and arguments omitted, will return a
context object used like this:

\begin{sphinxVerbatim}[commandchars=\\\{\}]
\PYG{k}{with} \PYG{n+nb+bp}{self}\PYG{o}{.}\PYG{n}{assertRaises}\PYG{p}{(}\PYG{n}{SomeException}\PYG{p}{)}\PYG{p}{:}
    \PYG{n}{do\PYGZus{}something}\PYG{p}{(}\PYG{p}{)}
\end{sphinxVerbatim}

\sphinxAtStartPar
An optional keyword argument ‘msg’ can be provided when assertRaises
is used as a context object.

\sphinxAtStartPar
The context manager keeps a reference to the exception as
the ‘exception’ attribute. This allows you to inspect the
exception after the assertion:

\begin{sphinxVerbatim}[commandchars=\\\{\}]
\PYG{k}{with} \PYG{n+nb+bp}{self}\PYG{o}{.}\PYG{n}{assertRaises}\PYG{p}{(}\PYG{n}{SomeException}\PYG{p}{)} \PYG{k}{as} \PYG{n}{cm}\PYG{p}{:}
    \PYG{n}{do\PYGZus{}something}\PYG{p}{(}\PYG{p}{)}
\PYG{n}{the\PYGZus{}exception} \PYG{o}{=} \PYG{n}{cm}\PYG{o}{.}\PYG{n}{exception}
\PYG{n+nb+bp}{self}\PYG{o}{.}\PYG{n}{assertEqual}\PYG{p}{(}\PYG{n}{the\PYGZus{}exception}\PYG{o}{.}\PYG{n}{error\PYGZus{}code}\PYG{p}{,} \PYG{l+m+mi}{3}\PYG{p}{)}
\end{sphinxVerbatim}

\end{fulllineitems}

\index{assertRaisesRegex() (tests.test\_unit.test\_df method)@\spxentry{assertRaisesRegex()}\spxextra{tests.test\_unit.test\_df method}}

\begin{fulllineitems}
\phantomsection\label{\detokenize{_autosummary/tests.test_unit.test_df:tests.test_unit.test_df.assertRaisesRegex}}
\pysigstartsignatures
\pysiglinewithargsret{\sphinxbfcode{\sphinxupquote{assertRaisesRegex}}}{\emph{\DUrole{n}{expected\_exception}}, \emph{\DUrole{n}{expected\_regex}}, \emph{\DUrole{o}{*}\DUrole{n}{args}}, \emph{\DUrole{o}{**}\DUrole{n}{kwargs}}}{}
\pysigstopsignatures
\sphinxAtStartPar
Asserts that the message in a raised exception matches a regex.
\begin{description}
\sphinxlineitem{Args:}
\sphinxAtStartPar
expected\_exception: Exception class expected to be raised.
expected\_regex: Regex (re.Pattern object or string) expected
\begin{quote}

\sphinxAtStartPar
to be found in error message.
\end{quote}

\sphinxAtStartPar
args: Function to be called and extra positional args.
kwargs: Extra kwargs.
msg: Optional message used in case of failure. Can only be used
\begin{quote}

\sphinxAtStartPar
when assertRaisesRegex is used as a context manager.
\end{quote}

\end{description}

\end{fulllineitems}

\index{assertRegex() (tests.test\_unit.test\_df method)@\spxentry{assertRegex()}\spxextra{tests.test\_unit.test\_df method}}

\begin{fulllineitems}
\phantomsection\label{\detokenize{_autosummary/tests.test_unit.test_df:tests.test_unit.test_df.assertRegex}}
\pysigstartsignatures
\pysiglinewithargsret{\sphinxbfcode{\sphinxupquote{assertRegex}}}{\emph{\DUrole{n}{text}}, \emph{\DUrole{n}{expected\_regex}}, \emph{\DUrole{n}{msg}\DUrole{o}{=}\DUrole{default_value}{None}}}{}
\pysigstopsignatures
\sphinxAtStartPar
Fail the test unless the text matches the regular expression.

\end{fulllineitems}

\index{assertSequenceEqual() (tests.test\_unit.test\_df method)@\spxentry{assertSequenceEqual()}\spxextra{tests.test\_unit.test\_df method}}

\begin{fulllineitems}
\phantomsection\label{\detokenize{_autosummary/tests.test_unit.test_df:tests.test_unit.test_df.assertSequenceEqual}}
\pysigstartsignatures
\pysiglinewithargsret{\sphinxbfcode{\sphinxupquote{assertSequenceEqual}}}{\emph{\DUrole{n}{seq1}}, \emph{\DUrole{n}{seq2}}, \emph{\DUrole{n}{msg}\DUrole{o}{=}\DUrole{default_value}{None}}, \emph{\DUrole{n}{seq\_type}\DUrole{o}{=}\DUrole{default_value}{None}}}{}
\pysigstopsignatures
\sphinxAtStartPar
An equality assertion for ordered sequences (like lists and tuples).

\sphinxAtStartPar
For the purposes of this function, a valid ordered sequence type is one
which can be indexed, has a length, and has an equality operator.
\begin{description}
\sphinxlineitem{Args:}
\sphinxAtStartPar
seq1: The first sequence to compare.
seq2: The second sequence to compare.
seq\_type: The expected datatype of the sequences, or None if no
\begin{quote}

\sphinxAtStartPar
datatype should be enforced.
\end{quote}
\begin{description}
\sphinxlineitem{msg: Optional message to use on failure instead of a list of}
\sphinxAtStartPar
differences.

\end{description}

\end{description}

\end{fulllineitems}

\index{assertSetEqual() (tests.test\_unit.test\_df method)@\spxentry{assertSetEqual()}\spxextra{tests.test\_unit.test\_df method}}

\begin{fulllineitems}
\phantomsection\label{\detokenize{_autosummary/tests.test_unit.test_df:tests.test_unit.test_df.assertSetEqual}}
\pysigstartsignatures
\pysiglinewithargsret{\sphinxbfcode{\sphinxupquote{assertSetEqual}}}{\emph{\DUrole{n}{set1}}, \emph{\DUrole{n}{set2}}, \emph{\DUrole{n}{msg}\DUrole{o}{=}\DUrole{default_value}{None}}}{}
\pysigstopsignatures
\sphinxAtStartPar
A set\sphinxhyphen{}specific equality assertion.
\begin{description}
\sphinxlineitem{Args:}
\sphinxAtStartPar
set1: The first set to compare.
set2: The second set to compare.
msg: Optional message to use on failure instead of a list of
\begin{quote}

\sphinxAtStartPar
differences.
\end{quote}

\end{description}

\sphinxAtStartPar
assertSetEqual uses ducktyping to support different types of sets, and
is optimized for sets specifically (parameters must support a
difference method).

\end{fulllineitems}

\index{assertTrue() (tests.test\_unit.test\_df method)@\spxentry{assertTrue()}\spxextra{tests.test\_unit.test\_df method}}

\begin{fulllineitems}
\phantomsection\label{\detokenize{_autosummary/tests.test_unit.test_df:tests.test_unit.test_df.assertTrue}}
\pysigstartsignatures
\pysiglinewithargsret{\sphinxbfcode{\sphinxupquote{assertTrue}}}{\emph{\DUrole{n}{expr}}, \emph{\DUrole{n}{msg}\DUrole{o}{=}\DUrole{default_value}{None}}}{}
\pysigstopsignatures
\sphinxAtStartPar
Check that the expression is true.

\end{fulllineitems}

\index{assertTupleEqual() (tests.test\_unit.test\_df method)@\spxentry{assertTupleEqual()}\spxextra{tests.test\_unit.test\_df method}}

\begin{fulllineitems}
\phantomsection\label{\detokenize{_autosummary/tests.test_unit.test_df:tests.test_unit.test_df.assertTupleEqual}}
\pysigstartsignatures
\pysiglinewithargsret{\sphinxbfcode{\sphinxupquote{assertTupleEqual}}}{\emph{\DUrole{n}{tuple1}}, \emph{\DUrole{n}{tuple2}}, \emph{\DUrole{n}{msg}\DUrole{o}{=}\DUrole{default_value}{None}}}{}
\pysigstopsignatures
\sphinxAtStartPar
A tuple\sphinxhyphen{}specific equality assertion.
\begin{description}
\sphinxlineitem{Args:}
\sphinxAtStartPar
tuple1: The first tuple to compare.
tuple2: The second tuple to compare.
msg: Optional message to use on failure instead of a list of
\begin{quote}

\sphinxAtStartPar
differences.
\end{quote}

\end{description}

\end{fulllineitems}

\index{assertWarns() (tests.test\_unit.test\_df method)@\spxentry{assertWarns()}\spxextra{tests.test\_unit.test\_df method}}

\begin{fulllineitems}
\phantomsection\label{\detokenize{_autosummary/tests.test_unit.test_df:tests.test_unit.test_df.assertWarns}}
\pysigstartsignatures
\pysiglinewithargsret{\sphinxbfcode{\sphinxupquote{assertWarns}}}{\emph{\DUrole{n}{expected\_warning}}, \emph{\DUrole{o}{*}\DUrole{n}{args}}, \emph{\DUrole{o}{**}\DUrole{n}{kwargs}}}{}
\pysigstopsignatures
\sphinxAtStartPar
Fail unless a warning of class warnClass is triggered
by the callable when invoked with specified positional and
keyword arguments.  If a different type of warning is
triggered, it will not be handled: depending on the other
warning filtering rules in effect, it might be silenced, printed
out, or raised as an exception.

\sphinxAtStartPar
If called with the callable and arguments omitted, will return a
context object used like this:

\begin{sphinxVerbatim}[commandchars=\\\{\}]
\PYG{k}{with} \PYG{n+nb+bp}{self}\PYG{o}{.}\PYG{n}{assertWarns}\PYG{p}{(}\PYG{n}{SomeWarning}\PYG{p}{)}\PYG{p}{:}
    \PYG{n}{do\PYGZus{}something}\PYG{p}{(}\PYG{p}{)}
\end{sphinxVerbatim}

\sphinxAtStartPar
An optional keyword argument ‘msg’ can be provided when assertWarns
is used as a context object.

\sphinxAtStartPar
The context manager keeps a reference to the first matching
warning as the ‘warning’ attribute; similarly, the ‘filename’
and ‘lineno’ attributes give you information about the line
of Python code from which the warning was triggered.
This allows you to inspect the warning after the assertion:

\begin{sphinxVerbatim}[commandchars=\\\{\}]
\PYG{k}{with} \PYG{n+nb+bp}{self}\PYG{o}{.}\PYG{n}{assertWarns}\PYG{p}{(}\PYG{n}{SomeWarning}\PYG{p}{)} \PYG{k}{as} \PYG{n}{cm}\PYG{p}{:}
    \PYG{n}{do\PYGZus{}something}\PYG{p}{(}\PYG{p}{)}
\PYG{n}{the\PYGZus{}warning} \PYG{o}{=} \PYG{n}{cm}\PYG{o}{.}\PYG{n}{warning}
\PYG{n+nb+bp}{self}\PYG{o}{.}\PYG{n}{assertEqual}\PYG{p}{(}\PYG{n}{the\PYGZus{}warning}\PYG{o}{.}\PYG{n}{some\PYGZus{}attribute}\PYG{p}{,} \PYG{l+m+mi}{147}\PYG{p}{)}
\end{sphinxVerbatim}

\end{fulllineitems}

\index{assertWarnsRegex() (tests.test\_unit.test\_df method)@\spxentry{assertWarnsRegex()}\spxextra{tests.test\_unit.test\_df method}}

\begin{fulllineitems}
\phantomsection\label{\detokenize{_autosummary/tests.test_unit.test_df:tests.test_unit.test_df.assertWarnsRegex}}
\pysigstartsignatures
\pysiglinewithargsret{\sphinxbfcode{\sphinxupquote{assertWarnsRegex}}}{\emph{\DUrole{n}{expected\_warning}}, \emph{\DUrole{n}{expected\_regex}}, \emph{\DUrole{o}{*}\DUrole{n}{args}}, \emph{\DUrole{o}{**}\DUrole{n}{kwargs}}}{}
\pysigstopsignatures
\sphinxAtStartPar
Asserts that the message in a triggered warning matches a regexp.
Basic functioning is similar to assertWarns() with the addition
that only warnings whose messages also match the regular expression
are considered successful matches.
\begin{description}
\sphinxlineitem{Args:}
\sphinxAtStartPar
expected\_warning: Warning class expected to be triggered.
expected\_regex: Regex (re.Pattern object or string) expected
\begin{quote}

\sphinxAtStartPar
to be found in error message.
\end{quote}

\sphinxAtStartPar
args: Function to be called and extra positional args.
kwargs: Extra kwargs.
msg: Optional message used in case of failure. Can only be used
\begin{quote}

\sphinxAtStartPar
when assertWarnsRegex is used as a context manager.
\end{quote}

\end{description}

\end{fulllineitems}

\index{debug() (tests.test\_unit.test\_df method)@\spxentry{debug()}\spxextra{tests.test\_unit.test\_df method}}

\begin{fulllineitems}
\phantomsection\label{\detokenize{_autosummary/tests.test_unit.test_df:tests.test_unit.test_df.debug}}
\pysigstartsignatures
\pysiglinewithargsret{\sphinxbfcode{\sphinxupquote{debug}}}{}{}
\pysigstopsignatures
\sphinxAtStartPar
Run the test without collecting errors in a TestResult

\end{fulllineitems}

\index{doClassCleanups() (tests.test\_unit.test\_df class method)@\spxentry{doClassCleanups()}\spxextra{tests.test\_unit.test\_df class method}}

\begin{fulllineitems}
\phantomsection\label{\detokenize{_autosummary/tests.test_unit.test_df:tests.test_unit.test_df.doClassCleanups}}
\pysigstartsignatures
\pysiglinewithargsret{\sphinxbfcode{\sphinxupquote{classmethod\DUrole{w}{  }}}\sphinxbfcode{\sphinxupquote{doClassCleanups}}}{}{}
\pysigstopsignatures
\sphinxAtStartPar
Execute all class cleanup functions. Normally called for you after
tearDownClass.

\end{fulllineitems}

\index{doCleanups() (tests.test\_unit.test\_df method)@\spxentry{doCleanups()}\spxextra{tests.test\_unit.test\_df method}}

\begin{fulllineitems}
\phantomsection\label{\detokenize{_autosummary/tests.test_unit.test_df:tests.test_unit.test_df.doCleanups}}
\pysigstartsignatures
\pysiglinewithargsret{\sphinxbfcode{\sphinxupquote{doCleanups}}}{}{}
\pysigstopsignatures
\sphinxAtStartPar
Execute all cleanup functions. Normally called for you after
tearDown.

\end{fulllineitems}

\index{enterClassContext() (tests.test\_unit.test\_df class method)@\spxentry{enterClassContext()}\spxextra{tests.test\_unit.test\_df class method}}

\begin{fulllineitems}
\phantomsection\label{\detokenize{_autosummary/tests.test_unit.test_df:tests.test_unit.test_df.enterClassContext}}
\pysigstartsignatures
\pysiglinewithargsret{\sphinxbfcode{\sphinxupquote{classmethod\DUrole{w}{  }}}\sphinxbfcode{\sphinxupquote{enterClassContext}}}{\emph{\DUrole{n}{cm}}}{}
\pysigstopsignatures
\sphinxAtStartPar
Same as enterContext, but class\sphinxhyphen{}wide.

\end{fulllineitems}

\index{enterContext() (tests.test\_unit.test\_df method)@\spxentry{enterContext()}\spxextra{tests.test\_unit.test\_df method}}

\begin{fulllineitems}
\phantomsection\label{\detokenize{_autosummary/tests.test_unit.test_df:tests.test_unit.test_df.enterContext}}
\pysigstartsignatures
\pysiglinewithargsret{\sphinxbfcode{\sphinxupquote{enterContext}}}{\emph{\DUrole{n}{cm}}}{}
\pysigstopsignatures
\sphinxAtStartPar
Enters the supplied context manager.

\sphinxAtStartPar
If successful, also adds its \_\_exit\_\_ method as a cleanup
function and returns the result of the \_\_enter\_\_ method.

\end{fulllineitems}

\index{fail() (tests.test\_unit.test\_df method)@\spxentry{fail()}\spxextra{tests.test\_unit.test\_df method}}

\begin{fulllineitems}
\phantomsection\label{\detokenize{_autosummary/tests.test_unit.test_df:tests.test_unit.test_df.fail}}
\pysigstartsignatures
\pysiglinewithargsret{\sphinxbfcode{\sphinxupquote{fail}}}{\emph{\DUrole{n}{msg}\DUrole{o}{=}\DUrole{default_value}{None}}}{}
\pysigstopsignatures
\sphinxAtStartPar
Fail immediately, with the given message.

\end{fulllineitems}

\index{failureException (tests.test\_unit.test\_df attribute)@\spxentry{failureException}\spxextra{tests.test\_unit.test\_df attribute}}

\begin{fulllineitems}
\phantomsection\label{\detokenize{_autosummary/tests.test_unit.test_df:tests.test_unit.test_df.failureException}}
\pysigstartsignatures
\pysigline{\sphinxbfcode{\sphinxupquote{failureException}}}
\pysigstopsignatures
\sphinxAtStartPar
alias of \sphinxcode{\sphinxupquote{AssertionError}}

\end{fulllineitems}

\index{setUp() (tests.test\_unit.test\_df method)@\spxentry{setUp()}\spxextra{tests.test\_unit.test\_df method}}

\begin{fulllineitems}
\phantomsection\label{\detokenize{_autosummary/tests.test_unit.test_df:tests.test_unit.test_df.setUp}}
\pysigstartsignatures
\pysiglinewithargsret{\sphinxbfcode{\sphinxupquote{setUp}}}{}{}
\pysigstopsignatures
\sphinxAtStartPar
Execute everytime before running a test of this class.

\end{fulllineitems}

\index{setUpClass() (tests.test\_unit.test\_df class method)@\spxentry{setUpClass()}\spxextra{tests.test\_unit.test\_df class method}}

\begin{fulllineitems}
\phantomsection\label{\detokenize{_autosummary/tests.test_unit.test_df:tests.test_unit.test_df.setUpClass}}
\pysigstartsignatures
\pysiglinewithargsret{\sphinxbfcode{\sphinxupquote{classmethod\DUrole{w}{  }}}\sphinxbfcode{\sphinxupquote{setUpClass}}}{}{}
\pysigstopsignatures
\sphinxAtStartPar
Execute before running tests of this class.

\end{fulllineitems}

\index{shortDescription() (tests.test\_unit.test\_df method)@\spxentry{shortDescription()}\spxextra{tests.test\_unit.test\_df method}}

\begin{fulllineitems}
\phantomsection\label{\detokenize{_autosummary/tests.test_unit.test_df:tests.test_unit.test_df.shortDescription}}
\pysigstartsignatures
\pysiglinewithargsret{\sphinxbfcode{\sphinxupquote{shortDescription}}}{}{}
\pysigstopsignatures
\sphinxAtStartPar
Returns a one\sphinxhyphen{}line description of the test, or None if no
description has been provided.

\sphinxAtStartPar
The default implementation of this method returns the first line of
the specified test method’s docstring.

\end{fulllineitems}

\index{skipTest() (tests.test\_unit.test\_df method)@\spxentry{skipTest()}\spxextra{tests.test\_unit.test\_df method}}

\begin{fulllineitems}
\phantomsection\label{\detokenize{_autosummary/tests.test_unit.test_df:tests.test_unit.test_df.skipTest}}
\pysigstartsignatures
\pysiglinewithargsret{\sphinxbfcode{\sphinxupquote{skipTest}}}{\emph{\DUrole{n}{reason}}}{}
\pysigstopsignatures
\sphinxAtStartPar
Skip this test.

\end{fulllineitems}

\index{subTest() (tests.test\_unit.test\_df method)@\spxentry{subTest()}\spxextra{tests.test\_unit.test\_df method}}

\begin{fulllineitems}
\phantomsection\label{\detokenize{_autosummary/tests.test_unit.test_df:tests.test_unit.test_df.subTest}}
\pysigstartsignatures
\pysiglinewithargsret{\sphinxbfcode{\sphinxupquote{subTest}}}{\emph{\DUrole{n}{msg=\textless{}object object\textgreater{}}}, \emph{\DUrole{n}{**params}}}{}
\pysigstopsignatures
\sphinxAtStartPar
Return a context manager that will return the enclosed block
of code in a subtest identified by the optional message and
keyword parameters.  A failure in the subtest marks the test
case as failed but resumes execution at the end of the enclosed
block, allowing further test code to be executed.

\end{fulllineitems}

\index{tearDown() (tests.test\_unit.test\_df method)@\spxentry{tearDown()}\spxextra{tests.test\_unit.test\_df method}}

\begin{fulllineitems}
\phantomsection\label{\detokenize{_autosummary/tests.test_unit.test_df:tests.test_unit.test_df.tearDown}}
\pysigstartsignatures
\pysiglinewithargsret{\sphinxbfcode{\sphinxupquote{tearDown}}}{}{}
\pysigstopsignatures
\sphinxAtStartPar
Execute everytime after running a test of this class.

\end{fulllineitems}

\index{tearDownClass() (tests.test\_unit.test\_df class method)@\spxentry{tearDownClass()}\spxextra{tests.test\_unit.test\_df class method}}

\begin{fulllineitems}
\phantomsection\label{\detokenize{_autosummary/tests.test_unit.test_df:tests.test_unit.test_df.tearDownClass}}
\pysigstartsignatures
\pysiglinewithargsret{\sphinxbfcode{\sphinxupquote{classmethod\DUrole{w}{  }}}\sphinxbfcode{\sphinxupquote{tearDownClass}}}{}{}
\pysigstopsignatures
\sphinxAtStartPar
Execute after running all tests of this class.

\end{fulllineitems}

\index{test\_idealstruct() (tests.test\_unit.test\_df method)@\spxentry{test\_idealstruct()}\spxextra{tests.test\_unit.test\_df method}}

\begin{fulllineitems}
\phantomsection\label{\detokenize{_autosummary/tests.test_unit.test_df:tests.test_unit.test_df.test_idealstruct}}
\pysigstartsignatures
\pysiglinewithargsret{\sphinxbfcode{\sphinxupquote{test\_idealstruct}}}{}{}
\pysigstopsignatures
\sphinxAtStartPar
Test structure of ideal dataset.

\end{fulllineitems}

\index{test\_instancecheck() (tests.test\_unit.test\_df method)@\spxentry{test\_instancecheck()}\spxextra{tests.test\_unit.test\_df method}}

\begin{fulllineitems}
\phantomsection\label{\detokenize{_autosummary/tests.test_unit.test_df:tests.test_unit.test_df.test_instancecheck}}
\pysigstartsignatures
\pysiglinewithargsret{\sphinxbfcode{\sphinxupquote{test\_instancecheck}}}{}{}
\pysigstopsignatures
\sphinxAtStartPar
Test functionality of datafunctions checktypes

\end{fulllineitems}

\index{test\_teststruct() (tests.test\_unit.test\_df method)@\spxentry{test\_teststruct()}\spxextra{tests.test\_unit.test\_df method}}

\begin{fulllineitems}
\phantomsection\label{\detokenize{_autosummary/tests.test_unit.test_df:tests.test_unit.test_df.test_teststruct}}
\pysigstartsignatures
\pysiglinewithargsret{\sphinxbfcode{\sphinxupquote{test\_teststruct}}}{}{}
\pysigstopsignatures
\sphinxAtStartPar
Test structure of test dataset.

\end{fulllineitems}

\index{test\_trainideal\_length() (tests.test\_unit.test\_df method)@\spxentry{test\_trainideal\_length()}\spxextra{tests.test\_unit.test\_df method}}

\begin{fulllineitems}
\phantomsection\label{\detokenize{_autosummary/tests.test_unit.test_df:tests.test_unit.test_df.test_trainideal_length}}
\pysigstartsignatures
\pysiglinewithargsret{\sphinxbfcode{\sphinxupquote{test\_trainideal\_length}}}{}{}
\pysigstopsignatures
\sphinxAtStartPar
Test if train dataset and ideal dataset are of same length.

\sphinxAtStartPar
Same length is needed to execute error calculation by conducting a math
operation to two pandas.Series objects.

\end{fulllineitems}

\index{test\_trainstruct() (tests.test\_unit.test\_df method)@\spxentry{test\_trainstruct()}\spxextra{tests.test\_unit.test\_df method}}

\begin{fulllineitems}
\phantomsection\label{\detokenize{_autosummary/tests.test_unit.test_df:tests.test_unit.test_df.test_trainstruct}}
\pysigstartsignatures
\pysiglinewithargsret{\sphinxbfcode{\sphinxupquote{test\_trainstruct}}}{}{}
\pysigstopsignatures
\sphinxAtStartPar
Test structure of train dataset.

\end{fulllineitems}


\end{fulllineitems}


\sphinxstepscope


\subsection{tests.test\_unit.test\_sqlite}
\label{\detokenize{_autosummary/tests.test_unit.test_sqlite:tests-test-unit-test-sqlite}}\label{\detokenize{_autosummary/tests.test_unit.test_sqlite::doc}}\index{test\_sqlite (class in tests.test\_unit)@\spxentry{test\_sqlite}\spxextra{class in tests.test\_unit}}

\begin{fulllineitems}
\phantomsection\label{\detokenize{_autosummary/tests.test_unit.test_sqlite:tests.test_unit.test_sqlite}}
\pysigstartsignatures
\pysiglinewithargsret{\sphinxbfcode{\sphinxupquote{class\DUrole{w}{  }}}\sphinxcode{\sphinxupquote{tests.test\_unit.}}\sphinxbfcode{\sphinxupquote{test\_sqlite}}}{\emph{\DUrole{n}{methodName}\DUrole{o}{=}\DUrole{default_value}{\textquotesingle{}runTest\textquotesingle{}}}}{}
\pysigstopsignatures
\sphinxAtStartPar
Bases: \sphinxcode{\sphinxupquote{TestCase}}
\index{\_\_init\_\_() (tests.test\_unit.test\_sqlite method)@\spxentry{\_\_init\_\_()}\spxextra{tests.test\_unit.test\_sqlite method}}

\begin{fulllineitems}
\phantomsection\label{\detokenize{_autosummary/tests.test_unit.test_sqlite:tests.test_unit.test_sqlite.__init__}}
\pysigstartsignatures
\pysiglinewithargsret{\sphinxbfcode{\sphinxupquote{\_\_init\_\_}}}{\emph{\DUrole{n}{methodName}\DUrole{o}{=}\DUrole{default_value}{\textquotesingle{}runTest\textquotesingle{}}}}{}
\pysigstopsignatures
\sphinxAtStartPar
Create an instance of the class that will use the named test
method when executed. Raises a ValueError if the instance does
not have a method with the specified name.

\end{fulllineitems}

\subsubsection*{Methods}


\begin{savenotes}
\sphinxatlongtablestart
\sphinxthistablewithglobalstyle
\sphinxthistablewithnovlinesstyle
\begin{longtable}[c]{\X{1}{2}\X{1}{2}}
\sphinxtoprule
\endfirsthead

\multicolumn{2}{c}{\sphinxnorowcolor
    \makebox[0pt]{\sphinxtablecontinued{\tablename\ \thetable{} \textendash{} continued from previous page}}%
}\\
\sphinxtoprule
\endhead

\sphinxbottomrule
\multicolumn{2}{r}{\sphinxnorowcolor
    \makebox[0pt][r]{\sphinxtablecontinued{continues on next page}}%
}\\
\endfoot

\endlastfoot
\sphinxtableatstartofbodyhook

\sphinxAtStartPar
{\hyperref[\detokenize{_autosummary/tests.test_unit.test_sqlite:tests.test_unit.test_sqlite.__init__}]{\sphinxcrossref{\sphinxcode{\sphinxupquote{\_\_init\_\_}}}}}({[}methodName{]})
&
\sphinxAtStartPar
Create an instance of the class that will use the named test method when executed.
\\
\sphinxhline
\sphinxAtStartPar
{\hyperref[\detokenize{_autosummary/tests.test_unit.test_sqlite:tests.test_unit.test_sqlite.addClassCleanup}]{\sphinxcrossref{\sphinxcode{\sphinxupquote{addClassCleanup}}}}}(function, /, *args, **kwargs)
&
\sphinxAtStartPar
Same as addCleanup, except the cleanup items are called even if setUpClass fails (unlike tearDownClass).
\\
\sphinxhline
\sphinxAtStartPar
{\hyperref[\detokenize{_autosummary/tests.test_unit.test_sqlite:tests.test_unit.test_sqlite.addCleanup}]{\sphinxcrossref{\sphinxcode{\sphinxupquote{addCleanup}}}}}(function, /, *args, **kwargs)
&
\sphinxAtStartPar
Add a function, with arguments, to be called when the test is completed.
\\
\sphinxhline
\sphinxAtStartPar
{\hyperref[\detokenize{_autosummary/tests.test_unit.test_sqlite:tests.test_unit.test_sqlite.addTypeEqualityFunc}]{\sphinxcrossref{\sphinxcode{\sphinxupquote{addTypeEqualityFunc}}}}}(typeobj, function)
&
\sphinxAtStartPar
Add a type specific assertEqual style function to compare a type.
\\
\sphinxhline
\sphinxAtStartPar
{\hyperref[\detokenize{_autosummary/tests.test_unit.test_sqlite:tests.test_unit.test_sqlite.assertAlmostEqual}]{\sphinxcrossref{\sphinxcode{\sphinxupquote{assertAlmostEqual}}}}}(first, second{[}, places, ...{]})
&
\sphinxAtStartPar
Fail if the two objects are unequal as determined by their difference rounded to the given number of decimal places (default 7) and comparing to zero, or by comparing that the difference between the two objects is more than the given delta.
\\
\sphinxhline
\sphinxAtStartPar
\sphinxcode{\sphinxupquote{assertAlmostEquals}}(**kwargs)
&
\sphinxAtStartPar

\\
\sphinxhline
\sphinxAtStartPar
{\hyperref[\detokenize{_autosummary/tests.test_unit.test_sqlite:tests.test_unit.test_sqlite.assertCountEqual}]{\sphinxcrossref{\sphinxcode{\sphinxupquote{assertCountEqual}}}}}(first, second{[}, msg{]})
&
\sphinxAtStartPar
Asserts that two iterables have the same elements, the same number of times, without regard to order.
\\
\sphinxhline
\sphinxAtStartPar
{\hyperref[\detokenize{_autosummary/tests.test_unit.test_sqlite:tests.test_unit.test_sqlite.assertDictContainsSubset}]{\sphinxcrossref{\sphinxcode{\sphinxupquote{assertDictContainsSubset}}}}}(subset, dictionary)
&
\sphinxAtStartPar
Checks whether dictionary is a superset of subset.
\\
\sphinxhline
\sphinxAtStartPar
\sphinxcode{\sphinxupquote{assertDictEqual}}(d1, d2{[}, msg{]})
&
\sphinxAtStartPar

\\
\sphinxhline
\sphinxAtStartPar
{\hyperref[\detokenize{_autosummary/tests.test_unit.test_sqlite:tests.test_unit.test_sqlite.assertEqual}]{\sphinxcrossref{\sphinxcode{\sphinxupquote{assertEqual}}}}}(first, second{[}, msg{]})
&
\sphinxAtStartPar
Fail if the two objects are unequal as determined by the \textquotesingle{}==\textquotesingle{} operator.
\\
\sphinxhline
\sphinxAtStartPar
\sphinxcode{\sphinxupquote{assertEquals}}(**kwargs)
&
\sphinxAtStartPar

\\
\sphinxhline
\sphinxAtStartPar
{\hyperref[\detokenize{_autosummary/tests.test_unit.test_sqlite:tests.test_unit.test_sqlite.assertFalse}]{\sphinxcrossref{\sphinxcode{\sphinxupquote{assertFalse}}}}}(expr{[}, msg{]})
&
\sphinxAtStartPar
Check that the expression is false.
\\
\sphinxhline
\sphinxAtStartPar
{\hyperref[\detokenize{_autosummary/tests.test_unit.test_sqlite:tests.test_unit.test_sqlite.assertGreater}]{\sphinxcrossref{\sphinxcode{\sphinxupquote{assertGreater}}}}}(a, b{[}, msg{]})
&
\sphinxAtStartPar
Just like self.assertTrue(a \textgreater{} b), but with a nicer default message.
\\
\sphinxhline
\sphinxAtStartPar
{\hyperref[\detokenize{_autosummary/tests.test_unit.test_sqlite:tests.test_unit.test_sqlite.assertGreaterEqual}]{\sphinxcrossref{\sphinxcode{\sphinxupquote{assertGreaterEqual}}}}}(a, b{[}, msg{]})
&
\sphinxAtStartPar
Just like self.assertTrue(a \textgreater{}= b), but with a nicer default message.
\\
\sphinxhline
\sphinxAtStartPar
{\hyperref[\detokenize{_autosummary/tests.test_unit.test_sqlite:tests.test_unit.test_sqlite.assertIn}]{\sphinxcrossref{\sphinxcode{\sphinxupquote{assertIn}}}}}(member, container{[}, msg{]})
&
\sphinxAtStartPar
Just like self.assertTrue(a in b), but with a nicer default message.
\\
\sphinxhline
\sphinxAtStartPar
{\hyperref[\detokenize{_autosummary/tests.test_unit.test_sqlite:tests.test_unit.test_sqlite.assertIs}]{\sphinxcrossref{\sphinxcode{\sphinxupquote{assertIs}}}}}(expr1, expr2{[}, msg{]})
&
\sphinxAtStartPar
Just like self.assertTrue(a is b), but with a nicer default message.
\\
\sphinxhline
\sphinxAtStartPar
{\hyperref[\detokenize{_autosummary/tests.test_unit.test_sqlite:tests.test_unit.test_sqlite.assertIsInstance}]{\sphinxcrossref{\sphinxcode{\sphinxupquote{assertIsInstance}}}}}(obj, cls{[}, msg{]})
&
\sphinxAtStartPar
Same as self.assertTrue(isinstance(obj, cls)), with a nicer default message.
\\
\sphinxhline
\sphinxAtStartPar
{\hyperref[\detokenize{_autosummary/tests.test_unit.test_sqlite:tests.test_unit.test_sqlite.assertIsNone}]{\sphinxcrossref{\sphinxcode{\sphinxupquote{assertIsNone}}}}}(obj{[}, msg{]})
&
\sphinxAtStartPar
Same as self.assertTrue(obj is None), with a nicer default message.
\\
\sphinxhline
\sphinxAtStartPar
{\hyperref[\detokenize{_autosummary/tests.test_unit.test_sqlite:tests.test_unit.test_sqlite.assertIsNot}]{\sphinxcrossref{\sphinxcode{\sphinxupquote{assertIsNot}}}}}(expr1, expr2{[}, msg{]})
&
\sphinxAtStartPar
Just like self.assertTrue(a is not b), but with a nicer default message.
\\
\sphinxhline
\sphinxAtStartPar
{\hyperref[\detokenize{_autosummary/tests.test_unit.test_sqlite:tests.test_unit.test_sqlite.assertIsNotNone}]{\sphinxcrossref{\sphinxcode{\sphinxupquote{assertIsNotNone}}}}}(obj{[}, msg{]})
&
\sphinxAtStartPar
Included for symmetry with assertIsNone.
\\
\sphinxhline
\sphinxAtStartPar
{\hyperref[\detokenize{_autosummary/tests.test_unit.test_sqlite:tests.test_unit.test_sqlite.assertLess}]{\sphinxcrossref{\sphinxcode{\sphinxupquote{assertLess}}}}}(a, b{[}, msg{]})
&
\sphinxAtStartPar
Just like self.assertTrue(a \textless{} b), but with a nicer default message.
\\
\sphinxhline
\sphinxAtStartPar
{\hyperref[\detokenize{_autosummary/tests.test_unit.test_sqlite:tests.test_unit.test_sqlite.assertLessEqual}]{\sphinxcrossref{\sphinxcode{\sphinxupquote{assertLessEqual}}}}}(a, b{[}, msg{]})
&
\sphinxAtStartPar
Just like self.assertTrue(a \textless{}= b), but with a nicer default message.
\\
\sphinxhline
\sphinxAtStartPar
{\hyperref[\detokenize{_autosummary/tests.test_unit.test_sqlite:tests.test_unit.test_sqlite.assertListEqual}]{\sphinxcrossref{\sphinxcode{\sphinxupquote{assertListEqual}}}}}(list1, list2{[}, msg{]})
&
\sphinxAtStartPar
A list\sphinxhyphen{}specific equality assertion.
\\
\sphinxhline
\sphinxAtStartPar
{\hyperref[\detokenize{_autosummary/tests.test_unit.test_sqlite:tests.test_unit.test_sqlite.assertLogs}]{\sphinxcrossref{\sphinxcode{\sphinxupquote{assertLogs}}}}}({[}logger, level{]})
&
\sphinxAtStartPar
Fail unless a log message of level \sphinxstyleemphasis{level} or higher is emitted on \sphinxstyleemphasis{logger\_name} or its children.
\\
\sphinxhline
\sphinxAtStartPar
{\hyperref[\detokenize{_autosummary/tests.test_unit.test_sqlite:tests.test_unit.test_sqlite.assertMultiLineEqual}]{\sphinxcrossref{\sphinxcode{\sphinxupquote{assertMultiLineEqual}}}}}(first, second{[}, msg{]})
&
\sphinxAtStartPar
Assert that two multi\sphinxhyphen{}line strings are equal.
\\
\sphinxhline
\sphinxAtStartPar
{\hyperref[\detokenize{_autosummary/tests.test_unit.test_sqlite:tests.test_unit.test_sqlite.assertNoLogs}]{\sphinxcrossref{\sphinxcode{\sphinxupquote{assertNoLogs}}}}}({[}logger, level{]})
&
\sphinxAtStartPar
Fail unless no log messages of level \sphinxstyleemphasis{level} or higher are emitted on \sphinxstyleemphasis{logger\_name} or its children.
\\
\sphinxhline
\sphinxAtStartPar
{\hyperref[\detokenize{_autosummary/tests.test_unit.test_sqlite:tests.test_unit.test_sqlite.assertNotAlmostEqual}]{\sphinxcrossref{\sphinxcode{\sphinxupquote{assertNotAlmostEqual}}}}}(first, second{[}, ...{]})
&
\sphinxAtStartPar
Fail if the two objects are equal as determined by their difference rounded to the given number of decimal places (default 7) and comparing to zero, or by comparing that the difference between the two objects is less than the given delta.
\\
\sphinxhline
\sphinxAtStartPar
\sphinxcode{\sphinxupquote{assertNotAlmostEquals}}(**kwargs)
&
\sphinxAtStartPar

\\
\sphinxhline
\sphinxAtStartPar
{\hyperref[\detokenize{_autosummary/tests.test_unit.test_sqlite:tests.test_unit.test_sqlite.assertNotEqual}]{\sphinxcrossref{\sphinxcode{\sphinxupquote{assertNotEqual}}}}}(first, second{[}, msg{]})
&
\sphinxAtStartPar
Fail if the two objects are equal as determined by the \textquotesingle{}!=\textquotesingle{} operator.
\\
\sphinxhline
\sphinxAtStartPar
\sphinxcode{\sphinxupquote{assertNotEquals}}(**kwargs)
&
\sphinxAtStartPar

\\
\sphinxhline
\sphinxAtStartPar
{\hyperref[\detokenize{_autosummary/tests.test_unit.test_sqlite:tests.test_unit.test_sqlite.assertNotIn}]{\sphinxcrossref{\sphinxcode{\sphinxupquote{assertNotIn}}}}}(member, container{[}, msg{]})
&
\sphinxAtStartPar
Just like self.assertTrue(a not in b), but with a nicer default message.
\\
\sphinxhline
\sphinxAtStartPar
{\hyperref[\detokenize{_autosummary/tests.test_unit.test_sqlite:tests.test_unit.test_sqlite.assertNotIsInstance}]{\sphinxcrossref{\sphinxcode{\sphinxupquote{assertNotIsInstance}}}}}(obj, cls{[}, msg{]})
&
\sphinxAtStartPar
Included for symmetry with assertIsInstance.
\\
\sphinxhline
\sphinxAtStartPar
{\hyperref[\detokenize{_autosummary/tests.test_unit.test_sqlite:tests.test_unit.test_sqlite.assertNotRegex}]{\sphinxcrossref{\sphinxcode{\sphinxupquote{assertNotRegex}}}}}(text, unexpected\_regex{[}, msg{]})
&
\sphinxAtStartPar
Fail the test if the text matches the regular expression.
\\
\sphinxhline
\sphinxAtStartPar
\sphinxcode{\sphinxupquote{assertNotRegexpMatches}}(**kwargs)
&
\sphinxAtStartPar

\\
\sphinxhline
\sphinxAtStartPar
{\hyperref[\detokenize{_autosummary/tests.test_unit.test_sqlite:tests.test_unit.test_sqlite.assertRaises}]{\sphinxcrossref{\sphinxcode{\sphinxupquote{assertRaises}}}}}(expected\_exception, *args, **kwargs)
&
\sphinxAtStartPar
Fail unless an exception of class expected\_exception is raised by the callable when invoked with specified positional and keyword arguments.
\\
\sphinxhline
\sphinxAtStartPar
{\hyperref[\detokenize{_autosummary/tests.test_unit.test_sqlite:tests.test_unit.test_sqlite.assertRaisesRegex}]{\sphinxcrossref{\sphinxcode{\sphinxupquote{assertRaisesRegex}}}}}(expected\_exception, ...)
&
\sphinxAtStartPar
Asserts that the message in a raised exception matches a regex.
\\
\sphinxhline
\sphinxAtStartPar
\sphinxcode{\sphinxupquote{assertRaisesRegexp}}(**kwargs)
&
\sphinxAtStartPar

\\
\sphinxhline
\sphinxAtStartPar
{\hyperref[\detokenize{_autosummary/tests.test_unit.test_sqlite:tests.test_unit.test_sqlite.assertRegex}]{\sphinxcrossref{\sphinxcode{\sphinxupquote{assertRegex}}}}}(text, expected\_regex{[}, msg{]})
&
\sphinxAtStartPar
Fail the test unless the text matches the regular expression.
\\
\sphinxhline
\sphinxAtStartPar
\sphinxcode{\sphinxupquote{assertRegexpMatches}}(**kwargs)
&
\sphinxAtStartPar

\\
\sphinxhline
\sphinxAtStartPar
{\hyperref[\detokenize{_autosummary/tests.test_unit.test_sqlite:tests.test_unit.test_sqlite.assertSequenceEqual}]{\sphinxcrossref{\sphinxcode{\sphinxupquote{assertSequenceEqual}}}}}(seq1, seq2{[}, msg, seq\_type{]})
&
\sphinxAtStartPar
An equality assertion for ordered sequences (like lists and tuples).
\\
\sphinxhline
\sphinxAtStartPar
{\hyperref[\detokenize{_autosummary/tests.test_unit.test_sqlite:tests.test_unit.test_sqlite.assertSetEqual}]{\sphinxcrossref{\sphinxcode{\sphinxupquote{assertSetEqual}}}}}(set1, set2{[}, msg{]})
&
\sphinxAtStartPar
A set\sphinxhyphen{}specific equality assertion.
\\
\sphinxhline
\sphinxAtStartPar
{\hyperref[\detokenize{_autosummary/tests.test_unit.test_sqlite:tests.test_unit.test_sqlite.assertTrue}]{\sphinxcrossref{\sphinxcode{\sphinxupquote{assertTrue}}}}}(expr{[}, msg{]})
&
\sphinxAtStartPar
Check that the expression is true.
\\
\sphinxhline
\sphinxAtStartPar
{\hyperref[\detokenize{_autosummary/tests.test_unit.test_sqlite:tests.test_unit.test_sqlite.assertTupleEqual}]{\sphinxcrossref{\sphinxcode{\sphinxupquote{assertTupleEqual}}}}}(tuple1, tuple2{[}, msg{]})
&
\sphinxAtStartPar
A tuple\sphinxhyphen{}specific equality assertion.
\\
\sphinxhline
\sphinxAtStartPar
{\hyperref[\detokenize{_autosummary/tests.test_unit.test_sqlite:tests.test_unit.test_sqlite.assertWarns}]{\sphinxcrossref{\sphinxcode{\sphinxupquote{assertWarns}}}}}(expected\_warning, *args, **kwargs)
&
\sphinxAtStartPar
Fail unless a warning of class warnClass is triggered by the callable when invoked with specified positional and keyword arguments.
\\
\sphinxhline
\sphinxAtStartPar
{\hyperref[\detokenize{_autosummary/tests.test_unit.test_sqlite:tests.test_unit.test_sqlite.assertWarnsRegex}]{\sphinxcrossref{\sphinxcode{\sphinxupquote{assertWarnsRegex}}}}}(expected\_warning, ...)
&
\sphinxAtStartPar
Asserts that the message in a triggered warning matches a regexp.
\\
\sphinxhline
\sphinxAtStartPar
\sphinxcode{\sphinxupquote{assert\_}}(**kwargs)
&
\sphinxAtStartPar

\\
\sphinxhline
\sphinxAtStartPar
\sphinxcode{\sphinxupquote{countTestCases}}()
&
\sphinxAtStartPar

\\
\sphinxhline
\sphinxAtStartPar
{\hyperref[\detokenize{_autosummary/tests.test_unit.test_sqlite:tests.test_unit.test_sqlite.debug}]{\sphinxcrossref{\sphinxcode{\sphinxupquote{debug}}}}}()
&
\sphinxAtStartPar
Run the test without collecting errors in a TestResult
\\
\sphinxhline
\sphinxAtStartPar
\sphinxcode{\sphinxupquote{defaultTestResult}}()
&
\sphinxAtStartPar

\\
\sphinxhline
\sphinxAtStartPar
{\hyperref[\detokenize{_autosummary/tests.test_unit.test_sqlite:tests.test_unit.test_sqlite.doClassCleanups}]{\sphinxcrossref{\sphinxcode{\sphinxupquote{doClassCleanups}}}}}()
&
\sphinxAtStartPar
Execute all class cleanup functions.
\\
\sphinxhline
\sphinxAtStartPar
{\hyperref[\detokenize{_autosummary/tests.test_unit.test_sqlite:tests.test_unit.test_sqlite.doCleanups}]{\sphinxcrossref{\sphinxcode{\sphinxupquote{doCleanups}}}}}()
&
\sphinxAtStartPar
Execute all cleanup functions.
\\
\sphinxhline
\sphinxAtStartPar
{\hyperref[\detokenize{_autosummary/tests.test_unit.test_sqlite:tests.test_unit.test_sqlite.enterClassContext}]{\sphinxcrossref{\sphinxcode{\sphinxupquote{enterClassContext}}}}}(cm)
&
\sphinxAtStartPar
Same as enterContext, but class\sphinxhyphen{}wide.
\\
\sphinxhline
\sphinxAtStartPar
{\hyperref[\detokenize{_autosummary/tests.test_unit.test_sqlite:tests.test_unit.test_sqlite.enterContext}]{\sphinxcrossref{\sphinxcode{\sphinxupquote{enterContext}}}}}(cm)
&
\sphinxAtStartPar
Enters the supplied context manager.
\\
\sphinxhline
\sphinxAtStartPar
{\hyperref[\detokenize{_autosummary/tests.test_unit.test_sqlite:tests.test_unit.test_sqlite.fail}]{\sphinxcrossref{\sphinxcode{\sphinxupquote{fail}}}}}({[}msg{]})
&
\sphinxAtStartPar
Fail immediately, with the given message.
\\
\sphinxhline
\sphinxAtStartPar
\sphinxcode{\sphinxupquote{failIf}}(**kwargs)
&
\sphinxAtStartPar

\\
\sphinxhline
\sphinxAtStartPar
\sphinxcode{\sphinxupquote{failIfAlmostEqual}}(**kwargs)
&
\sphinxAtStartPar

\\
\sphinxhline
\sphinxAtStartPar
\sphinxcode{\sphinxupquote{failIfEqual}}(**kwargs)
&
\sphinxAtStartPar

\\
\sphinxhline
\sphinxAtStartPar
\sphinxcode{\sphinxupquote{failUnless}}(**kwargs)
&
\sphinxAtStartPar

\\
\sphinxhline
\sphinxAtStartPar
\sphinxcode{\sphinxupquote{failUnlessAlmostEqual}}(**kwargs)
&
\sphinxAtStartPar

\\
\sphinxhline
\sphinxAtStartPar
\sphinxcode{\sphinxupquote{failUnlessEqual}}(**kwargs)
&
\sphinxAtStartPar

\\
\sphinxhline
\sphinxAtStartPar
\sphinxcode{\sphinxupquote{failUnlessRaises}}(**kwargs)
&
\sphinxAtStartPar

\\
\sphinxhline
\sphinxAtStartPar
\sphinxcode{\sphinxupquote{id}}()
&
\sphinxAtStartPar

\\
\sphinxhline
\sphinxAtStartPar
\sphinxcode{\sphinxupquote{run}}({[}result{]})
&
\sphinxAtStartPar

\\
\sphinxhline
\sphinxAtStartPar
{\hyperref[\detokenize{_autosummary/tests.test_unit.test_sqlite:tests.test_unit.test_sqlite.setUp}]{\sphinxcrossref{\sphinxcode{\sphinxupquote{setUp}}}}}()
&
\sphinxAtStartPar
Execute everytime before running a test of this class.
\\
\sphinxhline
\sphinxAtStartPar
{\hyperref[\detokenize{_autosummary/tests.test_unit.test_sqlite:tests.test_unit.test_sqlite.setUpClass}]{\sphinxcrossref{\sphinxcode{\sphinxupquote{setUpClass}}}}}()
&
\sphinxAtStartPar
Execute before running tests of this class.
\\
\sphinxhline
\sphinxAtStartPar
{\hyperref[\detokenize{_autosummary/tests.test_unit.test_sqlite:tests.test_unit.test_sqlite.shortDescription}]{\sphinxcrossref{\sphinxcode{\sphinxupquote{shortDescription}}}}}()
&
\sphinxAtStartPar
Returns a one\sphinxhyphen{}line description of the test, or None if no description has been provided.
\\
\sphinxhline
\sphinxAtStartPar
{\hyperref[\detokenize{_autosummary/tests.test_unit.test_sqlite:tests.test_unit.test_sqlite.skipTest}]{\sphinxcrossref{\sphinxcode{\sphinxupquote{skipTest}}}}}(reason)
&
\sphinxAtStartPar
Skip this test.
\\
\sphinxhline
\sphinxAtStartPar
{\hyperref[\detokenize{_autosummary/tests.test_unit.test_sqlite:tests.test_unit.test_sqlite.subTest}]{\sphinxcrossref{\sphinxcode{\sphinxupquote{subTest}}}}}({[}msg{]})
&
\sphinxAtStartPar
Return a context manager that will return the enclosed block of code in a subtest identified by the optional message and keyword parameters.
\\
\sphinxhline
\sphinxAtStartPar
{\hyperref[\detokenize{_autosummary/tests.test_unit.test_sqlite:tests.test_unit.test_sqlite.tearDown}]{\sphinxcrossref{\sphinxcode{\sphinxupquote{tearDown}}}}}()
&
\sphinxAtStartPar
Execute everytime after running a test of this class.
\\
\sphinxhline
\sphinxAtStartPar
{\hyperref[\detokenize{_autosummary/tests.test_unit.test_sqlite:tests.test_unit.test_sqlite.tearDownClass}]{\sphinxcrossref{\sphinxcode{\sphinxupquote{tearDownClass}}}}}()
&
\sphinxAtStartPar
Execute after running  all tests of this class.
\\
\sphinxhline
\sphinxAtStartPar
{\hyperref[\detokenize{_autosummary/tests.test_unit.test_sqlite:tests.test_unit.test_sqlite.test_creation}]{\sphinxcrossref{\sphinxcode{\sphinxupquote{test\_creation}}}}}()
&
\sphinxAtStartPar
Test creation of empty sqlite database.
\\
\sphinxhline
\sphinxAtStartPar
{\hyperref[\detokenize{_autosummary/tests.test_unit.test_sqlite:tests.test_unit.test_sqlite.test_table_from_csv}]{\sphinxcrossref{\sphinxcode{\sphinxupquote{test\_table\_from\_csv}}}}}()
&
\sphinxAtStartPar
Test import of csv files to SQLite using subprocess.
\\
\sphinxhline
\sphinxAtStartPar
{\hyperref[\detokenize{_autosummary/tests.test_unit.test_sqlite:tests.test_unit.test_sqlite.test_table_from_pandas}]{\sphinxcrossref{\sphinxcode{\sphinxupquote{test\_table\_from\_pandas}}}}}()
&
\sphinxAtStartPar
Test import of csv files to SQLite using pandas.
\\
\sphinxbottomrule
\end{longtable}
\sphinxtableafterendhook
\sphinxatlongtableend
\end{savenotes}
\subsubsection*{Attributes}


\begin{savenotes}\sphinxattablestart
\sphinxthistablewithglobalstyle
\sphinxthistablewithnovlinesstyle
\centering
\begin{tabulary}{\linewidth}[t]{\X{1}{2}\X{1}{2}}
\sphinxtoprule
\sphinxtableatstartofbodyhook
\sphinxAtStartPar
\sphinxcode{\sphinxupquote{longMessage}}
&
\sphinxAtStartPar

\\
\sphinxhline
\sphinxAtStartPar
\sphinxcode{\sphinxupquote{maxDiff}}
&
\sphinxAtStartPar

\\
\sphinxbottomrule
\end{tabulary}
\sphinxtableafterendhook\par
\sphinxattableend\end{savenotes}
\index{addClassCleanup() (tests.test\_unit.test\_sqlite class method)@\spxentry{addClassCleanup()}\spxextra{tests.test\_unit.test\_sqlite class method}}

\begin{fulllineitems}
\phantomsection\label{\detokenize{_autosummary/tests.test_unit.test_sqlite:tests.test_unit.test_sqlite.addClassCleanup}}
\pysigstartsignatures
\pysiglinewithargsret{\sphinxbfcode{\sphinxupquote{classmethod\DUrole{w}{  }}}\sphinxbfcode{\sphinxupquote{addClassCleanup}}}{\emph{\DUrole{n}{function}}, \emph{\DUrole{o}{/}}, \emph{\DUrole{o}{*}\DUrole{n}{args}}, \emph{\DUrole{o}{**}\DUrole{n}{kwargs}}}{}
\pysigstopsignatures
\sphinxAtStartPar
Same as addCleanup, except the cleanup items are called even if
setUpClass fails (unlike tearDownClass).

\end{fulllineitems}

\index{addCleanup() (tests.test\_unit.test\_sqlite method)@\spxentry{addCleanup()}\spxextra{tests.test\_unit.test\_sqlite method}}

\begin{fulllineitems}
\phantomsection\label{\detokenize{_autosummary/tests.test_unit.test_sqlite:tests.test_unit.test_sqlite.addCleanup}}
\pysigstartsignatures
\pysiglinewithargsret{\sphinxbfcode{\sphinxupquote{addCleanup}}}{\emph{\DUrole{n}{function}}, \emph{\DUrole{o}{/}}, \emph{\DUrole{o}{*}\DUrole{n}{args}}, \emph{\DUrole{o}{**}\DUrole{n}{kwargs}}}{}
\pysigstopsignatures
\sphinxAtStartPar
Add a function, with arguments, to be called when the test is
completed. Functions added are called on a LIFO basis and are
called after tearDown on test failure or success.

\sphinxAtStartPar
Cleanup items are called even if setUp fails (unlike tearDown).

\end{fulllineitems}

\index{addTypeEqualityFunc() (tests.test\_unit.test\_sqlite method)@\spxentry{addTypeEqualityFunc()}\spxextra{tests.test\_unit.test\_sqlite method}}

\begin{fulllineitems}
\phantomsection\label{\detokenize{_autosummary/tests.test_unit.test_sqlite:tests.test_unit.test_sqlite.addTypeEqualityFunc}}
\pysigstartsignatures
\pysiglinewithargsret{\sphinxbfcode{\sphinxupquote{addTypeEqualityFunc}}}{\emph{\DUrole{n}{typeobj}}, \emph{\DUrole{n}{function}}}{}
\pysigstopsignatures
\sphinxAtStartPar
Add a type specific assertEqual style function to compare a type.

\sphinxAtStartPar
This method is for use by TestCase subclasses that need to register
their own type equality functions to provide nicer error messages.
\begin{description}
\sphinxlineitem{Args:}\begin{description}
\sphinxlineitem{typeobj: The data type to call this function on when both values}
\sphinxAtStartPar
are of the same type in assertEqual().

\sphinxlineitem{function: The callable taking two arguments and an optional}
\sphinxAtStartPar
msg= argument that raises self.failureException with a
useful error message when the two arguments are not equal.

\end{description}

\end{description}

\end{fulllineitems}

\index{assertAlmostEqual() (tests.test\_unit.test\_sqlite method)@\spxentry{assertAlmostEqual()}\spxextra{tests.test\_unit.test\_sqlite method}}

\begin{fulllineitems}
\phantomsection\label{\detokenize{_autosummary/tests.test_unit.test_sqlite:tests.test_unit.test_sqlite.assertAlmostEqual}}
\pysigstartsignatures
\pysiglinewithargsret{\sphinxbfcode{\sphinxupquote{assertAlmostEqual}}}{\emph{\DUrole{n}{first}}, \emph{\DUrole{n}{second}}, \emph{\DUrole{n}{places}\DUrole{o}{=}\DUrole{default_value}{None}}, \emph{\DUrole{n}{msg}\DUrole{o}{=}\DUrole{default_value}{None}}, \emph{\DUrole{n}{delta}\DUrole{o}{=}\DUrole{default_value}{None}}}{}
\pysigstopsignatures
\sphinxAtStartPar
Fail if the two objects are unequal as determined by their
difference rounded to the given number of decimal places
(default 7) and comparing to zero, or by comparing that the
difference between the two objects is more than the given
delta.

\sphinxAtStartPar
Note that decimal places (from zero) are usually not the same
as significant digits (measured from the most significant digit).

\sphinxAtStartPar
If the two objects compare equal then they will automatically
compare almost equal.

\end{fulllineitems}

\index{assertCountEqual() (tests.test\_unit.test\_sqlite method)@\spxentry{assertCountEqual()}\spxextra{tests.test\_unit.test\_sqlite method}}

\begin{fulllineitems}
\phantomsection\label{\detokenize{_autosummary/tests.test_unit.test_sqlite:tests.test_unit.test_sqlite.assertCountEqual}}
\pysigstartsignatures
\pysiglinewithargsret{\sphinxbfcode{\sphinxupquote{assertCountEqual}}}{\emph{\DUrole{n}{first}}, \emph{\DUrole{n}{second}}, \emph{\DUrole{n}{msg}\DUrole{o}{=}\DUrole{default_value}{None}}}{}
\pysigstopsignatures
\sphinxAtStartPar
Asserts that two iterables have the same elements, the same number of
times, without regard to order.
\begin{quote}
\begin{quote}
\begin{description}
\sphinxlineitem{self.assertEqual(Counter(list(first)),}
\sphinxAtStartPar
Counter(list(second)))

\end{description}
\end{quote}
\begin{description}
\sphinxlineitem{Example:}\begin{itemize}
\item {} 
\sphinxAtStartPar
{[}0, 1, 1{]} and {[}1, 0, 1{]} compare equal.

\item {} 
\sphinxAtStartPar
{[}0, 0, 1{]} and {[}0, 1{]} compare unequal.

\end{itemize}

\end{description}
\end{quote}

\end{fulllineitems}

\index{assertDictContainsSubset() (tests.test\_unit.test\_sqlite method)@\spxentry{assertDictContainsSubset()}\spxextra{tests.test\_unit.test\_sqlite method}}

\begin{fulllineitems}
\phantomsection\label{\detokenize{_autosummary/tests.test_unit.test_sqlite:tests.test_unit.test_sqlite.assertDictContainsSubset}}
\pysigstartsignatures
\pysiglinewithargsret{\sphinxbfcode{\sphinxupquote{assertDictContainsSubset}}}{\emph{\DUrole{n}{subset}}, \emph{\DUrole{n}{dictionary}}, \emph{\DUrole{n}{msg}\DUrole{o}{=}\DUrole{default_value}{None}}}{}
\pysigstopsignatures
\sphinxAtStartPar
Checks whether dictionary is a superset of subset.

\end{fulllineitems}

\index{assertEqual() (tests.test\_unit.test\_sqlite method)@\spxentry{assertEqual()}\spxextra{tests.test\_unit.test\_sqlite method}}

\begin{fulllineitems}
\phantomsection\label{\detokenize{_autosummary/tests.test_unit.test_sqlite:tests.test_unit.test_sqlite.assertEqual}}
\pysigstartsignatures
\pysiglinewithargsret{\sphinxbfcode{\sphinxupquote{assertEqual}}}{\emph{\DUrole{n}{first}}, \emph{\DUrole{n}{second}}, \emph{\DUrole{n}{msg}\DUrole{o}{=}\DUrole{default_value}{None}}}{}
\pysigstopsignatures
\sphinxAtStartPar
Fail if the two objects are unequal as determined by the ‘==’
operator.

\end{fulllineitems}

\index{assertFalse() (tests.test\_unit.test\_sqlite method)@\spxentry{assertFalse()}\spxextra{tests.test\_unit.test\_sqlite method}}

\begin{fulllineitems}
\phantomsection\label{\detokenize{_autosummary/tests.test_unit.test_sqlite:tests.test_unit.test_sqlite.assertFalse}}
\pysigstartsignatures
\pysiglinewithargsret{\sphinxbfcode{\sphinxupquote{assertFalse}}}{\emph{\DUrole{n}{expr}}, \emph{\DUrole{n}{msg}\DUrole{o}{=}\DUrole{default_value}{None}}}{}
\pysigstopsignatures
\sphinxAtStartPar
Check that the expression is false.

\end{fulllineitems}

\index{assertGreater() (tests.test\_unit.test\_sqlite method)@\spxentry{assertGreater()}\spxextra{tests.test\_unit.test\_sqlite method}}

\begin{fulllineitems}
\phantomsection\label{\detokenize{_autosummary/tests.test_unit.test_sqlite:tests.test_unit.test_sqlite.assertGreater}}
\pysigstartsignatures
\pysiglinewithargsret{\sphinxbfcode{\sphinxupquote{assertGreater}}}{\emph{\DUrole{n}{a}}, \emph{\DUrole{n}{b}}, \emph{\DUrole{n}{msg}\DUrole{o}{=}\DUrole{default_value}{None}}}{}
\pysigstopsignatures
\sphinxAtStartPar
Just like self.assertTrue(a \textgreater{} b), but with a nicer default message.

\end{fulllineitems}

\index{assertGreaterEqual() (tests.test\_unit.test\_sqlite method)@\spxentry{assertGreaterEqual()}\spxextra{tests.test\_unit.test\_sqlite method}}

\begin{fulllineitems}
\phantomsection\label{\detokenize{_autosummary/tests.test_unit.test_sqlite:tests.test_unit.test_sqlite.assertGreaterEqual}}
\pysigstartsignatures
\pysiglinewithargsret{\sphinxbfcode{\sphinxupquote{assertGreaterEqual}}}{\emph{\DUrole{n}{a}}, \emph{\DUrole{n}{b}}, \emph{\DUrole{n}{msg}\DUrole{o}{=}\DUrole{default_value}{None}}}{}
\pysigstopsignatures
\sphinxAtStartPar
Just like self.assertTrue(a \textgreater{}= b), but with a nicer default message.

\end{fulllineitems}

\index{assertIn() (tests.test\_unit.test\_sqlite method)@\spxentry{assertIn()}\spxextra{tests.test\_unit.test\_sqlite method}}

\begin{fulllineitems}
\phantomsection\label{\detokenize{_autosummary/tests.test_unit.test_sqlite:tests.test_unit.test_sqlite.assertIn}}
\pysigstartsignatures
\pysiglinewithargsret{\sphinxbfcode{\sphinxupquote{assertIn}}}{\emph{\DUrole{n}{member}}, \emph{\DUrole{n}{container}}, \emph{\DUrole{n}{msg}\DUrole{o}{=}\DUrole{default_value}{None}}}{}
\pysigstopsignatures
\sphinxAtStartPar
Just like self.assertTrue(a in b), but with a nicer default message.

\end{fulllineitems}

\index{assertIs() (tests.test\_unit.test\_sqlite method)@\spxentry{assertIs()}\spxextra{tests.test\_unit.test\_sqlite method}}

\begin{fulllineitems}
\phantomsection\label{\detokenize{_autosummary/tests.test_unit.test_sqlite:tests.test_unit.test_sqlite.assertIs}}
\pysigstartsignatures
\pysiglinewithargsret{\sphinxbfcode{\sphinxupquote{assertIs}}}{\emph{\DUrole{n}{expr1}}, \emph{\DUrole{n}{expr2}}, \emph{\DUrole{n}{msg}\DUrole{o}{=}\DUrole{default_value}{None}}}{}
\pysigstopsignatures
\sphinxAtStartPar
Just like self.assertTrue(a is b), but with a nicer default message.

\end{fulllineitems}

\index{assertIsInstance() (tests.test\_unit.test\_sqlite method)@\spxentry{assertIsInstance()}\spxextra{tests.test\_unit.test\_sqlite method}}

\begin{fulllineitems}
\phantomsection\label{\detokenize{_autosummary/tests.test_unit.test_sqlite:tests.test_unit.test_sqlite.assertIsInstance}}
\pysigstartsignatures
\pysiglinewithargsret{\sphinxbfcode{\sphinxupquote{assertIsInstance}}}{\emph{\DUrole{n}{obj}}, \emph{\DUrole{n}{cls}}, \emph{\DUrole{n}{msg}\DUrole{o}{=}\DUrole{default_value}{None}}}{}
\pysigstopsignatures
\sphinxAtStartPar
Same as self.assertTrue(isinstance(obj, cls)), with a nicer
default message.

\end{fulllineitems}

\index{assertIsNone() (tests.test\_unit.test\_sqlite method)@\spxentry{assertIsNone()}\spxextra{tests.test\_unit.test\_sqlite method}}

\begin{fulllineitems}
\phantomsection\label{\detokenize{_autosummary/tests.test_unit.test_sqlite:tests.test_unit.test_sqlite.assertIsNone}}
\pysigstartsignatures
\pysiglinewithargsret{\sphinxbfcode{\sphinxupquote{assertIsNone}}}{\emph{\DUrole{n}{obj}}, \emph{\DUrole{n}{msg}\DUrole{o}{=}\DUrole{default_value}{None}}}{}
\pysigstopsignatures
\sphinxAtStartPar
Same as self.assertTrue(obj is None), with a nicer default message.

\end{fulllineitems}

\index{assertIsNot() (tests.test\_unit.test\_sqlite method)@\spxentry{assertIsNot()}\spxextra{tests.test\_unit.test\_sqlite method}}

\begin{fulllineitems}
\phantomsection\label{\detokenize{_autosummary/tests.test_unit.test_sqlite:tests.test_unit.test_sqlite.assertIsNot}}
\pysigstartsignatures
\pysiglinewithargsret{\sphinxbfcode{\sphinxupquote{assertIsNot}}}{\emph{\DUrole{n}{expr1}}, \emph{\DUrole{n}{expr2}}, \emph{\DUrole{n}{msg}\DUrole{o}{=}\DUrole{default_value}{None}}}{}
\pysigstopsignatures
\sphinxAtStartPar
Just like self.assertTrue(a is not b), but with a nicer default message.

\end{fulllineitems}

\index{assertIsNotNone() (tests.test\_unit.test\_sqlite method)@\spxentry{assertIsNotNone()}\spxextra{tests.test\_unit.test\_sqlite method}}

\begin{fulllineitems}
\phantomsection\label{\detokenize{_autosummary/tests.test_unit.test_sqlite:tests.test_unit.test_sqlite.assertIsNotNone}}
\pysigstartsignatures
\pysiglinewithargsret{\sphinxbfcode{\sphinxupquote{assertIsNotNone}}}{\emph{\DUrole{n}{obj}}, \emph{\DUrole{n}{msg}\DUrole{o}{=}\DUrole{default_value}{None}}}{}
\pysigstopsignatures
\sphinxAtStartPar
Included for symmetry with assertIsNone.

\end{fulllineitems}

\index{assertLess() (tests.test\_unit.test\_sqlite method)@\spxentry{assertLess()}\spxextra{tests.test\_unit.test\_sqlite method}}

\begin{fulllineitems}
\phantomsection\label{\detokenize{_autosummary/tests.test_unit.test_sqlite:tests.test_unit.test_sqlite.assertLess}}
\pysigstartsignatures
\pysiglinewithargsret{\sphinxbfcode{\sphinxupquote{assertLess}}}{\emph{\DUrole{n}{a}}, \emph{\DUrole{n}{b}}, \emph{\DUrole{n}{msg}\DUrole{o}{=}\DUrole{default_value}{None}}}{}
\pysigstopsignatures
\sphinxAtStartPar
Just like self.assertTrue(a \textless{} b), but with a nicer default message.

\end{fulllineitems}

\index{assertLessEqual() (tests.test\_unit.test\_sqlite method)@\spxentry{assertLessEqual()}\spxextra{tests.test\_unit.test\_sqlite method}}

\begin{fulllineitems}
\phantomsection\label{\detokenize{_autosummary/tests.test_unit.test_sqlite:tests.test_unit.test_sqlite.assertLessEqual}}
\pysigstartsignatures
\pysiglinewithargsret{\sphinxbfcode{\sphinxupquote{assertLessEqual}}}{\emph{\DUrole{n}{a}}, \emph{\DUrole{n}{b}}, \emph{\DUrole{n}{msg}\DUrole{o}{=}\DUrole{default_value}{None}}}{}
\pysigstopsignatures
\sphinxAtStartPar
Just like self.assertTrue(a \textless{}= b), but with a nicer default message.

\end{fulllineitems}

\index{assertListEqual() (tests.test\_unit.test\_sqlite method)@\spxentry{assertListEqual()}\spxextra{tests.test\_unit.test\_sqlite method}}

\begin{fulllineitems}
\phantomsection\label{\detokenize{_autosummary/tests.test_unit.test_sqlite:tests.test_unit.test_sqlite.assertListEqual}}
\pysigstartsignatures
\pysiglinewithargsret{\sphinxbfcode{\sphinxupquote{assertListEqual}}}{\emph{\DUrole{n}{list1}}, \emph{\DUrole{n}{list2}}, \emph{\DUrole{n}{msg}\DUrole{o}{=}\DUrole{default_value}{None}}}{}
\pysigstopsignatures
\sphinxAtStartPar
A list\sphinxhyphen{}specific equality assertion.
\begin{description}
\sphinxlineitem{Args:}
\sphinxAtStartPar
list1: The first list to compare.
list2: The second list to compare.
msg: Optional message to use on failure instead of a list of
\begin{quote}

\sphinxAtStartPar
differences.
\end{quote}

\end{description}

\end{fulllineitems}

\index{assertLogs() (tests.test\_unit.test\_sqlite method)@\spxentry{assertLogs()}\spxextra{tests.test\_unit.test\_sqlite method}}

\begin{fulllineitems}
\phantomsection\label{\detokenize{_autosummary/tests.test_unit.test_sqlite:tests.test_unit.test_sqlite.assertLogs}}
\pysigstartsignatures
\pysiglinewithargsret{\sphinxbfcode{\sphinxupquote{assertLogs}}}{\emph{\DUrole{n}{logger}\DUrole{o}{=}\DUrole{default_value}{None}}, \emph{\DUrole{n}{level}\DUrole{o}{=}\DUrole{default_value}{None}}}{}
\pysigstopsignatures
\sphinxAtStartPar
Fail unless a log message of level \sphinxstyleemphasis{level} or higher is emitted
on \sphinxstyleemphasis{logger\_name} or its children.  If omitted, \sphinxstyleemphasis{level} defaults to
INFO and \sphinxstyleemphasis{logger} defaults to the root logger.

\sphinxAtStartPar
This method must be used as a context manager, and will yield
a recording object with two attributes: \sphinxtitleref{output} and \sphinxtitleref{records}.
At the end of the context manager, the \sphinxtitleref{output} attribute will
be a list of the matching formatted log messages and the
\sphinxtitleref{records} attribute will be a list of the corresponding LogRecord
objects.

\sphinxAtStartPar
Example:

\begin{sphinxVerbatim}[commandchars=\\\{\}]
\PYG{k}{with} \PYG{n+nb+bp}{self}\PYG{o}{.}\PYG{n}{assertLogs}\PYG{p}{(}\PYG{l+s+s1}{\PYGZsq{}}\PYG{l+s+s1}{foo}\PYG{l+s+s1}{\PYGZsq{}}\PYG{p}{,} \PYG{n}{level}\PYG{o}{=}\PYG{l+s+s1}{\PYGZsq{}}\PYG{l+s+s1}{INFO}\PYG{l+s+s1}{\PYGZsq{}}\PYG{p}{)} \PYG{k}{as} \PYG{n}{cm}\PYG{p}{:}
    \PYG{n}{logging}\PYG{o}{.}\PYG{n}{getLogger}\PYG{p}{(}\PYG{l+s+s1}{\PYGZsq{}}\PYG{l+s+s1}{foo}\PYG{l+s+s1}{\PYGZsq{}}\PYG{p}{)}\PYG{o}{.}\PYG{n}{info}\PYG{p}{(}\PYG{l+s+s1}{\PYGZsq{}}\PYG{l+s+s1}{first message}\PYG{l+s+s1}{\PYGZsq{}}\PYG{p}{)}
    \PYG{n}{logging}\PYG{o}{.}\PYG{n}{getLogger}\PYG{p}{(}\PYG{l+s+s1}{\PYGZsq{}}\PYG{l+s+s1}{foo.bar}\PYG{l+s+s1}{\PYGZsq{}}\PYG{p}{)}\PYG{o}{.}\PYG{n}{error}\PYG{p}{(}\PYG{l+s+s1}{\PYGZsq{}}\PYG{l+s+s1}{second message}\PYG{l+s+s1}{\PYGZsq{}}\PYG{p}{)}
\PYG{n+nb+bp}{self}\PYG{o}{.}\PYG{n}{assertEqual}\PYG{p}{(}\PYG{n}{cm}\PYG{o}{.}\PYG{n}{output}\PYG{p}{,} \PYG{p}{[}\PYG{l+s+s1}{\PYGZsq{}}\PYG{l+s+s1}{INFO:foo:first message}\PYG{l+s+s1}{\PYGZsq{}}\PYG{p}{,}
                             \PYG{l+s+s1}{\PYGZsq{}}\PYG{l+s+s1}{ERROR:foo.bar:second message}\PYG{l+s+s1}{\PYGZsq{}}\PYG{p}{]}\PYG{p}{)}
\end{sphinxVerbatim}

\end{fulllineitems}

\index{assertMultiLineEqual() (tests.test\_unit.test\_sqlite method)@\spxentry{assertMultiLineEqual()}\spxextra{tests.test\_unit.test\_sqlite method}}

\begin{fulllineitems}
\phantomsection\label{\detokenize{_autosummary/tests.test_unit.test_sqlite:tests.test_unit.test_sqlite.assertMultiLineEqual}}
\pysigstartsignatures
\pysiglinewithargsret{\sphinxbfcode{\sphinxupquote{assertMultiLineEqual}}}{\emph{\DUrole{n}{first}}, \emph{\DUrole{n}{second}}, \emph{\DUrole{n}{msg}\DUrole{o}{=}\DUrole{default_value}{None}}}{}
\pysigstopsignatures
\sphinxAtStartPar
Assert that two multi\sphinxhyphen{}line strings are equal.

\end{fulllineitems}

\index{assertNoLogs() (tests.test\_unit.test\_sqlite method)@\spxentry{assertNoLogs()}\spxextra{tests.test\_unit.test\_sqlite method}}

\begin{fulllineitems}
\phantomsection\label{\detokenize{_autosummary/tests.test_unit.test_sqlite:tests.test_unit.test_sqlite.assertNoLogs}}
\pysigstartsignatures
\pysiglinewithargsret{\sphinxbfcode{\sphinxupquote{assertNoLogs}}}{\emph{\DUrole{n}{logger}\DUrole{o}{=}\DUrole{default_value}{None}}, \emph{\DUrole{n}{level}\DUrole{o}{=}\DUrole{default_value}{None}}}{}
\pysigstopsignatures
\sphinxAtStartPar
Fail unless no log messages of level \sphinxstyleemphasis{level} or higher are emitted
on \sphinxstyleemphasis{logger\_name} or its children.

\sphinxAtStartPar
This method must be used as a context manager.

\end{fulllineitems}

\index{assertNotAlmostEqual() (tests.test\_unit.test\_sqlite method)@\spxentry{assertNotAlmostEqual()}\spxextra{tests.test\_unit.test\_sqlite method}}

\begin{fulllineitems}
\phantomsection\label{\detokenize{_autosummary/tests.test_unit.test_sqlite:tests.test_unit.test_sqlite.assertNotAlmostEqual}}
\pysigstartsignatures
\pysiglinewithargsret{\sphinxbfcode{\sphinxupquote{assertNotAlmostEqual}}}{\emph{\DUrole{n}{first}}, \emph{\DUrole{n}{second}}, \emph{\DUrole{n}{places}\DUrole{o}{=}\DUrole{default_value}{None}}, \emph{\DUrole{n}{msg}\DUrole{o}{=}\DUrole{default_value}{None}}, \emph{\DUrole{n}{delta}\DUrole{o}{=}\DUrole{default_value}{None}}}{}
\pysigstopsignatures
\sphinxAtStartPar
Fail if the two objects are equal as determined by their
difference rounded to the given number of decimal places
(default 7) and comparing to zero, or by comparing that the
difference between the two objects is less than the given delta.

\sphinxAtStartPar
Note that decimal places (from zero) are usually not the same
as significant digits (measured from the most significant digit).

\sphinxAtStartPar
Objects that are equal automatically fail.

\end{fulllineitems}

\index{assertNotEqual() (tests.test\_unit.test\_sqlite method)@\spxentry{assertNotEqual()}\spxextra{tests.test\_unit.test\_sqlite method}}

\begin{fulllineitems}
\phantomsection\label{\detokenize{_autosummary/tests.test_unit.test_sqlite:tests.test_unit.test_sqlite.assertNotEqual}}
\pysigstartsignatures
\pysiglinewithargsret{\sphinxbfcode{\sphinxupquote{assertNotEqual}}}{\emph{\DUrole{n}{first}}, \emph{\DUrole{n}{second}}, \emph{\DUrole{n}{msg}\DUrole{o}{=}\DUrole{default_value}{None}}}{}
\pysigstopsignatures
\sphinxAtStartPar
Fail if the two objects are equal as determined by the ‘!=’
operator.

\end{fulllineitems}

\index{assertNotIn() (tests.test\_unit.test\_sqlite method)@\spxentry{assertNotIn()}\spxextra{tests.test\_unit.test\_sqlite method}}

\begin{fulllineitems}
\phantomsection\label{\detokenize{_autosummary/tests.test_unit.test_sqlite:tests.test_unit.test_sqlite.assertNotIn}}
\pysigstartsignatures
\pysiglinewithargsret{\sphinxbfcode{\sphinxupquote{assertNotIn}}}{\emph{\DUrole{n}{member}}, \emph{\DUrole{n}{container}}, \emph{\DUrole{n}{msg}\DUrole{o}{=}\DUrole{default_value}{None}}}{}
\pysigstopsignatures
\sphinxAtStartPar
Just like self.assertTrue(a not in b), but with a nicer default message.

\end{fulllineitems}

\index{assertNotIsInstance() (tests.test\_unit.test\_sqlite method)@\spxentry{assertNotIsInstance()}\spxextra{tests.test\_unit.test\_sqlite method}}

\begin{fulllineitems}
\phantomsection\label{\detokenize{_autosummary/tests.test_unit.test_sqlite:tests.test_unit.test_sqlite.assertNotIsInstance}}
\pysigstartsignatures
\pysiglinewithargsret{\sphinxbfcode{\sphinxupquote{assertNotIsInstance}}}{\emph{\DUrole{n}{obj}}, \emph{\DUrole{n}{cls}}, \emph{\DUrole{n}{msg}\DUrole{o}{=}\DUrole{default_value}{None}}}{}
\pysigstopsignatures
\sphinxAtStartPar
Included for symmetry with assertIsInstance.

\end{fulllineitems}

\index{assertNotRegex() (tests.test\_unit.test\_sqlite method)@\spxentry{assertNotRegex()}\spxextra{tests.test\_unit.test\_sqlite method}}

\begin{fulllineitems}
\phantomsection\label{\detokenize{_autosummary/tests.test_unit.test_sqlite:tests.test_unit.test_sqlite.assertNotRegex}}
\pysigstartsignatures
\pysiglinewithargsret{\sphinxbfcode{\sphinxupquote{assertNotRegex}}}{\emph{\DUrole{n}{text}}, \emph{\DUrole{n}{unexpected\_regex}}, \emph{\DUrole{n}{msg}\DUrole{o}{=}\DUrole{default_value}{None}}}{}
\pysigstopsignatures
\sphinxAtStartPar
Fail the test if the text matches the regular expression.

\end{fulllineitems}

\index{assertRaises() (tests.test\_unit.test\_sqlite method)@\spxentry{assertRaises()}\spxextra{tests.test\_unit.test\_sqlite method}}

\begin{fulllineitems}
\phantomsection\label{\detokenize{_autosummary/tests.test_unit.test_sqlite:tests.test_unit.test_sqlite.assertRaises}}
\pysigstartsignatures
\pysiglinewithargsret{\sphinxbfcode{\sphinxupquote{assertRaises}}}{\emph{\DUrole{n}{expected\_exception}}, \emph{\DUrole{o}{*}\DUrole{n}{args}}, \emph{\DUrole{o}{**}\DUrole{n}{kwargs}}}{}
\pysigstopsignatures
\sphinxAtStartPar
Fail unless an exception of class expected\_exception is raised
by the callable when invoked with specified positional and
keyword arguments. If a different type of exception is
raised, it will not be caught, and the test case will be
deemed to have suffered an error, exactly as for an
unexpected exception.

\sphinxAtStartPar
If called with the callable and arguments omitted, will return a
context object used like this:

\begin{sphinxVerbatim}[commandchars=\\\{\}]
\PYG{k}{with} \PYG{n+nb+bp}{self}\PYG{o}{.}\PYG{n}{assertRaises}\PYG{p}{(}\PYG{n}{SomeException}\PYG{p}{)}\PYG{p}{:}
    \PYG{n}{do\PYGZus{}something}\PYG{p}{(}\PYG{p}{)}
\end{sphinxVerbatim}

\sphinxAtStartPar
An optional keyword argument ‘msg’ can be provided when assertRaises
is used as a context object.

\sphinxAtStartPar
The context manager keeps a reference to the exception as
the ‘exception’ attribute. This allows you to inspect the
exception after the assertion:

\begin{sphinxVerbatim}[commandchars=\\\{\}]
\PYG{k}{with} \PYG{n+nb+bp}{self}\PYG{o}{.}\PYG{n}{assertRaises}\PYG{p}{(}\PYG{n}{SomeException}\PYG{p}{)} \PYG{k}{as} \PYG{n}{cm}\PYG{p}{:}
    \PYG{n}{do\PYGZus{}something}\PYG{p}{(}\PYG{p}{)}
\PYG{n}{the\PYGZus{}exception} \PYG{o}{=} \PYG{n}{cm}\PYG{o}{.}\PYG{n}{exception}
\PYG{n+nb+bp}{self}\PYG{o}{.}\PYG{n}{assertEqual}\PYG{p}{(}\PYG{n}{the\PYGZus{}exception}\PYG{o}{.}\PYG{n}{error\PYGZus{}code}\PYG{p}{,} \PYG{l+m+mi}{3}\PYG{p}{)}
\end{sphinxVerbatim}

\end{fulllineitems}

\index{assertRaisesRegex() (tests.test\_unit.test\_sqlite method)@\spxentry{assertRaisesRegex()}\spxextra{tests.test\_unit.test\_sqlite method}}

\begin{fulllineitems}
\phantomsection\label{\detokenize{_autosummary/tests.test_unit.test_sqlite:tests.test_unit.test_sqlite.assertRaisesRegex}}
\pysigstartsignatures
\pysiglinewithargsret{\sphinxbfcode{\sphinxupquote{assertRaisesRegex}}}{\emph{\DUrole{n}{expected\_exception}}, \emph{\DUrole{n}{expected\_regex}}, \emph{\DUrole{o}{*}\DUrole{n}{args}}, \emph{\DUrole{o}{**}\DUrole{n}{kwargs}}}{}
\pysigstopsignatures
\sphinxAtStartPar
Asserts that the message in a raised exception matches a regex.
\begin{description}
\sphinxlineitem{Args:}
\sphinxAtStartPar
expected\_exception: Exception class expected to be raised.
expected\_regex: Regex (re.Pattern object or string) expected
\begin{quote}

\sphinxAtStartPar
to be found in error message.
\end{quote}

\sphinxAtStartPar
args: Function to be called and extra positional args.
kwargs: Extra kwargs.
msg: Optional message used in case of failure. Can only be used
\begin{quote}

\sphinxAtStartPar
when assertRaisesRegex is used as a context manager.
\end{quote}

\end{description}

\end{fulllineitems}

\index{assertRegex() (tests.test\_unit.test\_sqlite method)@\spxentry{assertRegex()}\spxextra{tests.test\_unit.test\_sqlite method}}

\begin{fulllineitems}
\phantomsection\label{\detokenize{_autosummary/tests.test_unit.test_sqlite:tests.test_unit.test_sqlite.assertRegex}}
\pysigstartsignatures
\pysiglinewithargsret{\sphinxbfcode{\sphinxupquote{assertRegex}}}{\emph{\DUrole{n}{text}}, \emph{\DUrole{n}{expected\_regex}}, \emph{\DUrole{n}{msg}\DUrole{o}{=}\DUrole{default_value}{None}}}{}
\pysigstopsignatures
\sphinxAtStartPar
Fail the test unless the text matches the regular expression.

\end{fulllineitems}

\index{assertSequenceEqual() (tests.test\_unit.test\_sqlite method)@\spxentry{assertSequenceEqual()}\spxextra{tests.test\_unit.test\_sqlite method}}

\begin{fulllineitems}
\phantomsection\label{\detokenize{_autosummary/tests.test_unit.test_sqlite:tests.test_unit.test_sqlite.assertSequenceEqual}}
\pysigstartsignatures
\pysiglinewithargsret{\sphinxbfcode{\sphinxupquote{assertSequenceEqual}}}{\emph{\DUrole{n}{seq1}}, \emph{\DUrole{n}{seq2}}, \emph{\DUrole{n}{msg}\DUrole{o}{=}\DUrole{default_value}{None}}, \emph{\DUrole{n}{seq\_type}\DUrole{o}{=}\DUrole{default_value}{None}}}{}
\pysigstopsignatures
\sphinxAtStartPar
An equality assertion for ordered sequences (like lists and tuples).

\sphinxAtStartPar
For the purposes of this function, a valid ordered sequence type is one
which can be indexed, has a length, and has an equality operator.
\begin{description}
\sphinxlineitem{Args:}
\sphinxAtStartPar
seq1: The first sequence to compare.
seq2: The second sequence to compare.
seq\_type: The expected datatype of the sequences, or None if no
\begin{quote}

\sphinxAtStartPar
datatype should be enforced.
\end{quote}
\begin{description}
\sphinxlineitem{msg: Optional message to use on failure instead of a list of}
\sphinxAtStartPar
differences.

\end{description}

\end{description}

\end{fulllineitems}

\index{assertSetEqual() (tests.test\_unit.test\_sqlite method)@\spxentry{assertSetEqual()}\spxextra{tests.test\_unit.test\_sqlite method}}

\begin{fulllineitems}
\phantomsection\label{\detokenize{_autosummary/tests.test_unit.test_sqlite:tests.test_unit.test_sqlite.assertSetEqual}}
\pysigstartsignatures
\pysiglinewithargsret{\sphinxbfcode{\sphinxupquote{assertSetEqual}}}{\emph{\DUrole{n}{set1}}, \emph{\DUrole{n}{set2}}, \emph{\DUrole{n}{msg}\DUrole{o}{=}\DUrole{default_value}{None}}}{}
\pysigstopsignatures
\sphinxAtStartPar
A set\sphinxhyphen{}specific equality assertion.
\begin{description}
\sphinxlineitem{Args:}
\sphinxAtStartPar
set1: The first set to compare.
set2: The second set to compare.
msg: Optional message to use on failure instead of a list of
\begin{quote}

\sphinxAtStartPar
differences.
\end{quote}

\end{description}

\sphinxAtStartPar
assertSetEqual uses ducktyping to support different types of sets, and
is optimized for sets specifically (parameters must support a
difference method).

\end{fulllineitems}

\index{assertTrue() (tests.test\_unit.test\_sqlite method)@\spxentry{assertTrue()}\spxextra{tests.test\_unit.test\_sqlite method}}

\begin{fulllineitems}
\phantomsection\label{\detokenize{_autosummary/tests.test_unit.test_sqlite:tests.test_unit.test_sqlite.assertTrue}}
\pysigstartsignatures
\pysiglinewithargsret{\sphinxbfcode{\sphinxupquote{assertTrue}}}{\emph{\DUrole{n}{expr}}, \emph{\DUrole{n}{msg}\DUrole{o}{=}\DUrole{default_value}{None}}}{}
\pysigstopsignatures
\sphinxAtStartPar
Check that the expression is true.

\end{fulllineitems}

\index{assertTupleEqual() (tests.test\_unit.test\_sqlite method)@\spxentry{assertTupleEqual()}\spxextra{tests.test\_unit.test\_sqlite method}}

\begin{fulllineitems}
\phantomsection\label{\detokenize{_autosummary/tests.test_unit.test_sqlite:tests.test_unit.test_sqlite.assertTupleEqual}}
\pysigstartsignatures
\pysiglinewithargsret{\sphinxbfcode{\sphinxupquote{assertTupleEqual}}}{\emph{\DUrole{n}{tuple1}}, \emph{\DUrole{n}{tuple2}}, \emph{\DUrole{n}{msg}\DUrole{o}{=}\DUrole{default_value}{None}}}{}
\pysigstopsignatures
\sphinxAtStartPar
A tuple\sphinxhyphen{}specific equality assertion.
\begin{description}
\sphinxlineitem{Args:}
\sphinxAtStartPar
tuple1: The first tuple to compare.
tuple2: The second tuple to compare.
msg: Optional message to use on failure instead of a list of
\begin{quote}

\sphinxAtStartPar
differences.
\end{quote}

\end{description}

\end{fulllineitems}

\index{assertWarns() (tests.test\_unit.test\_sqlite method)@\spxentry{assertWarns()}\spxextra{tests.test\_unit.test\_sqlite method}}

\begin{fulllineitems}
\phantomsection\label{\detokenize{_autosummary/tests.test_unit.test_sqlite:tests.test_unit.test_sqlite.assertWarns}}
\pysigstartsignatures
\pysiglinewithargsret{\sphinxbfcode{\sphinxupquote{assertWarns}}}{\emph{\DUrole{n}{expected\_warning}}, \emph{\DUrole{o}{*}\DUrole{n}{args}}, \emph{\DUrole{o}{**}\DUrole{n}{kwargs}}}{}
\pysigstopsignatures
\sphinxAtStartPar
Fail unless a warning of class warnClass is triggered
by the callable when invoked with specified positional and
keyword arguments.  If a different type of warning is
triggered, it will not be handled: depending on the other
warning filtering rules in effect, it might be silenced, printed
out, or raised as an exception.

\sphinxAtStartPar
If called with the callable and arguments omitted, will return a
context object used like this:

\begin{sphinxVerbatim}[commandchars=\\\{\}]
\PYG{k}{with} \PYG{n+nb+bp}{self}\PYG{o}{.}\PYG{n}{assertWarns}\PYG{p}{(}\PYG{n}{SomeWarning}\PYG{p}{)}\PYG{p}{:}
    \PYG{n}{do\PYGZus{}something}\PYG{p}{(}\PYG{p}{)}
\end{sphinxVerbatim}

\sphinxAtStartPar
An optional keyword argument ‘msg’ can be provided when assertWarns
is used as a context object.

\sphinxAtStartPar
The context manager keeps a reference to the first matching
warning as the ‘warning’ attribute; similarly, the ‘filename’
and ‘lineno’ attributes give you information about the line
of Python code from which the warning was triggered.
This allows you to inspect the warning after the assertion:

\begin{sphinxVerbatim}[commandchars=\\\{\}]
\PYG{k}{with} \PYG{n+nb+bp}{self}\PYG{o}{.}\PYG{n}{assertWarns}\PYG{p}{(}\PYG{n}{SomeWarning}\PYG{p}{)} \PYG{k}{as} \PYG{n}{cm}\PYG{p}{:}
    \PYG{n}{do\PYGZus{}something}\PYG{p}{(}\PYG{p}{)}
\PYG{n}{the\PYGZus{}warning} \PYG{o}{=} \PYG{n}{cm}\PYG{o}{.}\PYG{n}{warning}
\PYG{n+nb+bp}{self}\PYG{o}{.}\PYG{n}{assertEqual}\PYG{p}{(}\PYG{n}{the\PYGZus{}warning}\PYG{o}{.}\PYG{n}{some\PYGZus{}attribute}\PYG{p}{,} \PYG{l+m+mi}{147}\PYG{p}{)}
\end{sphinxVerbatim}

\end{fulllineitems}

\index{assertWarnsRegex() (tests.test\_unit.test\_sqlite method)@\spxentry{assertWarnsRegex()}\spxextra{tests.test\_unit.test\_sqlite method}}

\begin{fulllineitems}
\phantomsection\label{\detokenize{_autosummary/tests.test_unit.test_sqlite:tests.test_unit.test_sqlite.assertWarnsRegex}}
\pysigstartsignatures
\pysiglinewithargsret{\sphinxbfcode{\sphinxupquote{assertWarnsRegex}}}{\emph{\DUrole{n}{expected\_warning}}, \emph{\DUrole{n}{expected\_regex}}, \emph{\DUrole{o}{*}\DUrole{n}{args}}, \emph{\DUrole{o}{**}\DUrole{n}{kwargs}}}{}
\pysigstopsignatures
\sphinxAtStartPar
Asserts that the message in a triggered warning matches a regexp.
Basic functioning is similar to assertWarns() with the addition
that only warnings whose messages also match the regular expression
are considered successful matches.
\begin{description}
\sphinxlineitem{Args:}
\sphinxAtStartPar
expected\_warning: Warning class expected to be triggered.
expected\_regex: Regex (re.Pattern object or string) expected
\begin{quote}

\sphinxAtStartPar
to be found in error message.
\end{quote}

\sphinxAtStartPar
args: Function to be called and extra positional args.
kwargs: Extra kwargs.
msg: Optional message used in case of failure. Can only be used
\begin{quote}

\sphinxAtStartPar
when assertWarnsRegex is used as a context manager.
\end{quote}

\end{description}

\end{fulllineitems}

\index{debug() (tests.test\_unit.test\_sqlite method)@\spxentry{debug()}\spxextra{tests.test\_unit.test\_sqlite method}}

\begin{fulllineitems}
\phantomsection\label{\detokenize{_autosummary/tests.test_unit.test_sqlite:tests.test_unit.test_sqlite.debug}}
\pysigstartsignatures
\pysiglinewithargsret{\sphinxbfcode{\sphinxupquote{debug}}}{}{}
\pysigstopsignatures
\sphinxAtStartPar
Run the test without collecting errors in a TestResult

\end{fulllineitems}

\index{doClassCleanups() (tests.test\_unit.test\_sqlite class method)@\spxentry{doClassCleanups()}\spxextra{tests.test\_unit.test\_sqlite class method}}

\begin{fulllineitems}
\phantomsection\label{\detokenize{_autosummary/tests.test_unit.test_sqlite:tests.test_unit.test_sqlite.doClassCleanups}}
\pysigstartsignatures
\pysiglinewithargsret{\sphinxbfcode{\sphinxupquote{classmethod\DUrole{w}{  }}}\sphinxbfcode{\sphinxupquote{doClassCleanups}}}{}{}
\pysigstopsignatures
\sphinxAtStartPar
Execute all class cleanup functions. Normally called for you after
tearDownClass.

\end{fulllineitems}

\index{doCleanups() (tests.test\_unit.test\_sqlite method)@\spxentry{doCleanups()}\spxextra{tests.test\_unit.test\_sqlite method}}

\begin{fulllineitems}
\phantomsection\label{\detokenize{_autosummary/tests.test_unit.test_sqlite:tests.test_unit.test_sqlite.doCleanups}}
\pysigstartsignatures
\pysiglinewithargsret{\sphinxbfcode{\sphinxupquote{doCleanups}}}{}{}
\pysigstopsignatures
\sphinxAtStartPar
Execute all cleanup functions. Normally called for you after
tearDown.

\end{fulllineitems}

\index{enterClassContext() (tests.test\_unit.test\_sqlite class method)@\spxentry{enterClassContext()}\spxextra{tests.test\_unit.test\_sqlite class method}}

\begin{fulllineitems}
\phantomsection\label{\detokenize{_autosummary/tests.test_unit.test_sqlite:tests.test_unit.test_sqlite.enterClassContext}}
\pysigstartsignatures
\pysiglinewithargsret{\sphinxbfcode{\sphinxupquote{classmethod\DUrole{w}{  }}}\sphinxbfcode{\sphinxupquote{enterClassContext}}}{\emph{\DUrole{n}{cm}}}{}
\pysigstopsignatures
\sphinxAtStartPar
Same as enterContext, but class\sphinxhyphen{}wide.

\end{fulllineitems}

\index{enterContext() (tests.test\_unit.test\_sqlite method)@\spxentry{enterContext()}\spxextra{tests.test\_unit.test\_sqlite method}}

\begin{fulllineitems}
\phantomsection\label{\detokenize{_autosummary/tests.test_unit.test_sqlite:tests.test_unit.test_sqlite.enterContext}}
\pysigstartsignatures
\pysiglinewithargsret{\sphinxbfcode{\sphinxupquote{enterContext}}}{\emph{\DUrole{n}{cm}}}{}
\pysigstopsignatures
\sphinxAtStartPar
Enters the supplied context manager.

\sphinxAtStartPar
If successful, also adds its \_\_exit\_\_ method as a cleanup
function and returns the result of the \_\_enter\_\_ method.

\end{fulllineitems}

\index{fail() (tests.test\_unit.test\_sqlite method)@\spxentry{fail()}\spxextra{tests.test\_unit.test\_sqlite method}}

\begin{fulllineitems}
\phantomsection\label{\detokenize{_autosummary/tests.test_unit.test_sqlite:tests.test_unit.test_sqlite.fail}}
\pysigstartsignatures
\pysiglinewithargsret{\sphinxbfcode{\sphinxupquote{fail}}}{\emph{\DUrole{n}{msg}\DUrole{o}{=}\DUrole{default_value}{None}}}{}
\pysigstopsignatures
\sphinxAtStartPar
Fail immediately, with the given message.

\end{fulllineitems}

\index{failureException (tests.test\_unit.test\_sqlite attribute)@\spxentry{failureException}\spxextra{tests.test\_unit.test\_sqlite attribute}}

\begin{fulllineitems}
\phantomsection\label{\detokenize{_autosummary/tests.test_unit.test_sqlite:tests.test_unit.test_sqlite.failureException}}
\pysigstartsignatures
\pysigline{\sphinxbfcode{\sphinxupquote{failureException}}}
\pysigstopsignatures
\sphinxAtStartPar
alias of \sphinxcode{\sphinxupquote{AssertionError}}

\end{fulllineitems}

\index{setUp() (tests.test\_unit.test\_sqlite method)@\spxentry{setUp()}\spxextra{tests.test\_unit.test\_sqlite method}}

\begin{fulllineitems}
\phantomsection\label{\detokenize{_autosummary/tests.test_unit.test_sqlite:tests.test_unit.test_sqlite.setUp}}
\pysigstartsignatures
\pysiglinewithargsret{\sphinxbfcode{\sphinxupquote{setUp}}}{}{}
\pysigstopsignatures
\sphinxAtStartPar
Execute everytime before running a test of this class.

\end{fulllineitems}

\index{setUpClass() (tests.test\_unit.test\_sqlite class method)@\spxentry{setUpClass()}\spxextra{tests.test\_unit.test\_sqlite class method}}

\begin{fulllineitems}
\phantomsection\label{\detokenize{_autosummary/tests.test_unit.test_sqlite:tests.test_unit.test_sqlite.setUpClass}}
\pysigstartsignatures
\pysiglinewithargsret{\sphinxbfcode{\sphinxupquote{classmethod\DUrole{w}{  }}}\sphinxbfcode{\sphinxupquote{setUpClass}}}{}{}
\pysigstopsignatures
\sphinxAtStartPar
Execute before running tests of this class.

\end{fulllineitems}

\index{shortDescription() (tests.test\_unit.test\_sqlite method)@\spxentry{shortDescription()}\spxextra{tests.test\_unit.test\_sqlite method}}

\begin{fulllineitems}
\phantomsection\label{\detokenize{_autosummary/tests.test_unit.test_sqlite:tests.test_unit.test_sqlite.shortDescription}}
\pysigstartsignatures
\pysiglinewithargsret{\sphinxbfcode{\sphinxupquote{shortDescription}}}{}{}
\pysigstopsignatures
\sphinxAtStartPar
Returns a one\sphinxhyphen{}line description of the test, or None if no
description has been provided.

\sphinxAtStartPar
The default implementation of this method returns the first line of
the specified test method’s docstring.

\end{fulllineitems}

\index{skipTest() (tests.test\_unit.test\_sqlite method)@\spxentry{skipTest()}\spxextra{tests.test\_unit.test\_sqlite method}}

\begin{fulllineitems}
\phantomsection\label{\detokenize{_autosummary/tests.test_unit.test_sqlite:tests.test_unit.test_sqlite.skipTest}}
\pysigstartsignatures
\pysiglinewithargsret{\sphinxbfcode{\sphinxupquote{skipTest}}}{\emph{\DUrole{n}{reason}}}{}
\pysigstopsignatures
\sphinxAtStartPar
Skip this test.

\end{fulllineitems}

\index{subTest() (tests.test\_unit.test\_sqlite method)@\spxentry{subTest()}\spxextra{tests.test\_unit.test\_sqlite method}}

\begin{fulllineitems}
\phantomsection\label{\detokenize{_autosummary/tests.test_unit.test_sqlite:tests.test_unit.test_sqlite.subTest}}
\pysigstartsignatures
\pysiglinewithargsret{\sphinxbfcode{\sphinxupquote{subTest}}}{\emph{\DUrole{n}{msg=\textless{}object object\textgreater{}}}, \emph{\DUrole{n}{**params}}}{}
\pysigstopsignatures
\sphinxAtStartPar
Return a context manager that will return the enclosed block
of code in a subtest identified by the optional message and
keyword parameters.  A failure in the subtest marks the test
case as failed but resumes execution at the end of the enclosed
block, allowing further test code to be executed.

\end{fulllineitems}

\index{tearDown() (tests.test\_unit.test\_sqlite method)@\spxentry{tearDown()}\spxextra{tests.test\_unit.test\_sqlite method}}

\begin{fulllineitems}
\phantomsection\label{\detokenize{_autosummary/tests.test_unit.test_sqlite:tests.test_unit.test_sqlite.tearDown}}
\pysigstartsignatures
\pysiglinewithargsret{\sphinxbfcode{\sphinxupquote{tearDown}}}{}{}
\pysigstopsignatures
\sphinxAtStartPar
Execute everytime after running a test of this class.

\end{fulllineitems}

\index{tearDownClass() (tests.test\_unit.test\_sqlite class method)@\spxentry{tearDownClass()}\spxextra{tests.test\_unit.test\_sqlite class method}}

\begin{fulllineitems}
\phantomsection\label{\detokenize{_autosummary/tests.test_unit.test_sqlite:tests.test_unit.test_sqlite.tearDownClass}}
\pysigstartsignatures
\pysiglinewithargsret{\sphinxbfcode{\sphinxupquote{classmethod\DUrole{w}{  }}}\sphinxbfcode{\sphinxupquote{tearDownClass}}}{}{}
\pysigstopsignatures
\sphinxAtStartPar
Execute after running  all tests of this class.

\sphinxAtStartPar
Try deleting created files during testing process. Pass if deletion is
not possible. The existance of test\sphinxhyphen{}files can be accepted an has no
negative effect on the result of the main program.

\end{fulllineitems}

\index{test\_creation() (tests.test\_unit.test\_sqlite method)@\spxentry{test\_creation()}\spxextra{tests.test\_unit.test\_sqlite method}}

\begin{fulllineitems}
\phantomsection\label{\detokenize{_autosummary/tests.test_unit.test_sqlite:tests.test_unit.test_sqlite.test_creation}}
\pysigstartsignatures
\pysiglinewithargsret{\sphinxbfcode{\sphinxupquote{test\_creation}}}{}{}
\pysigstopsignatures
\sphinxAtStartPar
Test creation of empty sqlite database.

\end{fulllineitems}

\index{test\_table\_from\_csv() (tests.test\_unit.test\_sqlite method)@\spxentry{test\_table\_from\_csv()}\spxextra{tests.test\_unit.test\_sqlite method}}

\begin{fulllineitems}
\phantomsection\label{\detokenize{_autosummary/tests.test_unit.test_sqlite:tests.test_unit.test_sqlite.test_table_from_csv}}
\pysigstartsignatures
\pysiglinewithargsret{\sphinxbfcode{\sphinxupquote{test\_table\_from\_csv}}}{}{}
\pysigstopsignatures
\sphinxAtStartPar
Test import of csv files to SQLite using subprocess.

\sphinxAtStartPar
Test csv2sql\_directly function inside the module datafunctions.py.
This Test is expected to fail, therefore a @expectedFailure decorator
was placed. In case of success, a log\sphinxhyphen{}file entry gets printed during
the setup process. Success would suggest, that the system is able to
handle extremely large csv files, due to the possibility to directly
write those files to the SQLite database.

\end{fulllineitems}

\index{test\_table\_from\_pandas() (tests.test\_unit.test\_sqlite method)@\spxentry{test\_table\_from\_pandas()}\spxextra{tests.test\_unit.test\_sqlite method}}

\begin{fulllineitems}
\phantomsection\label{\detokenize{_autosummary/tests.test_unit.test_sqlite:tests.test_unit.test_sqlite.test_table_from_pandas}}
\pysigstartsignatures
\pysiglinewithargsret{\sphinxbfcode{\sphinxupquote{test\_table\_from\_pandas}}}{}{}
\pysigstopsignatures
\sphinxAtStartPar
Test import of csv files to SQLite using pandas.

\sphinxAtStartPar
Test csv2sql\_pandas function of module datafunction.py

\end{fulllineitems}


\end{fulllineitems}



\renewcommand{\indexname}{Python Module Index}
\begin{sphinxtheindex}
\let\bigletter\sphinxstyleindexlettergroup
\bigletter{f}
\item\relax\sphinxstyleindexentry{functionfinder}\sphinxstyleindexpageref{_autosummary/functionfinder:\detokenize{module-functionfinder}}
\item\relax\sphinxstyleindexentry{functionfinder.classes}\sphinxstyleindexpageref{_autosummary/functionfinder.classes:\detokenize{module-functionfinder.classes}}
\item\relax\sphinxstyleindexentry{functionfinder.config}\sphinxstyleindexpageref{_autosummary/functionfinder.config:\detokenize{module-functionfinder.config}}
\item\relax\sphinxstyleindexentry{functionfinder.datafunctions}\sphinxstyleindexpageref{_autosummary/functionfinder.datafunctions:\detokenize{module-functionfinder.datafunctions}}
\item\relax\sphinxstyleindexentry{functionfinder.exceptions}\sphinxstyleindexpageref{_autosummary/functionfinder.exceptions:\detokenize{module-functionfinder.exceptions}}
\item\relax\sphinxstyleindexentry{functionfinder.ffrunner}\sphinxstyleindexpageref{_autosummary/functionfinder.ffrunner:\detokenize{module-functionfinder.ffrunner}}
\item\relax\sphinxstyleindexentry{functionfinder.log}\sphinxstyleindexpageref{_autosummary/functionfinder.log:\detokenize{module-functionfinder.log}}
\item\relax\sphinxstyleindexentry{functionfinder.setuplog}\sphinxstyleindexpageref{_autosummary/functionfinder.setuplog:\detokenize{module-functionfinder.setuplog}}
\indexspace
\bigletter{t}
\item\relax\sphinxstyleindexentry{tests}\sphinxstyleindexpageref{_autosummary/tests:\detokenize{module-tests}}
\item\relax\sphinxstyleindexentry{tests.test\_unit}\sphinxstyleindexpageref{_autosummary/tests.test_unit:\detokenize{module-tests.test_unit}}
\end{sphinxtheindex}

\renewcommand{\indexname}{Index}
\printindex
\end{document}